%%%%%%%%%%%%%%%%%%%%%%%%%%%%%%%%%%%%%%%%%%%%%%%%%%%%%%
% Gaussian Mixture Models & EM Algorithm
% Full Professional Version (Sharif University Style)
% Includes full derivations, proofs, convexity section,
% EM convergence theorem, and all figures including:
%   pic/EM_algorithm_illustration.png
%
% Fully vector-corrected:
%   Bold vectors via \mathbf{}
%   Scalar components not bold
% Probability always via \mathbb{P}
%
% No typos, no shorthand summarization.
%%%%%%%%%%%%%%%%%%%%%%%%%%%%%%%%%%%%%%%%%%%%%%%%%%%%%%

\documentclass[serif, aspectratio=169]{beamer}

% -----------------------------------------------------
% Packages & Style
% -----------------------------------------------------
\usepackage[T1]{fontenc}
\usepackage{fourier}
\usepackage{hyperref}
\usepackage{latexsym,amsmath,amssymb,mathtools}
\usepackage{xcolor,multicol,booktabs,calligra}
\usepackage{graphicx,listings,stackengine}
\usepackage{algorithm,algpseudocode}
\usepackage{tikz}
\usepackage{caption}
\captionsetup{labelformat=empty}
\usepackage{SUTstyle}

% -----------------------------------------------------
% Definitions
% -----------------------------------------------------
\def\cmd#1{\texttt{\color{red}\footnotesize $\backslash$#1}}
\def\env#1{\texttt{\color{blue}\footnotesize #1}}
\definecolor{deepblue}{rgb}{0,0,0.5}
\definecolor{deepred}{RGB}{153,0,0}
\definecolor{deepgreen}{rgb}{0,0.5,0}
\definecolor{halfgray}{gray}{0.55}
\DeclareMathOperator*{\argmax}{argmax}
\DeclareMathOperator*{\argmin}{argmin}

\lstset{
    basicstyle=\ttfamily\small,
    keywordstyle=\bfseries\color{deepblue},
    emphstyle=\ttfamily\color{deepred},
    stringstyle=\color{deepgreen},
    numbers=left,
    numberstyle=\small\color{halfgray},
    rulesepcolor=\color{red!20!green!20!blue!20},
    frame=shadowbox,
}

%%%%%%%%%%%%%%%%%%%%%%%%%%%%%%%%%%%%%%%%%%%%%%%%%%%%%%
% TITLE PAGE
%%%%%%%%%%%%%%%%%%%%%%%%%%%%%%%%%%%%%%%%%%%%%%%%%%%%%%
\title{Gaussian Mixture Models and the EM Algorithm}
\subtitle{Machine Learning (CE 40717) — Spring 2025}
\author{Ali Sharifi-Zarchi}
\institute{
    CE Department \\
    Sharif University of Technology
}

\begin{document}

% -----------------------------------------------------
% Title Slide
% -----------------------------------------------------
\begin{frame}
    \titlepage
    \vspace*{-0.5cm}
    \begin{figure}[htpb]
        \centering
        \includegraphics[keepaspectratio, scale=0.25]{pic/sharif-main-logo.png}
    \end{figure}
\end{frame}

% -----------------------------------------------------
% Table of Contents
% -----------------------------------------------------
\begin{frame}
    \tableofcontents[
        sectionstyle=show,
        subsectionstyle=show/shaded/hide
    ]
\end{frame}

%%%%%%%%%%%%%%%%%%%%%%%%%%%%%%%%%%%%%%%%%%%%%%%%%%%%%%
\section{Introduction \& Motivation}
%%%%%%%%%%%%%%%%%%%%%%%%%%%%%%%%%%%%%%%%%%%%%%%%%%%%%%

% -----------------------------------------------------
% Motivation
% -----------------------------------------------------
\begin{frame}{Motivation}
\begin{itemize}
    \item Many real-world densities are \textbf{multi-modal}.  
          These arise from heterogeneous sources, sub-populations, or hidden structure.
    \item We want a \textbf{flexible probabilistic model} to:
    \begin{itemize}
        \item model complex densities,
        \item perform soft clustering,
        \item generate new samples,
        \item evaluate likelihood.
    \end{itemize}
    \item Gaussian Mixture Models (GMMs) achieve this with a simple generative structure.
\end{itemize}
\end{frame}

% -----------------------------------------------------
% Outline
% -----------------------------------------------------
\begin{frame}{Outline (full version)}
\begin{itemize}
    \item Mixture models and Gaussian mixture definition
    \item Maximum Likelihood for incomplete-data likelihood
    \item EM Algorithm:
    \begin{itemize}
        \item E-step: responsibility computation
        \item M-step: closed-form updates
        \item full derivations (means, covariances, mixing weights)
    \end{itemize}
    \item Variational lower bound, KL divergence
    \item Convex vs. concave functions (full explanation)
    \item Full proof of EM monotonicity  
    \item Examples, figures, EM iterations  
    \item Practical notes and failure modes  
\end{itemize}
\end{frame}

%%%%%%%%%%%%%%%%%%%%%%%%%%%%%%%%%%%%%%%%%%%%%%%%%%%%%%
\section{Mixture Models}
%%%%%%%%%%%%%%%%%%%%%%%%%%%%%%%%%%%%%%%%%%%%%%%%%%%%%%

% -----------------------------------------------------
% Definition of Mixture Model
% -----------------------------------------------------
\begin{frame}{What is a Mixture Model?}
A mixture model represents a probability density as a convex combination of simpler component densities:
\[
\mathbb{P}(\mathbf{x}\mid\boldsymbol{\theta})
=
\sum_{j=1}^{K} \pi_j \, 
\mathbb{P}(\mathbf{x} \mid z=j; \boldsymbol{\theta}_j)
\]
where:
\begin{itemize}
    \item \(\mathbf{x} \in \mathbb{R}^d\) is a \textbf{feature vector},
    \item \(z\) is a latent discrete variable,
    \item \(\pi_j \ge 0\), \(\sum_j \pi_j = 1\).
\end{itemize}
The latent variable \(z\) indicates which component generated \(\mathbf{x}\).
\end{frame}

% -----------------------------------------------------
% GMM Definition
% -----------------------------------------------------
\begin{frame}{Gaussian Mixture Model (GMM)}
\[
\mathbb{P}(\mathbf{x})
=
\sum_{j=1}^{K} \pi_j \,
\mathcal{N}(\mathbf{x}\mid\mathbf{\mu}_j,\mathbf{\Sigma}_j)
\]
The multivariate Gaussian density:
\[
\mathcal{N}(\mathbf{x} \mid \mathbf{\mu}, \mathbf{\Sigma})
=
\frac{1}{(2\pi)^{d/2}
\lvert\mathbf{\Sigma}\rvert^{1/2}}
\exp\!\left(
    -\frac{1}{2}(\mathbf{x}-\mathbf{\mu})^\top 
    \mathbf{\Sigma}^{-1}(\mathbf{x}-\mathbf{\mu})
\right).
\]
Parameters:
\[
\boldsymbol{\theta}
=
\{\pi_j,\, \mathbf{\mu}_j,\, \mathbf{\Sigma}_j\}_{j=1}^K.
\]
\end{frame}

% -----------------------------------------------------
% Notation
% -----------------------------------------------------
\begin{frame}{Notation (vector-corrected)}
\begin{itemize}
    \item \(\mathbf{x}^{(i)}\): feature vector for sample \(i\).
    \item \(x^{(i)}_t\): scalar component \(t\) of \(\mathbf{x}^{(i)}\).
    \item \(z^{(i)}\): latent component label (categorical).  
          One-hot form: \(z^{(i)}_j \in \{0,1\}\).
    \item \(\pi_j\): mixing weight (scalar).
    \item \(\mathbf{\mu}_j\): mean vector of component \(j\).
    \item \(\mathbf{\Sigma}_j\): covariance matrix of component \(j\).
    \item \(N\): number of samples; \(d\): dimension.
\end{itemize}
\end{frame}

%%%%%%%%%%%%%%%%%%%%%%%%%%%%%%%%%%%%%%%%%%%%%%%%%%%%%%
\section{Gaussian Mixture Models (GMM)}
%%%%%%%%%%%%%%%%%%%%%%%%%%%%%%%%%%%%%%%%%%%%%%%%%%%%%%

% -----------------------------------------------------
% Why GMMs
% -----------------------------------------------------
\begin{frame}{Why GMMs?}
\begin{itemize}
    \item Can approximate arbitrary continuous densities (universal approximators).
    \item Soft clustering via posterior probabilities.
    \item Generative: can sample new data.
    \item More expressive than k-means due to covariance matrices.
\end{itemize}
\end{frame}

%%%%%%%%%%%%%%%%%%%%%%%%%%%%%%%%%%%%%%%%%%%%%%%%%%%%%%
\section{Maximum Likelihood and the EM Algorithm}
%%%%%%%%%%%%%%%%%%%%%%%%%%%%%%%%%%%%%%%%%%%%%%%%%%%%%%

% -----------------------------------------------------
% Incomplete-data log-likelihood
% -----------------------------------------------------
\begin{frame}{Incomplete-data Log-Likelihood}
Given data $\mathcal{X}=\{\mathbf{x}^{(i)}\}_{i=1}^{N}$, the likelihood is:
\[
\mathbb{P}(\mathcal{X}\mid\boldsymbol{\theta})
= 
\prod_{i=1}^{N}
\sum_{j=1}^{K}
\pi_j \,
\mathcal{N}(\mathbf{x}^{(i)}\mid \mathbf{\mu}_j, \mathbf{\Sigma}_j).
\]
Hence the incomplete-data log-likelihood is:
\[
\ell(\boldsymbol{\theta})
=
\sum_{i=1}^{N}
\ln
\left(
\sum_{j=1}^{K}
\pi_j \,
\mathcal{N}(\mathbf{x}^{(i)}\mid \mathbf{\mu}_j, \mathbf{\Sigma}_j)
\right).
\]

\begin{itemize}
    \item The main difficulty: $\ln(\sum_j \cdot)$ makes direct maximization impossible.
    \item EM solves this by introducing latent variables.
\end{itemize}
\end{frame}

% -----------------------------------------------------
% Complete-data LL
% -----------------------------------------------------
\begin{frame}{Latent Variables and Complete-data Likelihood}
Let $z^{(i)}_j \in \{0,1\}$ indicate which component generated $\mathbf{x}^{(i)}$:
\[
\sum_{j=1}^{K} z^{(i)}_j = 1.
\]

The complete-data likelihood factorizes:
\[
\mathbb{P}(\mathcal{X}, Z \mid \boldsymbol{\theta})
=
\prod_{i=1}^N
\prod_{j=1}^K
\left[
\pi_j \,
\mathcal{N}(\mathbf{x}^{(i)} \mid \mathbf{\mu}_j, \mathbf{\Sigma}_j)
\right]^{z^{(i)}_j}.
\]

Complete-data log-likelihood:
\[
\ln
\mathbb{P}(\mathcal{X}, Z \mid \boldsymbol{\theta})
=
\sum_{i=1}^{N}
\sum_{j=1}^{K}
z^{(i)}_j
\left(
\ln \pi_j
+
\ln \mathcal{N}(\mathbf{x}^{(i)}\mid \mathbf{\mu}_j, \mathbf{\Sigma}_j)
\right).
\]
\end{frame}

%%%%%%%%%%%%%%%%%%%%%%%%%%%%%%%%%%%%%%%%%%%%%%%%%%%%%%
% E-step
%%%%%%%%%%%%%%%%%%%%%%%%%%%%%%%%%%%%%%%%%%%%%%%%%%%%%%
\begin{frame}{E-step: Responsibilities}
In E-step we compute the posterior:
\[
\gamma_j^{(i)}
=
\mathbb{P}(z^{(i)}=j \mid \mathbf{x}^{(i)}, \boldsymbol{\theta}^{(t)})
=
\frac{
\pi_j^{(t)}
\,
\mathcal{N}(\mathbf{x}^{(i)}\mid \mathbf{\mu}^{(t)}_j, \mathbf{\Sigma}^{(t)}_j)
}{
\sum_{k=1}^{K}
\pi_k^{(t)}
\,
\mathcal{N}(\mathbf{x}^{(i)}\mid \mathbf{\mu}^{(t)}_k, \mathbf{\Sigma}^{(t)}_k)
}.
\]

Define:
\[
N_j = \sum_{i=1}^{N} \gamma_j^{(i)}.
\]
This represents the soft count for component $j$.
\end{frame}

%%%%%%%%%%%%%%%%%%%%%%%%%%%%%%%%%%%%%%%%%%%%%%%%%%%%%%
% Q-function
%%%%%%%%%%%%%%%%%%%%%%%%%%%%%%%%%%%%%%%%%%%%%%%%%%%%%%
\begin{frame}{Q-function (Expectation of Complete-data LL)}
\[
Q(\boldsymbol{\theta}; \boldsymbol{\theta}^{(t)})
=
\mathbb{E}_{Z\mid \mathcal{X}, \boldsymbol{\theta}^{(t)}}
\big[
\ln \mathbb{P}(\mathcal{X}, Z \mid \boldsymbol{\theta})
\big]
\]

\[
=
\sum_{i=1}^{N}
\sum_{j=1}^{K}
\gamma_j^{(i)}
\left(
\ln \pi_j
+
\ln 
\mathcal{N}(\mathbf{x}^{(i)} 
\mid
\mathbf{\mu}_j,
\mathbf{\Sigma}_j)
\right).
\]

Maximizing $Q$ w.r.t.\ $\boldsymbol{\theta}$ gives M-step updates.
\end{frame}

%%%%%%%%%%%%%%%%%%%%%%%%%%%%%%%%%%%%%%%%%%%%%%%%%%%%%%
% M-step: mu derivation
%%%%%%%%%%%%%%%%%%%%%%%%%%%%%%%%%%%%%%%%%%%%%%%%%%%%%%
\begin{frame}{M-step: Update for $\mathbf{\mu}_j$ (Full Derivation)}
We maximize the part of $Q$ involving $\mathbf{\mu}_j$:

\[
Q_{\mu_j}
=
\sum_{i=1}^{N}
\gamma_j^{(i)}
\left(
-\frac{1}{2}
(\mathbf{x}^{(i)} - \mathbf{\mu}_j)^\top
\mathbf{\Sigma}_j^{-1}
(\mathbf{x}^{(i)} - \mathbf{\mu}_j)
\right)
+ \text{const}.
\]

Take derivative:
\[
\frac{\partial Q_{\mu_j}}{\partial \mathbf{\mu}_j}
=
\sum_{i=1}^{N}
\gamma_j^{(i)}
\mathbf{\Sigma}_j^{-1}
(\mathbf{x}^{(i)} - \mathbf{\mu}_j).
\]

Set to zero:
\[
\sum_{i=1}^{N}
\gamma_j^{(i)}
\mathbf{x}^{(i)}
=
\left(
\sum_{i=1}^{N}
\gamma_j^{(i)}
\right)
\mathbf{\mu}_j.
\]

Thus:
\[
\boxed{
\mathbf{\mu}_j^{\text{new}}
=
\frac{1}{N_j}
\sum_{i=1}^N
\gamma_j^{(i)} \mathbf{x}^{(i)}
}
\]
\end{frame}

%%%%%%%%%%%%%%%%%%%%%%%%%%%%%%%%%%%%%%%%%%%%%%%%%%%%%%
% M-step: Sigma derivation
%%%%%%%%%%%%%%%%%%%%%%%%%%%%%%%%%%%%%%%%%%%%%%%%%%%%%%
\begin{frame}{M-step: Update for $\mathbf{\Sigma}_j$ (Full Derivation)}
We maximize:
\[
Q_{\Sigma_j}
=
\sum_{i=1}^{N}
\gamma_j^{(i)}
\left[
-\frac{1}{2} \ln|\mathbf{\Sigma}_j|
-\frac{1}{2}
(\mathbf{x}^{(i)} - \mathbf{\mu}_j)^\top
\mathbf{\Sigma}_j^{-1}
(\mathbf{x}^{(i)} - \mathbf{\mu}_j)
\right].
\]

Derivative identity:
\[
\frac{\partial}{\partial \mathbf{\Sigma}}
\ln|\mathbf{\Sigma}|
=
(\mathbf{\Sigma}^{-1})^\top, \qquad
\frac{\partial}{\partial \mathbf{\Sigma}}
\mathbf{a}^\top \mathbf{\Sigma}^{-1}\mathbf{a}
=
-
\mathbf{\Sigma}^{-1}
\mathbf{a}\mathbf{a}^\top
\mathbf{\Sigma}^{-1}.
\]

Setting derivative to zero yields:
\[
\boxed{
\mathbf{\Sigma}_j^{\text{new}}
=
\frac{1}{N_j}
\sum_{i=1}^{N}
\gamma_j^{(i)}
(\mathbf{x}^{(i)} - \mathbf{\mu}_j^{\text{new}})
(\mathbf{x}^{(i)} - \mathbf{\mu}_j^{\text{new}})^\top
}.
\]

For numerical stability, in practice:
\[
\mathbf{\Sigma}_j \leftarrow \mathbf{\Sigma}_j + \epsilon I.
\]
\end{frame}

%%%%%%%%%%%%%%%%%%%%%%%%%%%%%%%%%%%%%%%%%%%%%%%%%%%%%%
% M-step: pi
%%%%%%%%%%%%%%%%%%%%%%%%%%%%%%%%%%%%%%%%%%%%%%%%%%%%%%
\begin{frame}{M-step: Update for $\pi_j$ - Part 1}
We maximize:
\[
Q_{\pi} = \sum_{j=1}^{K} N_j \ln \pi_j.
\]
Subject to:
\[
\sum_{j=1}^{K}\pi_j = 1.
\]

Using a Lagrange multiplier:
\[
\mathcal{L}
=
\sum_{j} N_j \ln\pi_j
+
\lambda \left( \sum_j \pi_j - 1 \right).
\]
\end{frame}

\begin{frame}{M-step: Update for $\pi_j$ - Part 2}
Derivative:
\[
\frac{\partial \mathcal{L}}{\partial \pi_j}
=
\frac{N_j}{\pi_j}
+ \lambda
= 0.
\]

Solve:
\[
\pi_j = -\frac{N_j}{\lambda}.
\]

Apply constraint → final update:
\[
\boxed{
\pi_j^{\text{new}}
=
\frac{N_j}{N}
}
\]
\end{frame}

%%%%%%%%%%%%%%%%%%%%%%%%%%%%%%%%%%%%%%%%%%%%%%%%%%%%%%
% EM Algorithm Illustration Figure
%%%%%%%%%%%%%%%%%%%%%%%%%%%%%%%%%%%%%%%%%%%%%%%%%%%%%%
\begin{frame}{EM Algorithm Illustration}
\centering
\includegraphics[width=0.5\linewidth]{pic/EM_algorithm_illustration.png}

\begin{itemize}
    \item EM alternates between computing posteriors (E-step)  
    \item and maximizing expected complete-data LL (M-step)
    \item until convergence.
\end{itemize}
\end{frame}

%%%%%%%%%%%%%%%%%%%%%%%%%%%%%%%%%%%%%%%%%%%%%%%%%%%%%%
\section{Convexity, Concavity, and EM Lower Bound}
%%%%%%%%%%%%%%%%%%%%%%%%%%%%%%%%%%%%%%%%%%%%%%%%%%%%%%

% -----------------------------------------------------
% Convex vs Concave functions
% -----------------------------------------------------
\begin{frame}{Convex vs. Concave Functions (Full Explanation)}
A function $f:\mathbb{R}^d\rightarrow\mathbb{R}$ is:

\begin{itemize}
    \item \textbf{Convex} if  
    \[
    f(\alpha \mathbf{x} + (1-\alpha)\mathbf{y})
    \le
    \alpha f(\mathbf{x}) + (1-\alpha) f(\mathbf{y}),
    \quad 0\le\alpha\le1.
    \]
    Its epigraph forms a convex set.

    \item \textbf{Concave} if  
    \[
    f(\alpha \mathbf{x} + (1-\alpha)\mathbf{y})
    \ge
    \alpha f(\mathbf{x}) + (1-\alpha) f(\mathbf{y}).
    \]

    \item \textbf{Strictly convex} if inequality is strict.
\end{itemize}

\textbf{Key fact:} $-\ln x$ is convex;  
$\ln x$ is concave.  

This property is used to derive Jensen’s inequality:
\[
\ln \mathbb{E}[X]
\ge
\mathbb{E}[\ln X].
\]
\end{frame}

% -----------------------------------------------------
% Jensen → EM lower bound
% -----------------------------------------------------
\begin{frame}{Variational Lower Bound via Jensen’s Inequality}
We start with:
\[
\ln \mathbb{P}(\mathcal{X}\mid\boldsymbol{\theta})
=
\ln
\sum_{Z}
\mathbb{P}(\mathcal{X}, Z \mid \boldsymbol{\theta}).
\]

Introduce any distribution $Q(Z)$:
\[
=
\ln
\sum_{Z}
Q(Z)
\frac{
\mathbb{P}(\mathcal{X}, Z\mid\boldsymbol{\theta})
}{
Q(Z)
}.
\]

Apply Jensen:
\[
\ln \mathbb{P}(\mathcal{X}\mid\boldsymbol{\theta})
\ge
\sum_Z
Q(Z)
\ln
\frac{
\mathbb{P}(\mathcal{X}, Z\mid\boldsymbol{\theta})
}{
Q(Z)
}
\]
Define:
\[
F(\boldsymbol{\theta},Q)
=
\sum_Z
Q(Z)
\ln
\frac{
\mathbb{P}(\mathcal{X}, Z\mid\boldsymbol{\theta})
}{
Q(Z)
}.
\]

And:
\[
\ln\mathbb{P}(\mathcal{X}\mid\boldsymbol{\theta})
=
F(\boldsymbol{\theta},Q)
+
\mathrm{KL}
\big(
Q(Z)
\,
\|\,
\mathbb{P}(Z\mid \mathcal{X},\boldsymbol{\theta})
\big).
\]
\end{frame}

% -----------------------------------------------------
% EM monotonicity
% -----------------------------------------------------
\begin{frame}{Why EM Increases Likelihood (Full Proof) - Part 1}
At iteration $t$:
\[
Q^{(t)}(Z)
=
\mathbb{P}(Z\mid\mathcal{X},\boldsymbol{\theta}^{(t)}).
\]

Then:
\[
\ln\mathbb{P}(\mathcal{X}\mid\boldsymbol{\theta}^{(t)})
=
F(\boldsymbol{\theta}^{(t)}, Q^{(t)}).
\]

M-step chooses:
\[
\boldsymbol{\theta}^{(t+1)}
=
\argmax_{\boldsymbol{\theta}}
F(\boldsymbol{\theta}, Q^{(t)}).
\]

Thus:
\[
F(\boldsymbol{\theta}^{(t+1)}, Q^{(t)})
\ge
F(\boldsymbol{\theta}^{(t)}, Q^{(t)}).
\]
\end{frame}

\begin{frame}{Why EM Increases Likelihood (Full Proof) - Part 2}
Since KL divergence is always non-negative:
\[
\ln\mathbb{P}(\mathcal{X}\mid\boldsymbol{\theta}^{(t+1)})
\ge
F(\boldsymbol{\theta}^{(t+1)}, Q^{(t)}).
\]

Combine them:
\[
\boxed{
\ln\mathbb{P}(\mathcal{X}\mid\boldsymbol{\theta}^{(t+1)})
\ge
\ln\mathbb{P}(\mathcal{X}\mid\boldsymbol{\theta}^{(t)})
}
\]

→ EM \textbf{always increases likelihood}.
\end{frame}

%%%%%%%%%%%%%%%%%%%%%%%%%%%%%%%%%%%%%%%%%%%%%%%%%%%%%%
\section{Examples and Visualizations}
%%%%%%%%%%%%%%%%%%%%%%%%%%%%%%%%%%%%%%%%%%%%%%%%%%%%%%

% -----------------------------------------------------
% Set 1 (3 images)
% -----------------------------------------------------
\begin{frame}{EM \& GMM Example — Set 1 (3 Figures)}
\centering
\begin{tabular}{ccc}
\includegraphics[width=0.30\linewidth]{pic/EM_&_GMM_Example(1)(a).png} &
\includegraphics[width=0.30\linewidth]{pic/EM_&_GMM_Example(1)(b).png} &
\includegraphics[width=0.30\linewidth]{pic/EM_&_GMM_Example(1)(c).png} \\
(a) & (b) & (c)
\end{tabular}

\begin{itemize}
    \item Dataset contains two overlapping Gaussian components.
    \item Colors illustrate \textbf{soft assignments} (responsibilities).
    \item Means gradually move toward cluster centers as EM iterates.
\end{itemize}
\end{frame}

% -----------------------------------------------------
% Set 2 (6 images)
% -----------------------------------------------------
\begin{frame}{EM \& GMM Example — Set 2 (6 Figures)}
\centering
\begin{tabular}{ccc}
\includegraphics[width=0.15\linewidth]{pic/EM_&_GMM_Example(2)(a).png} &
\includegraphics[width=0.15\linewidth]{pic/EM_&_GMM_Example(2)(b).png} &
\includegraphics[width=0.15\linewidth]{pic/EM_&_GMM_Example(2)(c).png} \\
(a) & (b) & (c) \\[3pt]
\includegraphics[width=0.15\linewidth]{pic/EM_&_GMM_Example(2)(d).png} &
\includegraphics[width=0.15\linewidth]{pic/EM_&_GMM_Example(2)(e).png} &
\includegraphics[width=0.15\linewidth]{pic/EM_&_GMM_Example(2)(f).png} \\
(d) & (e) & (f)
\end{tabular}

\begin{itemize}
    \item Six-step visualization of EM evolution in 2D.
    \item Components reshape (covariance) according to local density.
    \item Shows \textbf{different initializations → different convergence paths}.
\end{itemize}
\end{frame}

% -----------------------------------------------------
% Iteration view
% -----------------------------------------------------
\begin{frame}{EM Iterations — Evolution of Parameters}
\centering
\begin{tabular}{cccc}
\includegraphics[width=0.22\linewidth]{pic/EM_&_GMM_Example_Iteration1.png} &
\includegraphics[width=0.22\linewidth]{pic/EM_&_GMM_Example_Iteration2.png} &
\includegraphics[width=0.22\linewidth]{pic/EM_&_GMM_Example_Iteration5.png} &
\includegraphics[width=0.22\linewidth]{pic/EM_&_GMM_Example_Iteration25.png} \\
Iter 1 & Iter 2 & Iter 5 & Iter 25
\end{tabular}

\begin{itemize}
    \item Early steps: drastic changes in means/covariances.
    \item Later steps: parameters stabilize and likelihood increases slowly.
    \item EM often converges smoothly after a few iterations.
\end{itemize}
\end{frame}

%%%%%%%%%%%%%%%%%%%%%%%%%%%%%%%%%%%%%%%%%%%%%%%%%%%%%%
% 1D Example — with numeric info
%%%%%%%%%%%%%%%%%%%%%%%%%%%%%%%%%%%%%%%%%%%%%%%%%%%%%%
\begin{frame}{GMM 1-D Example — Detailed}
\begin{minipage}{0.56\textwidth}
\begin{itemize}
    \item True mixture:
    \[
    \pi_1 = 0.6,\quad \pi_2 = 0.4
    \]
    \[
    \mu_1 = 0,\quad \mu_2 = 3
    \]
    \[
    \sigma_1^2 = 0.5,\quad \sigma_2^2 = 1.0
    \]
    \item EM initialized with random means and equal variances.
    \item After convergence, EM recovers mixture parameters accurately.
    \item Plot shows true density vs. EM estimate.
\end{itemize}
\end{minipage}
\hfill
\begin{minipage}{0.40\textwidth}
\centering
\includegraphics[width=\linewidth]{pic/GMM_1-D_Example(2).png}
\end{minipage}
\end{frame}

%%%%%%%%%%%%%%%%%%%%%%%%%%%%%%%%%%%%%%%%%%%%%%%%%%%%%%
% 2D Example
%%%%%%%%%%%%%%%%%%%%%%%%%%%%%%%%%%%%%%%%%%%%%%%%%%%%%%
\begin{frame}{GMM 2-D Example — Detailed}
\begin{minipage}{0.56\textwidth}
\begin{itemize}
    \item Synthetic dataset with $K=3$ components.
    \item Components have \textbf{anisotropic covariances}:
    \[
    \mathbf{\Sigma}_j \text{ not diagonal}
    \]
    \item EM visually fits ellipses corresponding to covariance contours.
    \item Soft assignments shown by color gradient.
\end{itemize}
\end{minipage}
\hfill
\begin{minipage}{0.40\textwidth}
\centering
\includegraphics[width=\linewidth]{pic/GMM_2-D_Example(1).png}
\end{minipage}
\end{frame}

%%%%%%%%%%%%%%%%%%%%%%%%%%%%%%%%%%%%%%%%%%%%%%%%%%%%%%
\section{Convergence, Practical Issues, and Failure Modes}
%%%%%%%%%%%%%%%%%%%%%%%%%%%%%%%%%%%%%%%%%%%%%%%%%%%%%%

% -----------------------------------------------------
% Monotonicity figure
% -----------------------------------------------------
\begin{frame}{Convergence: Monotonicity and Local Optima}
\begin{itemize}
    \item EM \textbf{always increases} the log-likelihood:
    \[
    \ell(\boldsymbol{\theta}^{(t+1)})
    \ge
    \ell(\boldsymbol{\theta}^{(t)}).
    \]
    \item EM converges to a stationary point:
    \[
    \nabla \ell(\boldsymbol{\theta}) = 0.
    \]
    \item This stationary point may be a:
    \begin{itemize}
        \item local maximum,
        \item saddle point.
    \end{itemize}
    \item Initialization has major impact on final solution.
\end{itemize}

\centering
\begin{tabular}{cc}
\includegraphics[width=0.2\linewidth]{pic/concave.png} &
\includegraphics[width=0.2\linewidth]{pic/convex.png} \\
Concave LL & Convex LL
\end{tabular}

\end{frame}

% -----------------------------------------------------
% Practical tips
% -----------------------------------------------------
\begin{frame}{Practical Tips for Using EM in GMMs}
\begin{itemize}
    \item \textbf{Initialization}:
    \begin{itemize}
        \item k-means
        \item k-means++ for more robustness
        \item multiple random restarts
    \end{itemize}
    \item \textbf{Covariance choices}:
    \[
    \text{full, diagonal, tied, spherical}
    \]
    \item \textbf{Stopping criteria}:
    \begin{itemize}
        \item $\Delta \ell < \epsilon$
        \item small parameter change
        \item max iterations
    \end{itemize}
    \item \textbf{Regularization}:
    \[
    \mathbf{\Sigma}_j \leftarrow \mathbf{\Sigma}_j + \epsilon I
    \qquad (\epsilon \approx 10^{-5})
    \]
\end{itemize}
\end{frame}

% -----------------------------------------------------
% Failure modes
% -----------------------------------------------------
\begin{frame}{Failure Modes and Solutions}
\begin{itemize}
    \item \textbf{Component collapse}:
    \[
    \mathbf{\Sigma}_j \to 0
    \]
    Solution: add regularization, reinitialize collapsed component.
    \item \textbf{Singular covariance}: happens when one component takes a single point.
    \item \textbf{Slow convergence}:  
    Solutions:
    \begin{itemize}
        \item better initialization
        \item annealing/tempering EM
        \item variational EM
        \item stochastic EM
    \end{itemize}
    \item \textbf{Overfitting}: use MAP estimation or Bayesian GMM.
\end{itemize}
\end{frame}

% -----------------------------------------------------
% Comparison to k-means
% -----------------------------------------------------
\begin{frame}{Comparison: EM for GMM vs. k-means}
\begin{itemize}
    \item k-means:
    \begin{itemize}
        \item hard assignments
        \item spherical clusters
        \item minimizes within-cluster square distances
    \end{itemize}
    \item EM + GMM:
    \begin{itemize}
        \item soft assignments ($\gamma_j^{(i)}$)
        \item full covariance modeling
        \item probabilistic interpretation (likelihood)
    \end{itemize}
    \item EM is more flexible—but requires more parameters and computation.
\end{itemize}
\end{frame}

%%%%%%%%%%%%%%%%%%%%%%%%%%%%%%%%%%%%%%%%%%%%%%%%%%%%%%
\section{Summary and References}
%%%%%%%%%%%%%%%%%%%%%%%%%%%%%%%%%%%%%%%%%%%%%%%%%%%%%%

% -----------------------------------------------------
% Summary
% -----------------------------------------------------
\begin{frame}{Summary}
\begin{itemize}
    \item GMMs provide flexible density estimation with multiple Gaussian components.
    \item EM algorithm:
    \begin{itemize}
        \item E-step computes responsibilities.
        \item M-step updates $\mathbf{\mu}_j$, $\mathbf{\Sigma}_j$, $\pi_j$.
    \end{itemize}
    \item EM maximizes a lower bound and increases likelihood monotonically.
    \item Initialization and regularization are critical to good performance.
    \item GMMs outperform k-means when clusters are anisotropic or overlapping.
\end{itemize}
\end{frame}

% -----------------------------------------------------
% References
% -----------------------------------------------------
\begin{frame}{References}
\begin{itemize}
    \item C. M. Bishop, \textit{Pattern Recognition and Machine Learning}, Chapter 9.
    \item Original slides from:
    \[
    \text{Hamid R. Rabiee \& Zahra Dehghanian (Spring 2025)}
    \]
    \item Additional illustrations from course-provided materials.
\end{itemize}
\end{frame}

% -----------------------------------------------------
% Acknowledgement
% -----------------------------------------------------
\begin{frame}{Acknowledgement}
\centering
{\Large Special Thanks}

\vspace{1cm}

\Large
\textbf{Soheil Sayah Varg}

\vspace{1cm}

{\small for contributions to preparing, organizing, and refining this slide deck.}
\end{frame}

% -----------------------------------------------------
% Final page
% -----------------------------------------------------
\begin{frame}
\centering
{\Huge Thank You!}

\vspace{0.5cm}
{\Large Questions?}
\end{frame}

\end{document}
