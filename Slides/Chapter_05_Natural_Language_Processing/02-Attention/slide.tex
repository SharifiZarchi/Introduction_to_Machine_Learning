%%%%%%%%%%%%%%%%%%%%%%%%%%%%%%%%%%%%%%%%%%%%%%%%%%%%%%
% A Beamer template for University of Wollongong     %
% Based on THU beamer theme                          %
% Author: Qiuyu Lu                                   %
% Date: July 2024                                    %
% LPPL Licensed.                                     %
%%%%%%%%%%%%%%%%%%%%%%%%%%%%%%%%%%%%%%%%%%%%%%%%%%%%%%
% Customized for Sharif University of Technology     %
%%%%%%%%%%%%%%%%%%%%%%%%%%%%%%%%%%%%%%%%%%%%%%%%%%%%%%


\documentclass[serif, aspectratio=169]{beamer}
\usepackage{pgfplots} % Required for plotting
\pgfplotsset{compat=1.17} % Compatibility level
%\documentclass[serif]{beamer}  % for 4:3 ratio
\usepackage[T1]{fontenc} 
\usepackage{fourier} % see "http://faq.ktug.org/wiki/uploads/MathFonts.pdf" for other options
\usepackage{hyperref}
\usepackage{latexsym,amsmath,xcolor,multicol,booktabs,calligra}
\usepackage{graphicx,pstricks,listings,stackengine}
\usepackage{lipsum}
\usepackage{array}      


\author{Ali Sharifi-Zarchi}
\title{Machine Learning (CE 40477)}
\subtitle{Fall 2024}
\institute{
    CE Department \\
    Sharif University of Technology
}
%\date{\small \today}
% \usepackage{UoWstyle}
\usepackage{SUTstyle}

% defs
\def\cmd#1{\texttt{\color{red}\footnotesize $\backslash$#1}}
\def\env#1{\texttt{\color{blue}\footnotesize #1}}
\definecolor{deepblue}{rgb}{0,0,0.5}
\definecolor{deepred}{RGB}{153,0,0}
\definecolor{deepgreen}{rgb}{0,0.5,0}
\definecolor{halfgray}{gray}{0.55}

\lstset{
    basicstyle=\ttfamily\small,
    keywordstyle=\bfseries\color{deepblue},
    emphstyle=\ttfamily\color{deepred},    % Custom highlighting style
    stringstyle=\color{deepgreen},
    numbers=left,
    numberstyle=\small\color{halfgray},
    rulesepcolor=\color{red!20!green!20!blue!20},
    frame=shadowbox,
}


\begin{document}

\begin{frame}
    \titlepage
    \vspace*{-0.6cm}
    \begin{figure}[htpb]
        \begin{center}
            \includegraphics[keepaspectratio, scale=0.25]{pic/sharif-main-logo.png}
        \end{center}
    \end{figure}
\end{frame}

\begin{frame}    
\tableofcontents[sectionstyle=show,
subsectionstyle=show/shaded/hide,
subsubsectionstyle=show/shaded/hide]
\end{frame}

\section{Contextualized Word Embeddings}

\subsection{Limitations of word2vec}

\begin{frame}{Limitations of word2vec}
    \begin{itemize} 
    
        \item One vector for each word type
        
        \item Word2vec has challenges in polysemous words, e.g. \texttt{Light}
       % {شیر جنگل، شیر خوراکی و شیر آب}
        \item Words don’t appear in isolation. The word use (e.g., syntax and semantics) depends on its context.
        
        \item \textbf{Why not learn the representations for each word in its context?}

        \end{itemize}
\end{frame}


\begin{frame}{Contextualized Word Embeddings}

    \begin{itemize}
        \item Build a vector for each word conditioned on its \textbf{context}.
    \end{itemize}

    \begin{figure}
        \centering
        \includegraphics[width=0.6\textwidth]{pic/Contextualized-word-embedding.png}
        \caption{representation for each token is a function of the entire input sentence}
        \label{fig:Contextualized-word-embedding}
    \end{figure}
\end{frame}

\begin{frame}{Contextualized Word Embeddings}
	Compute contextual vector:
	\[
	\mathbf{c_k} = f(w_k \mid w_1, w_2, \dots, w_n) \in \mathbb{R}^d
	\]
	\noindent Examples:
	\vspace{-5pt}
	\[
	f(\textcolor{green}{light} \mid \text{Please turn off the }\overbrace{\textcolor{green}{light}}^{\includegraphics[height=3em]{pic/1.png}}\text{.})
	\]
	\[\neq\]
	\[
	f(\textcolor{orange}{light} \mid \text{This box is very }\underbrace{\textcolor{orange}{light}}_{\includegraphics[height=4em]{pic/Flux-light-weight.jpeg}}\text{ to carry.})
	\]
	\textbf{How do we implement the context function $f$?}
	\vspace{5pt}
\end{frame}


\begin{frame}{Enter Recurrent Neural Networks}
    \begin{itemize}
        \item RNNs: Natural choice for contextual representations.
        \item Process words sequentially, maintaining context.
        \item Each hidden state captures information about:
        \begin{itemize}
            \item Current word
            \item All previous words in sequence
        \end{itemize}
    \end{itemize}
%    \begin{equation*}
%        h_t = f(W_{hh}h_{t-1} + W_{xh}x_t + b)
%    \end{equation*}

    \begin{figure}
        \centering
        \includegraphics[width=0.6\textwidth]{pic/image-RNN.png}
        \caption{where $h_t$ can serve as our contextualized representation.}
        \label{fig:image-RNN}
    \end{figure}
    
\end{frame}

\subsection{Recurrent Neural Networks}
\begin{frame}{RNN hidden state update}
    \begin{itemize}
        \item Process sequences by maintaining a hidden state.
        \item Updates state sequentially: $h_t = f_W(h_{t-1}, x_t)$
        \item This recurrence formula is applied at each time step to process a sequence of vectors x.
        \item The same function and the same set of parameters are used at every time step.
        
    \end{itemize}
     \begin{figure}
         \centering
         \includegraphics[width=0.4\textwidth]{pic/RNN-2.png}
     \end{figure}
\end{frame}

\begin{frame}{RNN Output Generation}
	\begin{itemize}
		\item After updating the hidden state, RNNs generate outputs at each time step.
		\item The output \( y_t \) is typically computed as a function of the current hidden state \( h_t \), often passed through a layer (like a softmax layer) for classification or regression tasks.
	\end{itemize}
	\begin{figure}
		\centering
		\includegraphics[width=0.4\textwidth]{pic/RNN-out.png}
	\end{figure}
\end{frame}

\begin{frame}{(Vanilla) Recurrent Neural Network}
	\begin{itemize}
		\item The state of the RNN consists of a single "hidden" vector \( h_t \), which updates as new inputs are processed.
	\end{itemize}
	\vspace{-35pt}
	\begin{figure}
		\includegraphics[width=0.7\textwidth]{pic/RNN-final.png}
		\caption{RNN hidden state update at time \( t \), where \( h_t \) represents the current context.}
	\end{figure}
	\vspace{-100pt}
	\begin{align*}
		h_t &= f_W(h_{t-1}, x_t) \\ 
		\downarrow \\
		h_t &= f(W_{hh} h_{t-1} + W_{xh} x_t + b) \\ 
		y_t &= W_{hy} h_t
	\end{align*}
\end{frame}

\begin{frame}{Challenges of RNNs}
	\begin{itemize}
		\item While improved versions of vanilla RNNs attempt to resolve some issues, they still struggle with:
		\begin{itemize}
			\item Long-term dependencies
			\item Sequential computation (can't parallelize)
%			\item Vanishing/exploding gradients (attempted to be resolved in improved versions, but this remains an issue in some cases)
		\end{itemize}
		\item \textbf{So, what solutions do we have?}
	\end{itemize}
\end{frame}

\section{Attention Mechanism}

\begin{frame}{Attention is all you need!}
	\begin{figure}
		\centering
		\includegraphics[width=0.9\linewidth]{pic/attention-is-all-you-need.pdf}
		\label{fig:attention-01}
	\end{figure}
	\begin{tikzpicture}[remember picture,overlay]
		\node[anchor=south west, xshift=0.09cm, yshift=0.3cm] at (current page.south west) {\tiny \href{https://arxiv.org/abs/1706.03762}{https://arxiv.org/abs/1706.03762}};
	\end{tikzpicture}
\end{frame}
\newpage

\subsection{Why Attention}
\begin{frame}{Why Attention?}
	\begin{itemize}
		\item Enhances the model's ability to focus on relevant parts of the input.
		\item Can compute relationships between inputs regardless of their position.
		\item Inspired by human visual attention.
		\item \textbf{Key Idea}
		\begin{itemize}
			\item Let the model learn which parts of the input are important for each output.
		\end{itemize}
	\end{itemize}
\end{frame}

\subsection{RNNs with Attention Mechanism}
\begin{frame}{Motivation for Attention in RNNs}
	
	\textbf{Attention Mechanism}
	\begin{itemize}
		\item Helps RNNs focus on important parts of the sequence.
		\item Allows model to “attend” the most relevant information from the entire sequence, rather than relying solely on the hidden state at the current time step.
		\item Improves the model's ability to handle long-range dependencies by:
		\begin{itemize}
			\item Directly accessing key information at any position in the input sequence.
			\item Avoiding the need to compress all information into a single hidden state.
		\end{itemize}
	\end{itemize}
\end{frame}

\begin{frame}{How Attention Works?}
    \begin{figure}
        \centering
        \includegraphics[width=0.415\textwidth]{pic/Attention-1.png}
        \label{fig:Attention-1}
    \end{figure}
\end{frame}

\begin{frame}{Attention as a Soft, Averaging Lookup Table}
\begin{figure}[!htb]
    \centering
    \begin{minipage}{0.4\textwidth}
        \centering
        \includegraphics[width=\textwidth]{pic/attention-3.png}
        \caption{In a \textcolor{orange}{lookup table}, we have a table of keys that map to values. The query matches one of the keys, returning its value.}
        \label{fig:attention-4}
        
    \end{minipage}%
    \hfill
    {\vrule width 1pt}
    \hfill
    \begin{minipage}{0.48\textwidth}
        \centering
        \includegraphics[width=\textwidth]{pic/attention-4.png}
                \caption{In the \textcolor{green}{attention}, the query matches all keys softly, to a weight between 0 and 1. The keys' values are multiplied and summed by the weights.}
        \label{fig:attention-3}
    \end{minipage}
\end{figure}
    
\end{frame}

\begin{frame}{Attention Mechanism: Basic Formula}
    \begin{itemize}
        \item Query (Q): What we're looking for
        \item Key (K): What we match against
        \item Value (V): What we retrieve
    \end{itemize}
    \begin{equation*}
        \text{Attention}(Q, K, V) = \text{softmax}\left(\frac{QK^T}{\sqrt{d_k}}\right)V
    \end{equation*}
    \begin{itemize}
        \item $d_k$: dimension of keys (scaling factor)
        \item Softmax converts scores to probabilities
    \end{itemize}
\end{frame}

\begin{frame}{Attention Mechanism: Basic Formula}
    \begin{figure}
        \centering
        \includegraphics[width=0.61\textwidth]{pic/attention-2__AIE.png}
        \label{fig:attention-2}
    \end{figure}
\end{frame}

\section{Types of Attention}
\begin{frame}{Self-Attention vs Cross-Attention}
    \begin{columns}
        \column{0.5\textwidth}
        \textbf{Self-Attention}
        \begin{itemize}
            \item Q, K, V from the same sequence
            \item Each position attends to all positions
            \item Used in encoder and decoder
        \end{itemize}
        \vspace{0.5em}
        \includegraphics[width=\textwidth]{pic/self-attention.png}
        
        \column{0.5\textwidth}
        \textbf{Cross-Attention}
        \begin{itemize}
            \item Q from one sequence
            \item K, V from another
            \item Used in encoder-decoder attention
        \end{itemize}
        \vspace{0.5em}
        \includegraphics[width=\textwidth]{pic/cross-attention.png}
    \end{columns}
\end{frame}

\section{Self-Attention Deep Dive}

\begin{frame}{Self-Attention: Core Theorem}
    \textcolor{blue}{\textbf{Self-Attention Properties}}
    \newline
        Self-attention layers can model dependencies between all elements in an input sequence in parallel, capturing long-range relationships efficiently.

    \begin{itemize}
            \item Global connectivity
            \item Parallel computation
            \item Position-independent weighting
            \item $O(n^2)$ complexity for sequence length $n$
    \end{itemize}
\end{frame}

\begin{frame}{Self-Attention: Mathematical Foundation}
    \begin{itemize}
        \item Input sequence: $X = [x_1, x_2, ..., x_n] \in \mathbb{R}^{n \times d}$
        \item Linear projections:
        \begin{align*}
            Q &= XW^Q \in \mathbb{R}^{n \times d_k} \\
            K &= XW^K \in \mathbb{R}^{n \times d_k} \\
            V &= XW^V \in \mathbb{R}^{n \times d_v}
        \end{align*}
        \item Attention computation:
        \begin{equation*}
            \text{Attention}(Q, K, V) = \text{softmax}\left(\frac{QK^T}{\sqrt{d_k}}\right)V
        \end{equation*}
    \end{itemize}
\end{frame}

\begin{frame}{Proof Component 1: Parallel Processing}
    \begin{itemize}
        \item Matrix multiplication enables parallel computation:
        \begin{equation*}
            S = QK^T = \begin{bmatrix}
            (q_1k_1^T) & \cdots & (q_1k_n^T) \\
            \vdots & \ddots & \vdots \\
            (q_nk_1^T) & \cdots & (q_nk_n^T)
            \end{bmatrix}
        \end{equation*}
        \item All $n^2$ interactions computed simultaneously
        \item No sequential dependencies between computations
    \end{itemize}
    \textcolor{blue}{\textbf{Key Insight}}
    \newline
        Unlike RNNs, there is no need to wait for previous timesteps
\end{frame}

\begin{frame}{Proof Component 2: Global Dependencies}
    \begin{itemize}
        \item Attention weights between positions $i$ and $j$:
        \begin{equation*}
            \alpha_{ij} = \frac{\exp(q_i^Tk_j/\sqrt{d_k})}{\sum_{l=1}^n \exp(q_i^Tk_l/\sqrt{d_k})}
        \end{equation*}
        \begin{itemize}
            \item Properties:
            \item $\alpha_{ij}$ depends only on compatibility of $i$ and $j$
            \item No distance-based attenuation
            \item Softmax ensures $\sum_j \alpha_{ij} = 1$
        \end{itemize}
    \end{itemize}
\end{frame}

\begin{frame}{Proof Component 2: Global Dependencies}
        \begin{figure}
        \centering
        \includegraphics[width=0.85\textwidth]{pic/attention-weights.png}
        \caption{Attention weight visualization}
    \end{figure}
\end{frame}

\begin{frame}{Proof Component 3: Computational Efficiency}
    \begin{columns}
        \column{0.5\textwidth}
        \textbf{Complexity Analysis}
        \begin{itemize}
            \item Matrix multiply: $O(n^2d)$
            \item Softmax: $O(n^2)$
            \item Total: $O(n^2d)$
        \end{itemize}
        
        \column{0.5\textwidth}
        \textbf{Optimizations}
        \begin{itemize}
            \item Sparse attention
            \item Linear attention
            \item Sliding window
        \end{itemize}
    \end{columns}

    \vspace{0.5cm}
    \textcolor{blue}{\textbf{Memory-Computation Tradeoff}}
    
    \vspace{0.2cm}
    More memory than RNNs ($O(n)$), but enables parallelization.
    
\end{frame}

\begin{frame}{Implementation Details}
    \begin{itemize}
        \item Practical considerations:
        \begin{itemize}
            \item Scale factor $\sqrt{d_k}$ prevents vanishing gradients
            \item Mask for causal attention (decoder)
            \item Dropout for regularization
        \end{itemize}
    \end{itemize}
    \begin{equation*}
        \text{MaskedAttention}(Q, K, V) = \text{softmax}\left(\frac{QK^T}{\sqrt{d_k}} + M\right)V
    \end{equation*}
    where $M_{ij} = -\infty$ for masked positions
\end{frame}

\begin{frame}{Formal Proof Conclusion}
    \textcolor{blue}{\textbf{Theorem Verification}}
    \newline
        Self-attention satisfies all claimed properties:
        \begin{enumerate}
            \item Parallel computation: Matrix operations
            \item Global dependencies: Direct pairwise attention
            \item Efficient computation: $O(n^2d)$ complexity
            \item Learnable patterns: Attention weights
        \end{enumerate}
    \textcolor{blue}{\textbf{Important Note}}
    \newline
        These properties make self-attention particularly suitable for:
        \begin{itemize}
            \item Natural language processing
            \item Graph neural networks
            \item Computer vision (with modifications)
        \end{itemize}
\end{frame}

\section{Positional Encoding}
\begin{frame}{Positional Encoding}
    \begin{itemize}
        \item Problem: Attention is position-agnostic
        \item Solution: Add position information to embeddings
        \item Using sinusoidal functions:
    \end{itemize}
    \begin{equation*}
        PE_{(pos,2i)} = \sin(pos/10000^{2i/d_{model}})
    \end{equation*}
    \begin{equation*}
        PE_{(pos,2i+1)} = \cos(pos/10000^{2i/d_{model}})
    \end{equation*}

    \begin{itemize}
        \item Allows model to learn relative positions
        \item Works for sequences of any length
    \end{itemize}
    
\end{frame}


\begin{frame}{Positional Encoding}

    \begin{figure}
        \centering
        \includegraphics[width=0.45\textwidth]{pic/positional-encoding-1.png}
        \label{fig:positional-encoding-1}
    \end{figure}
\end{frame}

\section{Multi-Head Attention}

\begin{frame}{Multi-Head Attention}

    \begin{figure}
        \centering
        \includegraphics[width=0.7\textwidth]{pic/multihead-attention-1.png}
        \label{fig:multihead_attention-1}
    \end{figure}
\end{frame}

\begin{frame}{Multi-Head Attention}
    \begin{itemize}
        \item Instead of single attention, use multiple heads
        \item Each head can focus on different aspects
        \item Process in parallel and concatenate
    \end{itemize}
    \begin{equation*}
        \text{MultiHead}(Q,K,V) = \text{Concat}(head_1,\ldots,head_h)W^O
    \end{equation*}
    \begin{equation*}
        head_i = \text{Attention}(QW^Q_i,KW^K_i,VW^V_i)
    \end{equation*}
\end{frame}

\begin{frame}{Multi-Head Attention}

    \begin{figure}
        \centering
        \includegraphics[width=0.35\textwidth]{pic/multihead-attention-2.png}
        \label{fig:multihead_attention}
    \end{figure}
\end{frame}

\begin{frame}{Multi-Head Attention Benefits}
    \begin{itemize}
        \item Multiple representation subspaces
        \item Can capture different types of relationships:
        \begin{itemize}
            \item Syntactic dependencies
            \item Semantic relationships
            \item Long-range dependencies
        \end{itemize}
        \item Improves model capacity and stability
    \end{itemize}
    % \begin{figure}
    %     \centering
    %     \includegraphics[width=0.7\textwidth]{pic/multihead-attention.png}
    %     \caption{Multi-head attention architecture}
    % \end{figure}
\end{frame}

\begin{frame}{Contributions}
	\textbf{These slides are authored by:}
	\begin{itemize}
		\item Faezeh Sarlakifar
	\end{itemize}
	
\end{frame}


\section{References}

\begin{frame}[allowframebreaks]
    \bibliography{ref}
    \bibliographystyle{ieeetr}
    \nocite{*} % used here because no citation happens in slides
    % if there are too many try use:
    % \tiny\bibliographystyle{alpha}
\end{frame}


\begin{frame}
    \begin{center}
        {\Huge Any Questions?}
    \end{center}
\end{frame}

\end{document}