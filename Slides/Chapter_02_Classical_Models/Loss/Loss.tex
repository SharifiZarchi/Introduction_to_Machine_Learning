%%%%%%%%%%%%%%%%%%%%%%%%%%%%%%% beamer %%%%%%%%%%%%%%%%%%%%%%%%%%%%%%%%%%%%%%%%%%%%%%%%%
% To run - pdflatex filename.tex
%      acroread filename.pdf
%%%%%%%%%%%%%%%%%%%%%%%%%%%%%%%%%%%%%%%%%%%%%%%%%%%%%%%%%%%%%%%%%%%%%%%%%%%%%%%%%%%%%%%%

\documentclass[compress,oilve]{beamer}
\mode<presentation>
\setbeamertemplate{itemize/enumerate body begin}{\normalsize}
\setbeamertemplate{itemize/enumerate subbody begin}{\small}

\usetheme[]{CambridgeUS}
% other themes: AnnArbor, Antibes, Bergen, Berkeley, Berlin, Boadilla, boxes, CambridgeUS, Copenhagen, Darmstadt, default, Dresden, Frankfurt, Goettingen,
% Hannover, Ilmenau, JuanLesPins, Luebeck, Madrid, Maloe, Marburg, Montpellier, PaloAlto, Pittsburg, Rochester, Singapore, Szeged, classic

\usecolortheme{beaver}
% color themes: albatross, beaver, beetle, crane, default, dolphin,  fly, lily, orchid, rose, seagull, seahorse, sidebartab, whale, wolverine

\usefonttheme{professionalfonts}
% font themes: default, professionalfonts, serif, structurebold, structureitalicserif, structuresmallcapsserif


\hypersetup{pdfpagemode=FullScreen} % makes your presentation go automatically to full screen

% define your own colors:
\definecolor{Red}{rgb}{1,0,0}
\definecolor{Blue}{rgb}{0,0,1}
\definecolor{Green}{rgb}{0,1,0}
\definecolor{magenta}{rgb}{1,0,.6}
\definecolor{lightblue}{rgb}{0,.5,1}
\definecolor{lightpurple}{rgb}{0.8, 0.6, 0.9}
\definecolor{gold}{rgb}{.6,.5,0}
\definecolor{orange}{rgb}{1,0.4,0}
\definecolor{hotpink}{rgb}{1,0,0.5}
\definecolor{newcolor2}{rgb}{.5,.3,.5}
\definecolor{newcolor}{rgb}{0,.3,1}
\definecolor{newcolor3}{rgb}{1,0,.35}
\definecolor{darkgreen1}{rgb}{0, .35, 0}
\definecolor{darkgreen}{rgb}{0, .6, 0}
\definecolor{darkred}{rgb}{.75,0,0}
\definecolor{skyblue}{HTML}{75bbfd}

\definecolor{olive}{cmyk}{0.64,0,0.95,0.4}
\definecolor{purpleish}{cmyk}{0.75,0.75,0,0}

% can also choose different themes for the "inside" and "outside"

% \usepackage{beamerinnertheme_______}
% inner themes include circles, default, inmargin, rectangles, rounded

% \usepackage{beamerouterthemesmoothbars}
% outer themes include default, infolines, miniframes, shadow, sidebar, smoothbars, smoothtree, split, tree


\useoutertheme[subsection=true, height=40pt]{smoothbars}

% to have the same footer on all slides
%\setbeamertemplate{footline}[text line]{STUFF HERE!}
\setbeamertemplate{footline}[text line]{} % makes the footer EMPTY
% include packages
%

%show the page numbers in footnote
%\addtobeamertemplate{navigation symbols}{}{%
%	\usebeamerfont{footline}%
%	\usebeamercolor[fg]{footline}%
%	\hspace{1em}%
%	\insertframenumber/\inserttotalframenumber
%}

\setbeamercolor{footline}{fg=purpleish}
\setbeamerfont{footline}{series=\bfseries}

%add color to curent subsection
\setbeamertemplate{section in head/foot}{\hfill\tikz\node[rectangle, fill=darkred, rounded corners=1pt,inner sep=1pt,] {\textcolor{white}{\insertsectionhead}};}
\setbeamertemplate{section in head/foot shaded}{\textcolor{darkred}{\hfill\insertsectionhead}}

% Remove bullet of subsections
\setbeamertemplate{headline}
{%
	\begin{beamercolorbox}{section in head/foot}
		\insertsectionnavigationhorizontal{\textwidth}{}{}
	\end{beamercolorbox}%
}


% modify headlline, specially headline size
\setbeamertemplate{headline}{%
	\leavevmode%
	\hbox{%
		\begin{beamercolorbox}[wd=\paperwidth,ht=3.5ex,dp=1.125ex]{palette quaternary}%
			\insertsectionnavigationhorizontal{\paperwidth}{}{\hskip0pt plus1filll}
		\end{beamercolorbox}%
	}
}

\setbeamertemplate{footline}{%
	\leavevmode%
	\hbox{\begin{beamercolorbox}[wd=.5\paperwidth,ht=2.5ex,dp=1.125ex,leftskip=.3cm plus1fill,rightskip=.3cm]{author in head/foot}%
			\usebeamerfont{author in head/foot}\insertshortauthor ~ \insertshortinstitute
		\end{beamercolorbox}%
		\begin{beamercolorbox}[wd=.5\paperwidth,ht=2.5ex,dp=1.125ex,leftskip=.3cm,rightskip=.3cm plus1fil]{title in head/foot}%
			\usebeamerfont{title in head/foot}\insertshorttitle\hfill\insertframenumber\,/\,\inserttotalframenumber
	\end{beamercolorbox}}%
	\vskip0pt%
}


%\setbeamertemplate{navigation symbols}{}

\title{Loss for Classification and Regression}
\author{ML Instruction Team, Fall 2022}
\institute[]{CE Department \newline  Sharif University of Technology \newline \newline}
\date[\today]{}
%\titlegraphic{\includegraphics[scale=.35]{example-image}}



%Write \usepackage{etex} just after the \documentclass line (it should be the first loaded package).
\usepackage{etex}
\usepackage{subcaption}
\usepackage{multicol}
\usepackage{amsmath}
\usepackage{epsfig}
\usepackage{graphicx}
\usepackage[all,knot]{xy}
\xyoption{arc}
\usepackage{url}
\usepackage{multimedia}
\usepackage{hyperref}
\hypersetup{colorlinks,linkcolor=blue,citecolor=redorange,urlcolor=darkred}
\usepackage{multirow}
\usepackage[font={scriptsize}]{caption}
\usepackage{pgf}
\usepackage{fontspec}
%\setsansfont[Scale=MatchLowercase, BoldFont = * Bold, ItalicFont = * Italic]{Caladea}

%\usepackage{enumitem,xcolor}
%\newcommand{\labelitemi}{$\blacksquare$}
%\newcommand{\labelitemii}{$\diamond$}
%\newcommand{\labelitemiii}{$\square$}
%\newcommand{\labelitemiv}{$\ast$}
%\setbeamercolor*{item}{fg=red}


\usefonttheme{professionalfonts} 
\setbeamertemplate{itemize item}{\color{skyblue}$\blacksquare$}
\setbeamertemplate{itemize subitem}{\color{hotpink}$\triangleright$}
\setbeamertemplate{itemize subsubitem}{\color{orange}$\bullet$}


\usepackage{anyfontsize}
\usepackage{t1enc}
\usepackage{tikz}
\usetikzlibrary{calc,trees,positioning,arrows,chains,shapes.geometric,decorations.pathreplacing,decorations.pathmorphing,shapes,matrix,shapes.symbols}


\usepackage{xcolor}
\newcommand{\tc}[2]{
	\textcolor{#1}{\hspace{-2pt}#2\hspace{-2pt}}
}

\newtheorem{proposition}[theorem]{Proposition}
\newtheorem{remark}[theorem]{Remark}
\newtheorem{assumption}[theorem]{Assumption}

%\usepackage{fontspec,unicode-math}
%\setmainfont[Scale=0.9]{Nimbus Roman No9 L}
%\setmonofont[Scale=0.9]{Monaco}
\setsansfont[Scale=1]{Times New Roman}

\newcommand{\vect}[1]{\boldsymbol{#1}}

\definecolor{strings}{rgb}{.624,.251,.259}
\definecolor{keywords}{rgb}{.893,.188,.01}
\definecolor{comment}{rgb}{.322,.451,.322}

%\usepackage{smartdiagram}
%\usesmartdiagramlibrary{additions}
%%%%%%%%%%%%%%%%%%%%%%%%%%%%%%%%%%%%%%%%%%%%%%%%%%%%%%%%%%%%%%%%%%%%%%%%%%%%%%%%%%%%%%%%%%%%
%%%%%%%%%%%%%%%%%%%%%%%%%%%%%% Title Page Info %%%%%%%%%%%%%%%%%%%%%%%%%%%%%%%%%%%%%%%%%%%
%%%%%%%%%%%%%%%%%%%%%%%%%%%%%%%%%%%%%%%%%%%%%%%%%%%%%%%%%%%%%%%%%%%%%%%%%%%%%%%%%%%%%%%%%%


%%%%%%%%%%%%%%%%%%%%%%%%%%%%%%%%%%%%%%%%%%%%%%%%%%%%%%%%%%%%%%%%%%%%%%%%%%%%%%%%%%%%%%%%%%
%%%%%%%%%%%%%%%%%%%%%%%%%%%%%% Begin Your Document %%%%%%%%%%%%%%%%%%%%%%%%%%%%%%%%%%%%%%%
%%%%%%%%%%%%%%%%%%%%%%%%%%%%%%%%%%%%%%%%%%%%%%%%%%%%%%%%%%%%%%%%%%%%%%%%%%%%%%%%%%%%%%%%%%
\begin{document}
	
%%%%%%%%%%%%%%%%%%%%%%%%%%%%%%%%%%%%%%%%%%%%%%%%%%%%%%%%%%%%%%%%%%%%%%%%%%%%%%%%%%%%%%%%%%
\fontsize{9}{9}
\begin{frame}[noframenumbering, plain]
	\titlepage
\end{frame}

%%%%%%%%%%%%%%%%%%%%%%%%%%%%%%%%%%%%%%%%%%%%%%%%%%%%%%%%%%%%%%%%%%%%%%%%%%%%%%%%%%%%%%%%%%
\section{Classification Loss}
%%%%%%%%%%%%%%%%%%%%%%%%%%%%%%%%%%%%%%%%%%%%%%%%%%%%%%%%%%%%%%%%%%%%%%%%
\frame{\frametitle{Classification Loss}
	
	\begin{itemize}
		\item Both the \tc{keywords}{cross-entropy} and the \tc{keywords}{Kullback–Leibler} (KL) divergence measures the distance between two probability distributions $P$ and $Q$.
		\begin{equation*}
			\begin{aligned}
				&H(P, Q)=-\sum_x P(x) \log Q(x) \\
				&K L(P \mid Q)=\sum_x P(x) \log \frac{P(x)}{Q(x)}=H(P,Q)- H(P)
			\end{aligned}
		\end{equation*}
	
		\medskip
		\item In the above formula, $H(P)= H(P,P)$ is the entropy of the distribution $P$, which is a constant term.
		
		\medskip
		\item
		Hence, it turns out that the minimization of KL divergence is equivalent to the minimization of cross-entropy.

	\end{itemize}

}
%%%%%%%%%%%%%%%%%%%%%%%%%%%%%%%%%%%%%%%%%%%%%%%%%%%%%%%%%%%%%%%%%%%%%%%
\frame{\frametitle{Classification Loss}
	
	\begin{itemize}
		
	\item Logarithmic loss indicates how close a distribution of prediction probability comes to the real probability distribution of the data in \tc{keywords}{binary classification}.
	\begin{equation*}
		H(p, q)=-\frac{1}{m} \sum_{i=1}^m y_i \log (p(y_i))+(1-y_i) \log (1-p(y_i))
	\end{equation*}
	
	where $y_i\in\{0,1\}$
	
	\item Cross-entropy loss indicates how close a distribution of prediction probability comes to the real probability distribution of the data in \tc{keywords}{multilabel classification}.
	
	\begin{equation*}
		H(p, q)=-\frac{1}{m} \sum_{i=1}^m\sum_{j=1}^{k} y_j^{(i)} \log (p(y_j^{(i)}))
	\end{equation*}
	
	where $y_j^{(i)}$ and $p(y_j^{(i)})$ is respectively the true and predicted probability of the class $i$-th of the sample $j$-th.
	
	
	\end{itemize}
}
%%%%%%%%%%%%%%%%%%%%%%%%%%%%%%%%%%%%%%%%%%%%%%%%%%%%%%%%%%%%%%%%%%%%%%
\frame{\frametitle{Classification Loss}
	\begin{figure}
		\centering
		\includegraphics[width=10cm, height=6cm]{Figs/cross_entropy.png}
		\caption{Cross Entropy Evaluation, \href{https://tinyurl.com/2mehqavm}{Source}}
		\label{fig:cross entropy evaluation}
	\end{figure}

}

%%%%%%%%%%%%%%%%%%%%%%%%%%%%%%%%%%%%%%%%%%%%%%%%%%%%%%%%%%%%%%%%%%%%%%%
\frame{\frametitle{Regression Loss}
	\vspace{-0.5cm}
	\begin{equation*}
		\text{MSE}(y, \hat{y}) = \frac{1}{m}\sum_{i=1}^{m}(y_i-\hat{y}_i)^2
	\end{equation*}

	\begin{equation*}
		\text{MAE}(y, \hat{y}) = \frac{1}{m}\sum_{i=1}^{m}|(y_i-\hat{y}_i)|
	\end{equation*}

	\begin{equation*}
		\text{MAPE}(y, \hat{y}) = \frac{1}{m}\sum_{i=1}^{m}|\frac{(y_i-\hat{y}_i)}{y_i}|
	\end{equation*}
	
	\begin{equation*}
		\text{Logcosh}(y, \hat{y}) = \frac{1}{m}\sum_{i=1}^{m}\cosh(y_i-\hat{y}_i)
	\end{equation*}
	
	\begin{equation*}
		L_\delta(y, \hat{y})= \begin{cases}\frac{1}{2} (y-\hat{y})^2 & \text { for }|y-\hat{y}| \leq \delta \\ \delta \cdot(|(y- \hat{y})|-\frac{1}{2} \delta), & \text { otherwise }\end{cases}
	\end{equation*}

}
%%%%%%%%%%%%%%%%%%%%%%%%%%%%%%%%%%%%%%%%%%%%%%%%%%%%%%%%%%%%%%%%%%%%%%%%
\frame{\frametitle{Regression Loss}

	\begin{figure}
		\centering
		\includegraphics[width=10cm, height=6cm]{Figs/regressionlosses.png}
			\caption{Various Regression Losses, \href{https://tinyurl.com/2ldxgbc5}{Source}}
		\label{fig:regtession losses}
	\end{figure}
}


%%%%%%%%%%%%%%%%%%%%%%%%%%%%%%%%%%%%%%%%%%%%%%%%%%%%%%%%%%%%%%%%%%%%%%%%

\frametitle{Final Notes}
\centering
\vspace{50 pt}
\textbf{Thank You!}
\vspace{50pt}

\textbf{Any Question?}
%%%%%%%%%%%%%%%%%%%%%%%%%%%%%%%%%%%%%%%%%%
\end{document}