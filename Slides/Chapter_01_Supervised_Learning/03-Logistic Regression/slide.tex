%%%%%%%%%%%%%%%%%%%%%%%%%%%%%%%%%%%%%%%%%%%%%%%%%%%%%%
% A Beamer template for University of Wollongong     %
% Based on THU beamer theme                          %
% Author: Qiuyu Lu                                   %
% Date: July 2024                                    %
% LPPL Licensed.                                     %
%%%%%%%%%%%%%%%%%%%%%%%%%%%%%%%%%%%%%%%%%%%%%%%%%%%%%%
% Customized for Sharif University of Technology     %
%%%%%%%%%%%%%%%%%%%%%%%%%%%%%%%%%%%%%%%%%%%%%%%%%%%%%%


\documentclass[serif, aspectratio=169]{beamer}
%\documentclass[serif]{beamer}  % for 4:3 ratio
\usepackage[T1]{fontenc} 
\usepackage{fourier} % see "http://faq.ktug.org/wiki/uploads/MathFonts.pdf" for other options
\usepackage{hyperref}
\usepackage{latexsym,amsmath,xcolor,multicol,booktabs,calligra}
\usepackage{graphicx,pstricks,listings,stackengine}
\usepackage{lipsum}
\usepackage[normalem]{ulem}
\usepackage{caption}
\usepackage{tikz}

\author{Ali Sharifi-Zarchi}
\title{Machine Learning (CE 40717)}
\subtitle{Fall 2024}
\institute{
    CE Department \\
    Sharif University of Technology
}
%\date{\small \today}
% \usepackage{UoWstyle}
\usepackage{SUTstyle}

% defs
\def\cmd#1{\texttt{\color{red}\footnotesize $\backslash$#1}}
\def\env#1{\texttt{\color{blue}\footnotesize #1}}
\definecolor{deepblue}{rgb}{0,0,0.5}
\definecolor{deepred}{RGB}{153,0,0}
\definecolor{deepgreen}{rgb}{0,0.5,0}
\definecolor{halfgray}{gray}{0.55}

\lstset{
    basicstyle=\ttfamily\small,
    keywordstyle=\bfseries\color{deepblue},
    emphstyle=\ttfamily\color{deepred},    % Custom highlighting style
    stringstyle=\color{deepgreen},
    numbers=left,
    numberstyle=\small\color{halfgray},
    rulesepcolor=\color{red!20!green!20!blue!20},
    frame=shadowbox,
}
\captionsetup{labelformat=empty}

\begin{document}

\begin{frame}
    \titlepage
    \vspace*{-0.6cm}
    \begin{figure}[htpb]
        \begin{center}
            \includegraphics[keepaspectratio, scale=0.25]{pic/sharif-main-logo.png}
        \end{center}
    \end{figure}
    \vfill % This pushes the next content to the bottom
    \vspace{-0.35cm}
    \centering\textit{\tiny Most Slides are adopted from Dr.Soleymani's ML course}
\end{frame}

\begin{frame}    
\tableofcontents[sectionstyle=show,
subsectionstyle=show/shaded/hide,
subsubsectionstyle=show/shaded/hide]
\end{frame}

\section{Introduction}
\subsection{Probabilistic view in Classification}
%%%% 0 %%%%%

\begin{frame}{Probabilistic view in classification problem}
    \begin{itemize}
        \item First consider a simple classification problem:
            \begin{itemize}
                \item We want to find the people with Diabetes and detect them.
                \item We have two \textbf{observations}: blood cell count and Plasma glucose value.
                \item Observations are our \textbf{features} and Having Diabetes or not is our \textbf{class label} in this classification problem.
            \end{itemize}
    \end{itemize}
        %%% next frame ->
    % \begin{center}
    %     \includegraphics[width=0.45\textwidth]{pic/classification_plot.png}
    %     \captionof{figure}{\scriptize [Sanja Fidler's Slides, University of Toronto, CSC411]}
    % \end{center}
    
\end{frame}
%%%%% 0.25 %%%%%%
\begin{frame}{Probabilistic view in classification problem Cont.}
    \begin{itemize}
        \item Dataset visualization (x-axis denotes white blood count and y-axis represents plasma glucose value:
    \end{itemize}
    \begin{center}
        \includegraphics[width=0.45\textwidth]{pic/classification_plot.png}
        \captionof{figure}{\footnotesize [Sanja Fidler's Slides, University of Toronto, CSC411]}
    \end{center}
\end{frame}

%%%%% 0.3 %%%%%
\begin{frame}{Probabilistic view in classification problem Cont.}
    \begin{itemize}
        \item Each \textbf{feature} is a \textbf{random variable} (e.g. white blood count could be between 15 and 65).
        \item \textbf{Class label} also as a \textbf{random variable} (e.g. a patient could have diabetes or not).
        \item We observe the feature values for a random sample and intend to find its class label
            \begin{itemize}
                \item Evidence : Feature vector $x$
                \item Objective : Class label
            \end{itemize}
    \end{itemize}
\end{frame}
%%%%%0.5%%%%%%
\begin{frame}{Definitions}
    \begin{itemize}
        \item Posterior probability : The probability of a class label $C_k$ given a sample $x$
            \[
                p(C_k|x)
            \]
            
        \item Likelihood or class conditional probability : pdf of feature vector $x$ for samples of class $C_k$
            \[
                 p(x|C_k)
            \]
        
        \item Prior probability : probability of the label be $C_k$ 
            \[
                p(C_k)
            \]
            
        \item $p(x)$: pdf of feature vector $x$ 
            \begin{itemize}
                \item From total probability theorem:   
                
                \[ p(x)=\sum_{k=1}^{K}p(x|C_k)p(C_k)
                \]
            \end{itemize}
        
    \end{itemize}
\end{frame}



%%% 1 %%%%
\subsection{Probabilistic Classifiers}

\begin{frame}{Probabilistic classifiers}
    \begin{itemize}
        \item Probabilistic approaches can be divided in two main categories:
            \begin{itemize}
                \item Generative
                    \begin{itemize}
                        \item Estimate pdf $p(x, C_k)$ for each class $C_k$ and then use it to find $p(C_k|x)$. Alternatively estimate both pdf $p(x|C_k)$ and $p(C_k)$ to find $p(C_k|x)$.
                    \end{itemize}
                \item Discriminative
                    \begin{itemize}
                        \item Directly estimate $p(C_k|x)$ for class $C_k$
                    \end{itemize}
            \end{itemize}
    \end{itemize}
\end{frame}
%%%%%%%% 1.5 %%%%%%%%%%%%
\begin{frame}{Probabilistic classifiers Cont.}
    \begin{itemize}
        \item Let's assume we have input data $x$ and want to classify the data into labels $y$.
        \item A generative model learns the \textbf{joint} probability distribution $p(x,y)$.
            
        \item A discriminative model learns the \textbf{conditional} probability distribution $p(y|x)$
           
    \end{itemize}
\end{frame}

%%%% 1.6 %%%%%%
\begin{frame}{Discriminative vs. Generative : example}
    \begin{itemize}
        \item Suppose we have the following dataset in form of $(x, y)$:
            \[
                (1,0), (1,0), (2,0), (2,1)
            \]
        \item $p(x,y)$ is :
            \[
            \begin{array}{c|cc}
                & y=0 & y=1 \\
                \hline
            x=1 & \frac{1}{2} & 0 \\
            x=2 & \frac{1}{4} & \frac{1}{4} \\
            \end{array}
            \]
        \item $p(y|x)$ is :
            \[
            \begin{array}{c|cc}
                & y=0 & y=1 \\
                \hline
            x=1 & 1 & 0 \\
            x=2 & \frac{1}{2} & \frac{1}{2} \\
            \end{array}
            \]
    \end{itemize}
\end{frame}

%%%%% 1.8 %%%%%

\begin{frame}{Discriminative vs. Generative : example Cont.}
    \begin{itemize}
        \item The distribution $p(y|x)$ is the natural distribution for classifying a given sample $x$ into a class $y$.
            \begin{itemize}
                \item This is why that algorithms which model this directly are called \textbf{discriminative} algorithms.
            \end{itemize}
        \item Generative algorithms model $p(x,y)$, which can be transformed into $p(y|x)$ by Bayes rule and then used for classification.
            \begin{itemize}
                \item However, the distribution $p(x,y)$ can also be used for other purposes.
                \item For example we can use $p(x,y)$ to \textbf{generate} likely $(x,y)$ pairs
            \end{itemize}
    \end{itemize}
\end{frame}

%%%% 2 %%%%%
\begin{frame}{Generative Approach}
    \begin{enumerate}
        \item Inference
        \begin{itemize}
            \item Determine class conditional densities $p(x|C_k)$ and priors $p(C_k)$
            \item Use Bayes theorem to find $p(C_k|x)$
        \end{itemize}
        \item Decision
        \begin{itemize}
            \item Make optimal assignment for new input (after learning the model in the inference stage)
            \item if $p(C_i|x) > p(C_j|x) \forall j \neq i$, then decide $C_i$ .
        \end{itemize}
    \end{enumerate}
\end{frame}

%%%% 2.5 %%%%%
\begin{frame}{Generative Approach Cont.}

    \begin{itemize}
        \item Generative approach for a binary classification problem:
    \end{itemize}
    \begin{figure}[h]
      \centering
      \includegraphics[width=0.4\textwidth]{pic/Generative.png}
      \caption*{\footnotesize [Bishop]}
      \end{figure}
\end{frame}
%%%% 3 %%%%
\begin{frame}{Discriminative Approach}
    \begin{enumerate}
        \item Inference
        \begin{itemize}
            \item Determine the posterior class probabilities $p(C_k|x)$ directly.
        \end{itemize}
        \item Decision
        \begin{itemize}
            \item Make optimal assignment for new input (after learning the model in the inference stage)
            \item if $p(C_i|x) > p(C_j|x) \forall j \neq i$, then decide $C_i$ .
        \end{itemize}
    \end{enumerate}
\end{frame}
%%%% 3.5 %%%%%%
\begin{frame}{Discriminative Approach Cont.}
    \begin{itemize}
        \item Discriminative approach for a binary classification problem:
    \end{itemize}
    \begin{figure}[h]
      \centering
      \includegraphics[width=0.4\textwidth]{pic/Disc.png}
      \caption*{\footnotesize [Bishop]}
      \end{figure}
\end{frame}
%%% 4 %%%%
    %% partition photos into 2 slides
% \begin{frame}{Discriminative vs. Generative approach}
%   \begin{figure}[h]
%   \centering
%   \includegraphics[width=\textwidth]{pic/GvD.png}
%   \caption*{[Bishop]}
%   % \label{fig:image}
% \end{figure}
% \end{frame}


\section{Logistic Regression}

\subsection{Basics}


%%%% 5 %%%%%
\begin{frame}{Basics}
    \begin{itemize}
        \item Logistic Regression is a \textbf{discriminative} approach.
    \end{itemize}
    \begin{figure}[h]
      \centering
      \includegraphics[width=0.6\textwidth]{pic/logisticR-Linear.png}
      % \label{fig:image}
    \end{figure}
    \begin{itemize}
        \item First think of a binary classification task (so $K=2$).
        \item We try to find a $f(x;w)$ which predicts \textbf{posterior} probabilities $P(y=1|x)$.
    \end{itemize}
    
\end{frame}
%%%% 5.1 %%%%%%%%%
\begin{frame}{Basics Cont.}
    \begin{itemize}
        \item $f(x;w)$: probability that $y=1$ given $x$ (parameterized by \textbf{$\textbf{w}$})
      \begin{align*}
        P(y=1|x,\mathbf{w}) &= f(x;\mathbf{w}) \\
        P(y=0|x,\mathbf{w}) &= 1 - f(x;\mathbf{w})
      \end{align*}

        \item We need to look for a function which gives us an output in range [0, 1]. (like a probability).

        \item Let's note this function with $\sigma (.)$ and name it \textbf{activation function}.
        
    \end{itemize}
\end{frame}

%%%% 6 %%%%%
\begin{frame}{Basics Cont.}
    \begin{minipage}{0.55\textwidth}
    \begin{itemize}
        \item Sigmoid (logistic) function.
        \[
            \sigma (z) = \frac{1}{1 + e^{-z}}
        \]
        \item A good candidate for activation function.
        
        \item It gives us a number between 0 and 1 \textbf{smoothly}.
        \item It is also \textbf{differentiable}

    \end{itemize}
    \end{minipage}% 
    \begin{minipage}{0.4\textwidth}
        \centering
        \includegraphics[width=0.8\textwidth]{pic/sigmoid.png}
    \end{minipage}
\end{frame}
%%%% 7 %%%%
% \begin{frame}{Basics Cont.}
%     \begin{itemize}
%       \item Sigmoid function plot
%       \begin{figure}[h]
    
%       % \caption{Sigmoid plot}
%     % \label{fig:image}
%     \end{figure}
%     \end{itemize}
% \end{frame}
%%%% 8 %%%%%
\begin{frame}{Basics Cont.}
    \begin{itemize}
      \item The sigmoid function takes a number as input but we have:
    \end{itemize}
        \begin{align*}
            x &= [x_0=1,x_1, \dots, x_d] \\
            w &= [w_0, w_1, \dots, w_d]
        \end{align*}
    \begin{itemize} 
      \item So we can use the \textbf{dot product} of $x$ and $w$.
      
      \item We have $f(x;\mathbf{w}) = \sigma (\mathbf{w}^Tx)$, hence $0\leq f(x;\mathbf{w}) \leq 1$. which is the estimated probability of $y=1$ on input $x$.

      \item An Example : A basketball game (Win, Lose)
        \begin{itemize}
            \item $f(x;\mathbf{w}) = 0.7$
            \item In other terms $70$ percent chance of winning the game.
        \end{itemize}
        
    \end{itemize}
\end{frame}
%%%%% 9 %%%%%%%
\subsection{Decision Surface}
\begin{frame}{Decision Surface}
    \begin{itemize}
      \item Decision surface or decision boundary is the region of a problem space in which the output label of a classifier is ambiguous.
      \item In binary classification it is where the probability of a sample belonging to each $y=0$ and $y=1$ is equal.
      \item Decision boundary hyperplane has always 
      \textbf{one less dimension} than the feature space.
      
      
      %%% next slide
    %   \item an example of decision boundaries:
      
     
    \end{itemize}
\end{frame}
%%%%% 9.4 %%%%%%
\begin{frame}{Decision Surface Cont.}

    \begin{itemize}
        \item An example of decision boundaries:
    \end{itemize}
    \begin{center}
        \includegraphics[width=0.85\textwidth]{pic/DBoundary.png}
        \captionof{figure}{\footnotesize [Eric Xing, Machine Learning, CMU]}
    \end{center}
\end{frame}
%%%% 9.5 %%%%%

\begin{frame}{Decision Surface Cont.}
    \begin{itemize}
      \item Back to our logistic regression problem.
      \item Decision surface $f(x;w) = $ \textbf{constant}.
        \[
            f(x;\mathbf{w})=\sigma (\mathbf{w}^Tx) = \frac{1}{1 + e^{-(\mathbf{w}^Tx)}} = 0.5
        \]
      \item Decision surfaces are \textbf{linear functions} of $x$
        \begin{itemize}
            \item if $f(x;\mathbf{w}) \geq 0.5$ then $\hat{y}=1$, else $\hat{y} = 0$
            \item Equivalently, if $\mathbf{w}^Tx + w_0 \geq 0.5$ then decide $\hat{y}=1$, else $\hat{y}=0$
        \end{itemize}% ayoub eshgh
        \vfill
        \begin{center}
            \( \hat{y} \) \textbf{is the predicted label}
        \end{center}
    \end{itemize}
\end{frame}

%%%% 10 %%%%

\subsection{ML Estimation}

\begin{frame}{ML Estimation}
    \begin{itemize}
        \item We had posterior of a sample as:
        \begin{align*}
            p(y^{(i)}|x^{(i)},\mathbf{w})
        \end{align*}
        \item Logistic Regressions should maximize production of all these sample posteriors.
        
        \item Maximum (conditional) log likelihood:
        \begin{align*}
             \mathbf{\hat{w}} = \underset{w}{\arg\max} \hspace{0.2cm} \log \prod_{i=1}^{n} p(y^{(i)}|x^{(i)},\mathbf{w})
        \end{align*}
        \item Note that in \textbf{binary} classification $y$ is either $1$ or $0$, So we can have posterior term simplified as follows:
        
        \begin{align*}
            p(y^{(i)}|x^{(i)},\mathbf{w})=f(x^{(i)}; \mathbf{w})^{y^{(i)}} (1 - f(x^{(i)}; \mathbf{w}))^{(1 - y^{(i)})}
        \end{align*}
    \end{itemize}
\end{frame}
%%%%%% 10.5 %%%%%%%%%%
\begin{frame}{ML Estimation}
    \begin{itemize}
        \item Log of posterior probability:
            \[
            \log p(y^{(i)}|x^{(i)}, \mathbf{x})=
            y^{(i)}\log (f(x^{(i)}; \mathbf{w})) + 
            (1-y^{(i)})\log (1 - f(x^{(i)}; \mathbf{w}))
            \]
        \item Hence the log likelihood is as follows:
    \end{itemize}
    \[
    \log \prod _{i=1}^{n} p(y^{(i)}|x^{(i)}, \mathbf{w}) = \sum  _{i=1}^{n} \log p(y^{(i)}|x^{(i)}, \mathbf{w})
    \]
    \[
    = \sum_{i=1}^{n}[y^{(i)}\log (f(x^{(i)}; \mathbf{w})) + 
    (1-y^{(i)})\log (1 - f(x^{(i)}; \mathbf{w}))]
    \]
\end{frame}


%%%%% 11 %%%%%
\subsection{Cost Function}
\begin{frame}{Cost Function}
    \begin{itemize}
    \item We should find 
        \begin{align*}
            \mathbf{\hat{w}} = \underset{w}{\arg\min} \hspace{0.2cm} J(w)
        \end{align*}
        \item MLE finds parameters that best describes a classification problem so cost function should be negative of log likelihood term:
        \begin{align*}
            J(w) &= -\sum_{i=1}^{n} \log p(y^{(i)}|wx^{(i)}, \mathbf{w})\\
            &= \sum_{i=1}^{n}-y^{(i)}\log (f(x^{(i)}; \mathbf{w})) - 
            (1-y^{(i)})\log (1 - f(x^{(i)}; \mathbf{w}))
        \end{align*}
        \item No closed form solution for $\nabla _w J(w) = 0$
        \item However $J(w)$ is \textbf{convex}.
    \end{itemize}
\end{frame}

%%%%% 11.5 %%%%%%%%%%
\begin{frame}{Cost Function Cont.}
    \begin{itemize}
    \item Convexity of $J(w)$ can easily be proved:
        \begin{itemize}
            \item We use the lemma that sum of several convex functions is still convex (you can prove it on you own).
            \item Each term in the summation is differentiable (twice).
            \item If you twice get derivative of (with respect to $f$):
                \[
                    -y^{(i)}\log (f(x^{(i)}; \mathbf{w})) - 
            (1-y^{(i)})\log (1 - f(x^{(i)}; \mathbf{w}))
                \]
            \item You get:
                \[
                    \frac{y}{f^2} + \frac{1-y}{(1-f)^2}
                \]
            \item Which for both $y=0$ and $y=1$ is positive.
            \item Each $\log p(y^{(i)}|x^{(i)}, \mathbf{w})$ is convex, hence the summation is convex as well.
        \end{itemize}
    \end{itemize}
\end{frame}
%%%%% 11.7 %%%%%%
\begin{frame}{Cost Function Cont.}
    \begin{itemize}
    \item Visualization of each binary cross entropy loss term ($p$ is our prediction or $f(x;w)$):
    \end{itemize}
    \begin{center}
        \includegraphics[width=0.7\textwidth]{pic/BCE.png}
        % \captionof{figure}{\footnotesize [Eric Xing, Machine Learning, CMU]}
    \end{center}
    %% https://medium.com/@shrividya.gs/log-loss-penalty-for-overconfidence-ce8cb540eb45
    \vfill
    \begin{tikzpicture}[remember picture,overlay]
    \node[anchor=south west, xshift=0.1cm, yshift=0.22cm] at (current page.south west) {
        \scriptsize Figure adopted from medium.com/@shrividya.gs/log-loss-penalty-for-overconfidence-ce8cb540eb45
    };
    \end{tikzpicture}
\end{frame}

%%%% 12 %%%%%%

\subsection{Gradient Descent}
\begin{frame}{Gradient Descent}
    \begin{itemize}
    \item Remember from previous slides:
        \[
        J(w) = \sum_{i=1}^{n}-y^{(i)}\log (f(x^{(i)}; \mathbf{w})) - 
            (1-y^{(i)})\log (1 - f(x^{(i)}; \mathbf{w}))
        \]
    \item Update rule for \textbf{gradient descent}: 
        \begin{align*}
            w^{t+1} = w^t - \eta \nabla _w J(w^t)
        \end{align*}
    \item With $J(w)$ definition for logistic regression we get:
        \begin{align*}
            \nabla _w J(w) = \sum_{i=1}^{n} (f(x^{(i)}; \mathbf{w}) - y^{(i)})x^{(i)} 
        \end{align*}
    \item Also keep in mind $f(x^{(i)}; \mathbf{w})= \sigma (\mathbf{w}^Tx^{(i)})$
    % \item Compare with the gradient of \textbf{SSE} in \textbf{linear regression} :
    %     \begin{align*}
    %         \nabla _w J(w) = \sum_{i=1}^{n} (w^Tx^{(i)} - y^{(i)})x^{(i)}
    %     \end{align*}
        
    \end{itemize}
\end{frame}
%%%%%% 12.5 %%%%%%%
\begin{frame}{Gradient Descent}
    \begin{itemize}
    
    \item Compare the gradient of \textcolor{deepgreen}{logistic regression} with the gradient of \textcolor{blue}{SSE} in \textcolor{blue}{linear regression} :
    \end{itemize}
        \begin{align*}
        \color{deepgreen}
             \nabla _w J(w) = \sum_{i=1}^{n} (\sigma (\mathbf{w}^Tx^{(i)}) - y^{(i)})x^{(i)} 
        \end{align*}
        \begin{align*}
            \color{blue}
            \nabla _w J(w) = \sum_{i=1}^{n} (\mathbf{w}^Tx^{(i)} - y^{(i)})x^{(i)}
        \end{align*}
        
\end{frame}

%%%%% 13 %%%%%%%%

\begin{frame}{Loss Function}
    \begin{itemize}
        %% delete paranthesis
        \item Loss function is a single overall measure of loss incurred for taking our decisions (over entire dataset).
        \item We have:
        \[ Loss(y, f(x; \mathbf{w})) = -y \times \log (f(x; \mathbf{w})) - (1-y) \times \log 
        (1 - f(x; \mathbf{w}))
        \]
        
        \item Since in binary classification either $y=1$ or $y=0$ we have:
            \[
                Loss(y, f(x; \mathbf{w})) = \begin{cases}
                    - \log (f(x; \mathbf{w})) & \textbf{if } y = 1 \\
                    - \log (1 - f(x; \mathbf{w})) & \textbf{if } y = 0
                \end{cases}
            \]
        \item How is it related to zero-one loss? ($\hat{y}$ is the predicted and $y$ is the ture label)
           \[
               Loss(y, \hat{y}) =  \begin{cases}
                    1 & \textbf{if } y \neq \hat{y} \\
                    0 & \textbf{if } y = \hat{y}
                \end{cases}
            \]
    \end{itemize}
\end{frame}

%%%%% 14 %%%%%%
\begin{frame}{Cost Function Summary}
    \begin{itemize}
        \item Logistic Regression (LR) has a more proper cost function for classification than SSE and Perceptron.

        \item Why is the cost function of LR also more suitable than 
            \begin{align*}
                J(w) = \frac{1}{n}\sum_{i=1}^{n}(y^{(i)} - f(x^{(i)}; \mathbf{w}))^2
            \end{align*}
        Where $f(x; \mathbf{w}) = \sigma(\mathbf{w}^Tx)$?
            \begin{itemize}
                \item The conditional distribution $p(y|x, \mathbf{w})$ in the classification problem is not Guassian (it is \textbf{Bernoulli}).
                \item The cost function of LR is also convex.
            \end{itemize}
    \end{itemize}
\end{frame}

%%%%%% 15 %%%%%%%

\subsection{Multi-class logistic regression}
\begin{frame}{Multi-class logistic regression}
    \begin{itemize}
        \item Now consider a problem where we have $K$ classes and every sample only belongs to one class (for simplicity).
        \item For each class $k$, $f_k(x; \mathbf{W})$ predicts the probability of $y=k$.
            \begin{itemize}
                \item i.e., $P(y=k|x, \mathbf{W})$
            \end{itemize}
        \item For each data point $x_0$, $\sum _{k=1}^{K} p(y=k|x_0, \mathbf{W})$ must be $1$
            \begin{itemize}
                \item $W$ denotes a matrix of $w_i$'s in which each $w_i$ is a weight vector dedicated for class label $i$.
            \end{itemize}
        \item On a new input $x$, to make a prediction, we pick the class that maximizes $f_k(x; \mathbf{W})$:
            \begin{align*}
                \alpha (x) &= \underset{k=1, \dots , K}{\arg\max} \hspace{0.2cm} f_k(x) 
            \end{align*}
            \begin{center}
                \textbf{if $\color{red} f_k(x) > f_j(x)$ $\color{red} \forall j \neq k$ then decide $\color{red} C_k$}
            \end{center}
    \end{itemize}
\end{frame}

%%%%% 16 %%%%%%%
\begin{frame}{Multi-class logistic regression Cont.}
    \begin{itemize}
        \item $K > 2$ and $y \in \{1,2,\dots,K\}$
        \begin{align*}
            f_k(x, \mathbf{W}) = p(y=k|x) = \frac{\exp{(w^T_kx)}}{\sum_{j=1}^{K}\exp{(w_j^Tx)}}
        \end{align*}
        \item Normalized exponential (Aka \textbf{Softmax})
          
            \item if $w_k^Tx \gg w_j^Tx$ for all $j \neq k$ then $p(C_k|x) \approx 1$ and $p(C_j|x) \approx 0$
            \item Note : remember from Bayes theorem:
                \[
                p(C_k|x) = \frac{p(x|C_k)p(C_k)}
            {\sum_{j=1}^{K}p(x|C_j)p(C_j)}
                \]
          
    \end{itemize}
\end{frame}
%%%%% 16.5 %%%%%
\begin{frame}{Multi-class logistic regression Cont.}
    \begin{itemize}
        \item Softmax function \textbf{smoothly} highlights the maximum probability and is differentiable.
        \item Compare it with $\max (.)$ function which is strict and non-differentiable
        \item Softmax can also handle negative values because we are using exponential function
        \item And it gives us probability for each class since:
            \begin{align*}
                \displaystyle \sum _{k=1}^{K} \frac{\exp (w_k^Tx)}{\sum _{j=1}^{K} \exp (w_j^Tx) } = 1
            \end{align*}
    \end{itemize}
\end{frame}

%%%% 16.6 %%%%%
\begin{frame}{Multi-class logistic regression Cont.}
    \begin{itemize}
        \item An example of applying softmax (note that $z_i=w^Tx_i$):
    \end{itemize}
    \begin{center}
        \includegraphics[width=0.7\textwidth]{pic/softmax0.png}
       
    \end{center}
\end{frame}

%%%% 17 %%%%%
\begin{frame}{Multi-class logistic regression Cont.}
    \begin{itemize}
        \item Again we set $J(W)$ as negative of log likelihood.
        \item We need $\hat{W} = \underset{W}{\arg\min} \hspace{0.2cm} J(W)$
        \begin{align*}
            J(W) &= -\log \prod_{i=1}^{n} \textcolor{red}{p(y^{(i)}|x^{(i)}, \mathbf{W})} \\
            &= -\log \prod_{i=1}^{n}\textcolor{red}{\prod_{k=1}^{K}f_k(x^{(i)}; \mathbf{W})^{y_k^{(i)}} }\\
            &= -\sum_{i=1}^{n}\sum_{k=1}^{K}y_k^{(i)} \log (f_k(x^{(i)}; \mathbf{W}))
        \end{align*}
        \item If \textbf{$\textbf{i}$-th} sample belongs to class $k$ then $y^{(i)}_k$ is 1 else 0.
        \item Again no closed-from solution for $\hat{W}$
    \end{itemize}
    
\end{frame}
%%%%%%% 17.5 %%%%%%%
\begin{frame}{Multi-class logistic regression Cont.}
    \begin{itemize}
        \item From previous slides we have:
            \begin{align*}
                J(W) = -\sum_{i=1}^{n}\sum_{k=1}^{K}y_k^{(i)} \log (f_k(x^{(i)}; \mathbf{W}))
            \end{align*}
        \item In which:
             \[
        W = [w_1,w_2, \dots, w_K], \quad Y = 
        \text{\small $
        \begin{pmatrix}
            y^{(1)} \\
            y^{(2)} \\
            \vdots \\
            y^{(n)}
        \end{pmatrix}
        $}
        =
        \begin{pmatrix}
            y_1^{(1)} & \dots & y_K^{(1)} \\
            y_1^{(2)} & \dots & y_K^{(2)} \\
            \vdots    & \ddots & \vdots \\
            y_1^{(n)} & \dots & y_K^{(n)}
        \end{pmatrix}
    \]
    
        \item $y$ is a vector of length $K$ (1-of-$K$ encoding)
            \begin{itemize}
                % \item Each vector $y^{(i)}$ only consists of zeros and ones.
                \item For example $y=[0,0,1,0]^T$ when the target class is $C_3$.
            \end{itemize}
    \end{itemize}
\end{frame}
%%%%%% 18 %%%%%%%%
\begin{frame}{Multi-class logistic regression Cont.}
    \begin{itemize}
        \item Update rule for gradient descent:
        
    \end{itemize}
    \begin{align*}
            w_j^{t+1} &= w_j^t - \eta \nabla _W J(W^t) \\
            \nabla _{w_{j}} J(W) &= \sum_{i=1}^{n} (f_j(x^{(i)}; \mathbf{W}) - y_j^{(i)})x^{(i)}
        \end{align*}
        
    \begin{itemize}
        \item $w_j^t$ denotes the weight vector for class $j$ (since in multi-class LR each class has its own weight vector) in the $t$-th iteration
    \end{itemize}
\end{frame}


\section{Summary}

\begin{frame}{Logistic Regression (LR) Summary}
    \begin{itemize}
        \item LR is a \textbf{linear} classifier
        \item LR optimization problem is obtained by \textbf{maximum likelihood}
            \begin{itemize}
                \item when assuming \textbf{Bernoulli} distribution for conditional probabilities whose mean is $\frac{1}{1 + e^{-(w^Tx)}}$
            \end{itemize}
        \item No closed-form solution for its optimization problem
            \begin{itemize}
                \item But convex cost function and global optimum can be found by gradient ascent
            \end{itemize}
    \end{itemize}
\end{frame}


% \section{Methods}

% \subsection{Diffusion Model}

% \begin{frame}{Title}
%     \begin{itemize}
%         \item \lipsum[3][1-4]
%     \end{itemize}
%     \begin{table}[h]
%         \centering
%         \begin{tabular}{c|c}
%             Microsoft\textsuperscript{\textregistered}  Windows & Apple\textsuperscript{\textregistered}  Mac OS \\
%             \hline
%             Windows-Kernel & Unix-like \\
%             Arm, Intel & Intel, Apple Silicon \\
%             Sudden update & Stable update \\
%             Less security & More security \\
%             ... & ... \\
%         \end{tabular}
%     \end{table}
% \end{frame}

% \begin{frame}{Algorithms}
%     \begin{exampleblock}{Non-Numbering Formula}
%         \begin{equation*}
%             J(\theta) = \mathbb{E}_{\pi_\theta}[G_t] = \sum_{s\in\mathcal{S}} d^\pi (s)V^\pi(s)=\sum_{s\in\mathcal{S}} d^\pi(s)\sum_{a\in\mathcal{A}}\pi_\theta(a|s)Q^\pi(s,a)
%         \end{equation*}
%     \end{exampleblock}
%     \begin{exampleblock}{Multi-Row Formula\footnote{If text appears in the formula,use $\backslash$mathrm\{\} or $\backslash$text\{\} instead}}
%         \begin{align}
%             Q_\mathrm{target}&=r+\gamma Q^\pi(s^\prime, \pi_\theta(s^\prime)+\epsilon)\\
%             \epsilon&\sim\mathrm{clip}(\mathcal{N}(0, \sigma), -c, c)\nonumber
%         \end{align}
%     \end{exampleblock}
% \end{frame}

% \begin{frame}
%     \begin{exampleblock}{Numbered Multi-line Formula}
%         % Taken from Mathmode.tex
%         \begin{multline}
%             A=\lim_{n\rightarrow\infty}\Delta x\left(a^{2}+\left(a^{2}+2a\Delta x+\left(\Delta x\right)^{2}\right)\right.\label{eq:reset}\\
%             +\left(a^{2}+2\cdot2a\Delta x+2^{2}\left(\Delta x\right)^{2}\right)\\
%             +\left(a^{2}+2\cdot3a\Delta x+3^{2}\left(\Delta x\right)^{2}\right)\\
%             +\ldots\\
%             \left.+\left(a^{2}+2\cdot(n-1)a\Delta x+(n-1)^{2}\left(\Delta x\right)^{2}\right)\right)\\
%             =\frac{1}{3}\left(b^{3}-a^{3}\right)
%         \end{multline}
%     \end{exampleblock}
% \end{frame}

% \begin{frame}{Graphics and Columns}
%     \begin{minipage}[c]{0.3\linewidth}
%         \psset{unit=0.8cm}
%         \begin{pspicture}(-1.75,-3)(3.25,4)
%             \psline[linewidth=0.25pt](0,0)(0,4)
%             \rput[tl]{0}(0.2,2){$\vec e_z$}
%             \rput[tr]{0}(-0.9,1.4){$\vec e$}
%             \rput[tl]{0}(2.8,-1.1){$\vec C_{ptm{ext}}$}
%             \rput[br]{0}(-0.3,2.1){$\theta$}
%             \rput{25}(0,0){%
%             \psframe[fillstyle=solid,fillcolor=lightgray,linewidth=.8pt](-0.1,-3.2)(0.1,0)}
%             \rput{25}(0,0){%
%             \psellipse[fillstyle=solid,fillcolor=yellow,linewidth=3pt](0,0)(1.5,0.5)}
%             \rput{25}(0,0){%
%             \psframe[fillstyle=solid,fillcolor=lightgray,linewidth=.8pt](-0.1,0)(0.1,3.2)}
%             \rput{25}(0,0){\psline[linecolor=red,linewidth=1.5pt]{->}(0,0)(0.,2)}
% %           \psRotation{0}(0,3.5){$\dot\phi$}
% %           \psRotation{25}(-1.2,2.6){$\dot\psi$}
%             \psline[linecolor=red,linewidth=1.25pt]{->}(0,0)(0,2)
%             \psline[linecolor=red,linewidth=1.25pt]{->}(0,0)(3,-1)
%             \psline[linecolor=red,linewidth=1.25pt]{->}(0,0)(2.85,-0.95)
%             \psarc{->}{2.1}{90}{112.5}
%             \rput[bl](.1,.01){C}
%         \end{pspicture}
%     \end{minipage}\hspace{2cm}
%     \begin{minipage}{0.5\linewidth}
%         \medskip
%         % \hspace{2cm}
%         \begin{figure}[h]
%             \centering
%             \includegraphics[height=.4\textheight]{pic/sample.pdf}
%         \end{figure}
%     \end{minipage}
% \end{frame}

% \begin{frame}[fragile]{\LaTeX{} Common Commands}
%     \begin{exampleblock}{Commands}
%         \centering
%         \footnotesize
%         \begin{tabular}{llll}
%             \cmd{chapter} & \cmd{section} & \cmd{subsection} & \cmd{paragraph} \\
%             chapter & section & sub-section & paragraph \\\hline
%             \cmd{centering} & \cmd{emph} & \cmd{verb} & \cmd{url} \\
%             center & emphasize & original & hyperlink \\\hline
%             \cmd{footnote} & \cmd{item} & \cmd{caption} & \cmd{includegraphics} \\
%             footnote & list item & caption & insert image \\\hline
%             \cmd{label} & \cmd{cite} & \cmd{ref} \\
%             label & citation & refer\\\hline
%         \end{tabular}
%     \end{exampleblock}
%     \begin{exampleblock}{Environment}
%         \centering
%         \footnotesize
%         \begin{tabular}{lll}
%             \env{table} & \env{figure} & \env{equation}\\
%             table & figure & formula \\\hline
%             \env{itemize} & \env{enumerate} & \env{description}\\
%             non-numbering item & numbering item & description \\\hline
%         \end{tabular}
%     \end{exampleblock}
% \end{frame}

% \begin{frame}[fragile]{\LaTeX{} Examples of environmental commands}
%     \begin{minipage}{0.5\linewidth}
% \begin{lstlisting}[language=TeX]
% \begin{itemize}
%   \item A \item B
%   \item C
%   \begin{itemize}
%     \item C-1
%   \end{itemize}
% \end{itemize}
% \end{lstlisting}
%     \end{minipage}\hspace{1cm}
%     \begin{minipage}{0.3\linewidth}
%         \begin{itemize}
%             \item A
%             \item B
%             \item C
%             \begin{itemize}
%                 \item C-1
%             \end{itemize}
%         \end{itemize}
%     \end{minipage}
%     \medskip
%     \pause
%     \begin{minipage}{0.5\linewidth}
% \begin{lstlisting}[language=TeX]
% \begin{enumerate}
%   \item A \item B
%   \item C
%   \begin{itemize}
%     \item[n+e]
%   \end{itemize}
% \end{enumerate}
% \end{lstlisting}
%     \end{minipage}\hspace{1cm}
%     \begin{minipage}{0.3\linewidth}
%         \begin{enumerate}
%             \item A
%             \item B
%             \item C
%             \begin{itemize}
%                 \item[n+e]
%             \end{itemize}
%         \end{enumerate}
%     \end{minipage}
% \end{frame}

% \begin{frame}[fragile]{\LaTeX{} Formulas}
%     \begin{columns}
%         \begin{column}{.55\textwidth}
% \begin{lstlisting}[language=TeX]
% $V = \frac{4}{3}\pi r^3$

% \[
%   V = \frac{4}{3}\pi r^3
% \]

% \begin{equation}
%   \label{eq:vsphere}
%   V = \frac{4}{3}\pi r^3
% \end{equation}
% \end{lstlisting}
%         \end{column}
%         \begin{column}{.4\textwidth}
%             $V = \frac{4}{3}\pi r^3$
%             \[
%                 V = \frac{4}{3}\pi r^3
%             \]
%             \begin{equation}
%                 \label{eq:vsphere}
%                 V = \frac{4}{3}\pi r^3
%             \end{equation}
%         \end{column}
%     \end{columns}
%     \begin{itemize}
%         \item more information \href{https://ja.overleaf.com/learn/latex/Mathematical_expressions}{\color{purple}{here}}
%     \end{itemize}
% \end{frame}

% \begin{frame}[fragile]
%     \begin{columns}
%         \column{.6\textwidth}
% \begin{lstlisting}[language=TeX]
% \begin{table}[htbp]
%   \caption{numbers & meaning}
%   \label{tab:number}
%   \centering
%   \begin{tabular}{cl}
%     \toprule
%     number & meaning \\
%     \midrule
%     1 & 4.0 \\
%     2 & 3.7 \\
%     \bottomrule
%   \end{tabular}
% \end{table}
% \end{lstlisting}
%         \column{.4\textwidth}
%         \begin{table}[htpb]
%             \centering
%             \caption{numbers \& meaning}
%             \label{tab:number}
%             \begin{tabular}{cl}\toprule
%                 numbers & meaning \\\midrule
%                 1 & 4.0\\
%                 2 & 3.7\\\bottomrule
%             \end{tabular}
%         \end{table}
%         \normalsize formula~(\ref{eq:vsphere}) at previous slide and Table~\ref{tab:number}。
%     \end{columns}
% \end{frame}

% \section{Results}
% \begin{frame}
%     \begin{itemize}
%         \item \lipsum[4][1-4]
%         \item \lipsum[4][5-9]
%         \item \lipsum[5][1-4]
%         \item \lipsum[5][5-8]
%     \end{itemize}
% \end{frame}

\section{References}

\begin{frame}[allowframebreaks]
    \bibliography{ref}
    \bibliographystyle{ieeetr}
    [1]. Dr.Soleymani Baghshah, Machine Learning course, Sharif University of Technology.

    [2]. C. Bishop, “Pattern Recognition and Machine Learning”, Chapter 4.2-4.3.

    [3]. Sanja Fidler’s Slides, University of Toronto, CSC411
    % [2]. Andrew Ng and Tenguya Ma, CS299 Lecture Notes
    % \nocite{*} % used here because no citation happens in slides
    % if there are too many try use:
    % \tiny\bibliographystyle{alpha}
\end{frame}


\begin{frame}
    \begin{center}
        {\Huge Any Questions?}
    \end{center}
\end{frame}

\end{document}