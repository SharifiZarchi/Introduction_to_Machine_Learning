%%%%%%%%%%%%%%%%%%%%%%%%%%%%%%%%%%%%%%%%%%%%%%%%%%%%%%
% A Beamer template for University of Wollongong     %
% Based on THU beamer theme                          %
% Author: Qiuyu Lu                                   %
% Date: July 2024                                    %
% LPPL Licensed.                                     %
%%%%%%%%%%%%%%%%%%%%%%%%%%%%%%%%%%%%%%%%%%%%%%%%%%%%%%
% Customized for Sharif University of Technology     %
%%%%%%%%%%%%%%%%%%%%%%%%%%%%%%%%%%%%%%%%%%%%%%%%%%%%%%


\documentclass[serif, aspectratio=169]{beamer}
%\documentclass[serif]{beamer}  % for 4:3 ratio
\usepackage[T1]{fontenc}
\usepackage{fourier} % see "http://faq.ktug.org/wiki/uploads/MathFonts.pdf" for other options
\usepackage{hyperref}
\usepackage{latexsym,amsmath,xcolor,multicol,booktabs,calligra}
\usepackage{graphicx,pstricks,listings,stackengine}
\usepackage{lipsum}
\usepackage[normalem]{ulem}
\usepackage{caption}
\usepackage{tikz}

\author{Ali Sharifi-Zarchi}
\title{Machine Learning (CE 40717)}
\subtitle{Fall 2025}
\institute{
    CE Department \\
    Sharif University of Technology
}
%\date{\small \today}
% \usepackage{UoWstyle}
\usepackage{SUTstyle}

% defs
\def\cmd#1{\texttt{\color{red}\footnotesize $\backslash$#1}}
\def\env#1{\texttt{\color{blue}\footnotesize #1}}
\definecolor{deepblue}{rgb}{0,0,0.5}
\definecolor{deepred}{RGB}{153,0,0}
\definecolor{deepgreen}{rgb}{0,0.5,0}
\definecolor{halfgray}{gray}{0.55}

\lstset{
    basicstyle=\ttfamily\small,
    keywordstyle=\bfseries\color{deepblue},
    emphstyle=\ttfamily\color{deepred},    % Custom highlighting style
    stringstyle=\color{deepgreen},
    numbers=left,
    numberstyle=\small\color{halfgray},
    rulesepcolor=\color{red!20!green!20!blue!20},
    frame=shadowbox,
}
\captionsetup{labelformat=empty}

\begin{document}

    \begin{frame}
        \titlepage
        \vspace*{-0.6cm}
        \begin{figure}[htpb]
            \begin{center}
                \includegraphics[keepaspectratio, scale=0.25]{pic/sharif-main-logo.png}
            \end{center}
        \end{figure}
        \vfill % This pushes the next content to the bottom
        \vspace{-0.35cm}
        % \centering\textit{\tiny Most slides are adapted from Dr. Mahdie Soleymani's ML course}
    \end{frame}

    \begin{frame}
        \tableofcontents[sectionstyle=show,
            subsectionstyle=show/shaded/hide,
            subsubsectionstyle=show/shaded/hide]
    \end{frame}


    \section{Introduction}

    \begin{frame}{Binary Classification Problem}
        \begin{itemize}
            \item Consider a \textbf{binary classification} task:
            \begin{itemize}
                \item Email classification: Spam / Not Spam
                \item Online transactions: Fraudulent / Genuine
                \item Tumor diagnosis: Malignant / Benign
            \end{itemize}
        \end{itemize}

        \vspace{0.5em}
        Define the target variable formally:
        \[
            y \in \{0, 1\}, \quad
            \begin{cases}
                0 & \text{Negative class (e.g., benign tumor)}\\
                1 & \text{Positive class (e.g., malignant tumor)}
            \end{cases}
        \]
    \end{frame}

    \begin{frame}{Linear Regression for Classification}
        \begin{itemize}
            \item A natural approach is to use linear regression:
            \[
                h_\theta(x) = \theta_0 + \theta_1 x
            \]
            and define a threshold at $0.5$ for prediction:
            \[
                \hat{y} =
                \begin{cases}
                    1, & h_\theta(x) \ge 0.5\\
                    0, & h_\theta(x) < 0.5
                \end{cases}
            \]
        \end{itemize}

        \vspace{0.5em}
        \begin{minipage}{0.48\linewidth}
            \centering
            \includegraphics[width=\linewidth]{pic/lrClassification1.png}
        \end{minipage}
        \hfill
        \begin{minipage}{0.48\linewidth}
            \centering
            \includegraphics[width=\linewidth]{pic/lrClassification3.png}
        \end{minipage}
    \end{frame}

    \begin{frame}{Limitations of Linear Regression for Classification}
        \begin{itemize}
            \item Addition of new data points can yield predictions outside $[0,1]$.
            \item Linear regression does not provide probabilistic outputs.
            \item The decision boundary may be highly sensitive to outliers.
        \end{itemize}

        \vspace{0.5em}
        \centering
        \includegraphics[width=0.68\linewidth]{pic/lrClassification2.png}

        \[
            \text{Requirement: } 0 \le h_\theta(x) \le 1
        \]

        \[
            \text{Logistic regression: } f(x;\mathbf{w}) = \sigma(\mathbf{w}^\top x), \quad \sigma(z) = \frac{1}{1+e^{-z}}
        \]
    \end{frame}




    \section{Logistic Regression}

    \subsection{Fundamentals}


%%%% 5 %%%%%
    \begin{frame}{Introduction}
        % \begin{itemize}
        %     \item Logistic regression is a \textbf{discriminative} approach.
        % \end{itemize}
        % \begin{figure}[h]
        %   \centering
        %   \includegraphics[width=0.6\textwidth]{pic/logisticR-Linear.png}
        %   % \label{fig:image}
        % \end{figure}
        \begin{itemize}
            \item Suppose we have a binary classification task (so $K=2$).
            \item By observing \textcolor{deepred}{age}, \textcolor{deepred}{gender}, \textcolor{deepred}{height}, \textcolor{deepred}{weight} and \textcolor{deepred}{BMI} we try to distinguish if a person is \textcolor{deepgreen}{overweight} or \textcolor{deepgreen}{not overweight}.


            %%%
            \begin{table}[h!]
                \centering
                \begin{tabular}{|c|c|c|c|c|c|}
                    \hline
                    \textcolor{deepred}{Age} & \textcolor{deepred}{Gender} & \textcolor{deepred}{Height (cm)} & \textcolor{deepred}{Weight (kg)} & \textcolor{deepred}{BMI} & \textcolor{deepgreen}{Overweight} \\ \hline
                    25 & Male & 175 & 80 & 25.3 & 0 \\ \hline
                    30 & Female & 160 & 60 & 22.5 & 0 \\ \hline
                    \multicolumn{6}{|c|}{\dots} \\ \hline
                    35 & Male & 180 & 90 & 27.3 & 1 \\ \hline
                \end{tabular}
                % \caption{Sample Table}
                % \label{tab:sample}
            \end{table}
            %%%%
            \item We denote the \textcolor{deepred}{features} of a sample with vector $x$ and the \textcolor{deepgreen}{label} with $y$.
            \item In logistic regression we try to find an $\sigma (w^Tx)$ which predicts \textbf{posterior} probabilities $P(y=1|x)$.
        \end{itemize}

    \end{frame}
%%%% 5.1 %%%%%%%%%
    \begin{frame}{Introduction (cont.)}
        \begin{itemize}
            \item $\sigma (w^Tx)$: probability that $y=1$ given $x$ (parameterized by \textbf{$\textbf{w}$})
            \begin{align*}
                P(y=1|x,\mathbf{w}) &= \sigma (\mathbf{w}^Tx) \\
                P(y=0|x,\mathbf{w}) &= 1 - \sigma (\mathbf{w}^Tx)
            \end{align*}

            \item We need to look for a function which gives us an output in the range [0, 1]. (like a probability).

            \item Let's denote this function with $\sigma (.)$ and call it the \textbf{activation function}.

        \end{itemize}
    \end{frame}

%%%% 6 %%%%%
    \begin{frame}{Introduction (cont.)}
        \begin{minipage}{0.55\textwidth}
            \begin{itemize}
                \item Sigmoid (logistic) function.
                \[
                    \sigma (z) = \frac{1}{1 + e^{-z}}
                \]
                \item A good candidate for activation function.

                \item It gives us a number between 0 and 1 \textbf{smoothly}.
                \item It is also \textbf{differentiable}

            \end{itemize}
        \end{minipage}%
        \begin{minipage}{0.4\textwidth}
            \centering
            \includegraphics[width=0.8\textwidth]{pic/sigmoid.png}
        \end{minipage}
    \end{frame}
%%%%%%%%%%%%%%%%%%%%%%%%%%%%%%%%%%%%%%%%
    \begin{frame}{Sigmoid function \& its derivative}
        \centering
        \includegraphics[width=0.65\textwidth]{pic/sigmoidDer.png}
    \end{frame}

%%%%%%%%%%%%%%%%%%%%%%%%%%%%%%%%%%%%%%%%
%%%% 7 %%%%
% \begin{frame}{Basics Cont.}
%     \begin{itemize}
%       \item Sigmoid function plot
%       \begin{figure}[h]

%       % \caption{Sigmoid plot}
%     % \label{fig:image}
%     \end{figure}
%     \end{itemize}
% \end{frame}
%%%% 8 %%%%%
    \begin{frame}{Introduction (cont.)}
        \begin{itemize}
            \item The sigmoid function takes a number as input but we have:
        \end{itemize}
        \begin{align*}
            x &= [x_0=1,x_1, \dots, x_d] \\
            w &= [w_0, w_1, \dots, w_d]
        \end{align*}
        \begin{itemize}
            \item So we can use the \textbf{dot product} of $x$ and $w$.

            \item We have $0\leq \sigma (\mathbf{w}^Tx) \leq 1$. which is the estimated probability of $y=1$ on input $x$.

            \item An Example : A basketball game (Win, Lose)
            \begin{itemize}
                \item $\sigma (\mathbf{w}^T x) = 0.7$
                \item In other terms $70$ percent chance of winning the game.
            \end{itemize}

        \end{itemize}
    \end{frame}
%%%%% 9 %%%%%%%
    \subsection{Decision surface}
    \begin{frame}{Decision surface}
        \begin{itemize}
            \item Decision surface or decision boundary is the region of a problem space in which the output label of a classifier is ambiguous. (could be linear or non-linear)
            \item In binary classification it is where the probability of a sample belonging to each $y=0$ and $y=1$ is equal.
        \end{itemize}


        \begin{center}
            \includegraphics[width=0.4\textwidth]{pic/decision boundary.png}
            % \captionof{figure}{\footnotesize [Eric Xing, Machine Learning, CMU]}
        \end{center}

        \begin{itemize}

            \item Decision boundary hyperplane always has \textbf{one less dimension} than the feature space.


            %%% next slide
            %   \item an example of decision boundaries:


        \end{itemize}
    \end{frame}
%%%%% 9.4 %%%%%%
    \begin{frame}{Decision surface (cont.)}

        \begin{itemize}
            \item An example of linear decision boundaries:
        \end{itemize}
        \begin{center}
            \includegraphics[width=0.85\textwidth]{pic/DBoundary.png}
            % \captionof{figure}{\footnotesize [Eric Xing, Machine Learning, CMU]}
        \end{center}
        \begin{tikzpicture}[remember picture,overlay]
            \node[anchor=south west, xshift=0.1cm, yshift=0.22cm] at (current page.south west) {
                \scriptsize Figure adapted from Eric Xing, Machine Learning, CMU
            };
        \end{tikzpicture}
    \end{frame}
%%%% 9.5 %%%%%

    \begin{frame}{Decision surface (cont.)}
        \begin{itemize}
            \item Back to our logistic regression problem.
            \item Decision surface $\sigma (\mathbf{w}^Tx) = $ \textbf{constant}.
            \[
                \sigma (\mathbf{w}^Tx) = \frac{1}{1 + e^{-(\mathbf{w}^Tx)}} = 0.5
            \]
            \item Decision surfaces are \textbf{linear functions} of $x$
            \begin{itemize}
                \item if $\sigma (\mathbf{w}^Tx) \geq 0.5$ then $\hat{y}=1$, else $\hat{y} = 0$
                \item Equivalently, if $\mathbf{w}^Tx + w_0 \geq 0.5$ then decide $\hat{y}=1$, else $\hat{y}=0$
            \end{itemize}% ayoub eshgh
            \vfill
            \begin{center}
                \( \hat{y} \) \textbf{is the predicted label}
            \end{center}
        \end{itemize}
    \end{frame}

%%%% 10 %%%%
    \begin{frame}{Decision boundary example}
        \begin{align*}
            \sigma (\mathbf{w}^Tx) = \sigma (w_0 + w_1 x_1 + w_2 x_2)
        \end{align*}

        \begin{minipage}{0.35\linewidth}
            \centering
            \includegraphics[width=\linewidth]{pic/lrDB1.png}
        \end{minipage}
        \hfill
        \begin{minipage}{0.35\linewidth}
            \centering
            \includegraphics[width=\linewidth]{pic/lrDB2.png}
        \end{minipage}

        \begin{align*}
            \text{Predict } y=1 \text{ if } -3 + x_1 + x_2 \geq 0
        \end{align*}

    \end{frame}
%%%%%%%%%%%%%
    \begin{frame}{Non-linear decision boundary example}
        \begin{align*}
            \sigma (\mathbf{w}^Tx) &= \sigma (w_0 + w_1 x_1 + w_2 x_2 + w_3x_1^2 + w_4x_2^2) \\
            \text{We can learn } & \text{more complex decision boundaries when having higher order terms}
        \end{align*}

        \begin{minipage}{0.30\linewidth}
            \centering
            \includegraphics[width=\linewidth]{pic/lrDB3.png}
        \end{minipage}
        \hfill
        \begin{minipage}{0.30\linewidth}
            \centering
            \includegraphics[width=\linewidth]{pic/lrDB4.png}
        \end{minipage}

        \begin{align*}
            \text{Predict } y=1 \text{ if } -1 + x_1^2 + x_2^2 \geq 0
        \end{align*}

    \end{frame}
%%%%%%%%%%%%%
    \subsection{MLE and MAP}

%%%%%%%%%%%%%%%%%%%%%%%%%%%%%%%%%%%%%%
    \begin{frame}{Maximum Likelihood Estimation (MLE)}
        \begin{itemize}
            \item For $n$ independent samples, the likelihood is:
            \[
                L(\mathbf{w}) = \prod_{i=1}^{n} P(y^{(i)}|x^{(i)}, \mathbf{w})
            \]
            \item For binary classification ($y \in \{0,1\}$):
            \[
                P(y^{(i)}|x^{(i)},\mathbf{w}) = \sigma (\mathbf{w}^Tx^{(i)})^{y^{(i)}} (1 - \sigma (\mathbf{w}^Tx^{(i)}))^{1 - y^{(i)}}
            \]
            \item Log-likelihood:
            \[
                \ell(\mathbf{w}) = \sum_{i=1}^n \big[ y^{(i)} \log \sigma(\mathbf{w}^Tx^{(i)}) + (1-y^{(i)})\log (1-\sigma(\mathbf{w}^Tx^{(i)})) \big]
            \]
        \end{itemize}
    \end{frame}

%%%%%%%%%%%%%%%%%%%%%%%%%%%%%%%%%%%%%%
    \begin{frame}{From Likelihood to Cost Function}
        \begin{itemize}
            \item Maximizing the likelihood $\Leftrightarrow$ minimizing the negative log-likelihood (NLL):
            \[
                J_{\text{MLE}}(\mathbf{w}) = -\ell(\mathbf{w})
            \]
            \item Can be written as an integral over the data distribution:
            \[
                J_{\text{MLE}}(\mathbf{w}) = - \int p(x,y) \log P(y|x, \mathbf{w}) \, dx\,dy
            \]
            \item Empirical estimate (training data):
            \[
                J_{\text{MLE}}(\mathbf{w}) = -\frac{1}{n}\sum_{i=1}^{n} \log P(y^{(i)}|x^{(i)}, \mathbf{w})
            \]
            \item This is exactly the \textbf{cross-entropy loss} used in classification.
        \end{itemize}
    \end{frame}

%%%%%%%%%%%%%%%%%%%%%%%%%%%%%%%%%%%%%%
    \begin{frame}{Maximum A Posteriori Estimation (MAP)}
        \begin{itemize}
            \item MAP incorporates prior knowledge about parameters:
            \[
                \hat{\mathbf{w}}_{\text{MAP}} = \arg\max_{\mathbf{w}} P(\mathbf{w}|D)
            \]
            \item Using Bayes’ rule:
            \[
                P(\mathbf{w}|D) \propto P(D|\mathbf{w})P(\mathbf{w})
            \]
            \item Equivalently:
            \[
                \hat{\mathbf{w}}_{\text{MAP}} = \arg\max_{\mathbf{w}} \Big[ \log P(D|\mathbf{w}) + \log P(\mathbf{w}) \Big]
            \]
            \item Cost function:
            \[
                J_{\text{MAP}}(\mathbf{w}) = - \sum_{i=1}^n \log P(y^{(i)}|x^{(i)}, \mathbf{w}) - \log P(\mathbf{w})
            \]
        \end{itemize}
    \end{frame}

%%%%%%%%%%%%%%%%%%%%%%%%%%%%%%%%%%%%%%
    \begin{frame}{MAP with Gaussian Prior (L2 Regularization)}
        \begin{itemize}
            \item Gaussian prior: $\mathbf{w} \sim \mathcal{N}(0, \tau^2 I)$
            \[
                P(\mathbf{w}) \propto \exp\left(-\frac{\|\mathbf{w}\|^2}{2\tau^2}\right)
            \]
            \item MAP cost function:
            \[
                J_{\text{MAP}}(\mathbf{w}) = J_{\text{MLE}}(\mathbf{w}) + \frac{1}{2\tau^2}\|\mathbf{w}\|^2
            \]
            \item Let $\lambda = 1/\tau^2$:
            \[
                J_{\text{MAP}}(\mathbf{w}) = J_{\text{MLE}}(\mathbf{w}) + \frac{\lambda}{2}\|\mathbf{w}\|^2
            \]
            \item Equivalent to \textbf{L2-regularized logistic regression}.
        \end{itemize}
    \end{frame}

%%%%%%%%%%%%%%%%%%%%%%%%%%%%%%%%%%%%%%
    \begin{frame}{MAP with Laplace Prior (L1 Regularization)}
        \begin{itemize}
            \item Laplace prior: $\mathbf{w} \sim \text{Laplace}(0, b)$
            \[
                P(\mathbf{w}) \propto \exp\left(-\frac{\|\mathbf{w}\|_1}{b}\right)
            \]
            \item MAP cost:
            \[
                J_{\text{MAP}}(\mathbf{w}) = J_{\text{MLE}}(\mathbf{w}) + \lambda \|\mathbf{w}\|_1
            \]
            where $\lambda = 1/b$
            \item L1 penalty encourages \textbf{sparsity} (feature selection).
        \end{itemize}
    \end{frame}

%%%%%%%%%%%%%%%%%%%%%%%%%%%%%%%%%%%%%%
    \begin{frame}{Regularization Effects (L1 vs L2)}
        \begin{itemize}
            \item \textbf{L2 (Ridge):} smooths weights, keeps all features but shrinks them.
            \item \textbf{L1 (Lasso):} drives some weights to zero, performs feature selection.
            \item Both prevent overfitting by penalizing model complexity.
        \end{itemize}
        \centering
        \includegraphics[width=0.6\textwidth]{pic/regularizers.jpg}
        \vspace{0.3em}
        \\
        {\scriptsize Image adapted from CS 4780, Cornell University}
    \end{frame}
%%%%%%%%%%%%%%%%%%%%%%%%%%%%%%%%%%%%%%
    \begin{frame}{Regularizers: Types and Properties}
        \centering
        \small
        \renewcommand{\arraystretch}{2} % Increase row height
        \begin{tabular}{|p{1cm}|p{3.2cm}|p{4.5cm}|p{4cm}|}
            \hline
            \textbf{Type} & \textbf{Form} & \textbf{Properties / Advantages} & \textbf{Disadvantages / Effect} \\ \hline
            L2 & $r(\mathbf{w}) = \|\mathbf{w}\|_2^2$
            & Strictly convex, differentiable
            & Dense solutions (uses all features) \\ \hline
            L1 & $r(\mathbf{w}) = \|\mathbf{w}\|_1$
            & Convex, encourages sparsity
            & Not differentiable at 0 \\ \hline
            Lp & $r(\mathbf{w}) = \|\mathbf{w}\|_p = (\sum_i |v_i|^p)^{1/p}, \; 0<p\le 1$
            & Very sparse solutions, initialization dependent
            & Non-convex, not differentiable \\ \hline
        \end{tabular}

        \vspace{0.8em}
        \scriptsize
        \textit{Note:} Choice of regularizer affects sparsity and smoothness of learned weights.
    \end{frame}


%%%%%%%%%%%%%%%%%%%%%%%%%%%%%%%%%%%%%%
    \begin{frame}{MLE vs MAP Comparison}
        \centering
        \small
        \renewcommand{\arraystretch}{1.5} % increase row height
        \begin{tabular}{|p{3cm}|p{5cm}|p{5cm}|}
            \hline
            & \textbf{MLE} & \textbf{MAP} \\ \hline
            Objective & Maximize likelihood $P(D|\mathbf{w})$ & Maximize posterior $P(D|\mathbf{w})P(\mathbf{w})$ \\ \hline
            Prior & None (uniform) & Explicit prior $P(\mathbf{w})$ \\ \hline
            Cost function & $-\log P(D|\mathbf{w})$ & $-\log P(D|\mathbf{w}) - \log P(\mathbf{w})$ \\ \hline
            Regularization & None & Arises from prior \\ \hline
            Overfitting control & No & Yes \\ \hline
        \end{tabular}
    \end{frame}



%%%% 12 %%%%%%

    \subsection{Gradient descent}
    \begin{frame}{Gradient descent}
        \begin{itemize}
            \item Remember from previous slides:
            \[
                J(w) = \sum_{i=1}^{n}-y^{(i)}\log (\sigma (\mathbf{w}^T x^{(i)})) -
                (1-y^{(i)})\log (1 - \sigma (\mathbf{w}^T x^{(i)}))
            \]
            \item Update rule for \textbf{gradient descent}:
            \begin{align*}
                w^{t+1} = w^t - \eta \nabla _w J(w^t)
            \end{align*}
            \item With $J(w)$ definition for logistic regression we get:
            \begin{align*}
                \nabla _w J(w) = \sum_{i=1}^{n} (\sigma (\mathbf{w}^T x^{(i)}) - y^{(i)})x^{(i)}
            \end{align*}
            % \item Also keep in mind $f(x^{(i)}; \mathbf{w})= \sigma (\mathbf{w}^Tx^{(i)})$
            % \item Compare with the gradient of \textbf{SSE} in \textbf{linear regression} :
            %     \begin{align*}
            %         \nabla _w J(w) = \sum_{i=1}^{n} (w^Tx^{(i)} - y^{(i)})x^{(i)}
            %     \end{align*}

        \end{itemize}
    \end{frame}
%%%%%% 12.5 %%%%%%%
    \begin{frame}{Gradient descent}
        \begin{itemize}

            \item Compare the gradient of \textcolor{deepgreen}{logistic regression} with the gradient of \textcolor{blue}{SSE} in \textcolor{blue}{linear regression} :
        \end{itemize}
        \begin{align*}
            \color{deepgreen}
            \nabla _w J(w) = \sum_{i=1}^{n} (\sigma (\mathbf{w}^Tx^{(i)}) - y^{(i)})x^{(i)}
        \end{align*}
        \begin{align*}
            \color{blue}
            \nabla _w J(w) = \sum_{i=1}^{n} (\mathbf{w}^Tx^{(i)} - y^{(i)})x^{(i)}
        \end{align*}

    \end{frame}

%%%%% 13 %%%%%%%%

    \begin{frame}{Loss function}
        \begin{itemize}
            %% delete paranthesis
            \item Loss function is a single overall measure of loss incurred for taking our decisions (over entire dataset).
            \item We have:
            \[ Loss(y, \sigma (\mathbf{w}^T x)) = -y \times \log (\sigma ( \mathbf{w}^T x)) - (1-y) \times \log
            (1 - \sigma (\mathbf{w}^T x))
            \]

            \item Since in binary classification either $y=1$ or $y=0$ we have:
            \[
                Loss(y, \sigma (\mathbf{w}^T x)) = \begin{cases}
                                                       - \log (\sigma (\mathbf{w}^T x)) & \textbf{if } y = 1 \\
                                                       - \log (1 - \sigma (\mathbf{w}^T x)) & \textbf{if } y = 0
                \end{cases}
            \]
            \item How is it related to zero-one loss? ($\hat{y}$ is the predicted label and $y$ is the ture label)
            \[
                Loss(y, \hat{y}) =  \begin{cases}
                                        1 & \textbf{if } y \neq \hat{y} \\
                                        0 & \textbf{if } y = \hat{y}
                \end{cases}
            \]
        \end{itemize}
    \end{frame}

%%%%% 14 %%%%%%
% \begin{frame}{Cost Function Summary}
%     \begin{itemize}
%         \item Logistic regression (LR) has a more proper cost function for classification than SSE and Perceptron.

%         \item Why is the cost function of LR also more suitable than
%             \begin{align*}
%                 J(w) = \frac{1}{n}\sum_{i=1}^{n}(y^{(i)} - f(x^{(i)}; \mathbf{w}))^2
%             \end{align*}
%         Where $f(x; \mathbf{w}) = \sigma(\mathbf{w}^Tx)$?
%             \begin{itemize}
%                 \item The conditional distribution $p(y|x, \mathbf{w})$ in the classification problem is not Guassian (it is \textbf{Bernoulli}).
%                 \item The cost function of LR is also convex.
%             \end{itemize}
%     \end{itemize}
% \end{frame}

%%%%%% 15 %%%%%%%

    \subsection{Multi-class logistic regression}
    \begin{frame}{Multi-class logistic regression}
        \begin{itemize}
            \item Now consider a problem where we have $K$ classes and every sample only belongs to one class (for simplicity).
            % \item For each class $k$, $f_k(x; \mathbf{W})$ predicts the probability of $y=k$.
            %     \begin{itemize}
            %         \item i.e., $P(y=k|x, \mathbf{W})$
            %     \end{itemize}
            % \item For each data point $x_0$, $\sum _{k=1}^{K} p(y=k|x_0, \mathbf{W})$ must be $1$
            %     \begin{itemize}
            %         \item $W$ denotes a matrix of $w_i$'s in which each $w_i$ is a weight vector dedicated for class label $i$.
            %     \end{itemize}
            % \item On a new input $x$, to make a prediction, we pick the class that maximizes $f_k(x; \mathbf{W})$:
            %     \begin{align*}
            %         \alpha (x) &= \underset{k=1, \dots , K}{\arg\max} \hspace{0.2cm} f_k(x)
            %     \end{align*}
            %     \begin{center}
            %         \textbf{if $\color{red} f_k(x) > f_j(x)$ $\color{red} \forall j \neq k$ then decide $\color{red} C_k$}
            %     \end{center}
        \end{itemize}
        \centering
        \includegraphics[width=0.8 \linewidth]{pic/multiVsBinaryC.png}
    \end{frame}
%%%%%%%%%%%%%%%%%%%%%%%%
    \begin{frame}{Multi-class logistic regression (cont.)}
        \begin{itemize}
            \item For each class $k$, $\sigma _k(x; \mathbf{W})$ predicts the probability of $y=k$.
            \begin{itemize}
                \item i.e., $P(y=k|x, \mathbf{W})$
            \end{itemize}
            \item For each data point $x_0$, $\sum _{k=1}^{K} P(y=k|x_0, \mathbf{W})$ must be $1$
            \begin{itemize}
                \item $W$ denotes a matrix of $w_i$'s in which each $w_i$ is a weight vector dedicated for class label $i$.
            \end{itemize}
            \item On a new input $x$, to make a prediction, we pick the class that maximizes $\sigma _k(x; \mathbf{W})$:
            \begin{align*}
                \alpha (x) &= \underset{k=1, \dots , K}{\arg\max} \hspace{0.2cm} \sigma _k(x; \mathbf{W})
            \end{align*}
            \begin{center}
                \textbf{if $\color{red} \sigma _k(x; \mathbf{W}) > \sigma _j(x; \mathbf{W})$ $\color{red} \forall j \neq k$ then decide $\color{red} C_k$}
            \end{center}
        \end{itemize}
    \end{frame}


%%%%% 16 %%%%%%%
    \begin{frame}{Multi-class logistic regression (cont.)}
        \begin{itemize}
            \item $K > 2$ and $y \in \{1,2,\dots,K\}$
            \begin{align*}
                \sigma _k(x, \mathbf{W}) = P(y=k|x) = \frac{\exp{(w^T_kx)}}{\sum_{j=1}^{K}\exp{(w_j^Tx)}}
            \end{align*}
            \item Normalized exponential (Aka \textbf{Softmax})

            \item if $w_k^Tx \gg w_j^Tx$ for all $j \neq k$ then $P(C_k|x) \approx 1$ and $P(C_j|x) \approx 0$
            \item Note : remember from Bayes theorem:
            \[
                P(C_k|x) = \frac{P(x|C_k)P(C_k)}
                {\sum_{j=1}^{K}P(x|C_j)P(C_j)}
            \]

        \end{itemize}
    \end{frame}
%%%%% 16.5 %%%%%
    \begin{frame}{Multi-class logistic regression (cont.)}
        \begin{itemize}
            \item Softmax function \textbf{smoothly} highlights the maximum probability and is differentiable.
            \item Compare it with $\max (.)$ function which is strict and non-differentiable
            \item Softmax can also handle negative values because we are using exponential function
            \item And it gives us probability for each class since:
            \begin{align*}
                \displaystyle \sum _{k=1}^{K} \frac{\exp (w_k^Tx)}{\sum _{j=1}^{K} \exp (w_j^Tx) } = 1
            \end{align*}
        \end{itemize}
    \end{frame}

%%%% 16.6 %%%%%
    \begin{frame}{Multi-class logistic regression (cont.)}
        \begin{itemize}
            \item An example of applying softmax (note that $z_i=w^Tx_i$):
        \end{itemize}
        \begin{center}
            \includegraphics[width=0.7\textwidth]{pic/softmax0.png}

        \end{center}
    \end{frame}

%%%% 17 %%%%%
    \begin{frame}{Multi-class logistic regression (cont.)}
        \begin{itemize}
            \item Again we set $J(W)$ as negative of log likelihood.
            \item We need $\hat{W} = \underset{W}{\arg\min} \hspace{0.2cm} J(W)$
            \begin{align*}
                J(W) &= -\log \prod_{i=1}^{n} \textcolor{red}{P(y^{(i)}|x^{(i)}, \mathbf{W})} \\
                &= -\log \prod_{i=1}^{n}\textcolor{red}{\prod_{k=1}^{K}\sigma _k(x^{(i)}; \mathbf{W})^{y_k^{(i)}} }\\
                &= -\sum_{i=1}^{n}\sum_{k=1}^{K}y_k^{(i)} \log (\sigma _k(x^{(i)}; \mathbf{W}))
            \end{align*}
            \item If \textbf{$\textbf{i}$-th} sample belongs to class $k$ then $y^{(i)}_k$ is 1 else 0.
            \item Again no closed-from solution for $\hat{W}$
        \end{itemize}

    \end{frame}
%%%%%%% 17.5 %%%%%%%
    \begin{frame}{Multi-class logistic regression (cont.)}
        \begin{itemize}
            \item From previous slides we have:
            \begin{align*}
                J(W) = -\sum_{i=1}^{n}\sum_{k=1}^{K}y_k^{(i)} \log (\sigma _k(x^{(i)}; \mathbf{W}))
            \end{align*}
            \item In which:
            \[
                W = [w_1,w_2, \dots, w_K], \quad Y =
                \text{\small $
                \begin{pmatrix}
                    y^{(1)} \\
                    y^{(2)} \\
                    \vdots \\
                    y^{(n)}
                \end{pmatrix}
                $}
                =
                \begin{pmatrix}
                    y_1^{(1)} & \dots & y_K^{(1)} \\
                    y_1^{(2)} & \dots & y_K^{(2)} \\
                    \vdots    & \ddots & \vdots \\
                    y_1^{(n)} & \dots & y_K^{(n)}
                \end{pmatrix}
            \]

            \item $y$ is a vector of length $K$ (1-of-$K$ encoding)
            \begin{itemize}
                % \item Each vector $y^{(i)}$ only consists of zeros and ones.
                \item For example $y=[0,0,1,0]^T$ when the target class is $C_3$.
            \end{itemize}
        \end{itemize}
    \end{frame}
%%%%%% 18 %%%%%%%%
    \begin{frame}{Multi-class logistic regression (cont.)}
        \begin{itemize}
            \item Update rule for gradient descent:

        \end{itemize}
        \begin{align*}
            w_j^{t+1} &= w_j^t - \eta \nabla _W J(W^t) \\
            \nabla _{w_{j}} J(W) &= \sum_{i=1}^{n} (\sigma _j(x^{(i)}; \mathbf{W}) - y_j^{(i)})x^{(i)}
        \end{align*}

        \begin{itemize}
            \item $w_j^t$ denotes the weight vector for class $j$ (since in multi-class LR, each class has its own weight vector) in the $t$-th iteration
        \end{itemize}
    \end{frame}


    \section{Evaluation Metrics}

    \begin{frame}{Evaluation Context}
        \begin{itemize}
            \item Model evaluation quantifies predictive performance on unseen data.
            \item Metrics are derived from the confusion matrix:
        \end{itemize}
        \vspace{0.3cm}
        \begin{center}
            \begin{tabular}{c|cc}
                \toprule
                & Predicted Positive & Predicted Negative \\
                \midrule
                Actual Positive & TP & FN \\
                Actual Negative & FP & TN \\
                \bottomrule
            \end{tabular}
        \end{center}
        \vspace{0.3cm}
        \begin{itemize}
            \item TP, TN, FP, FN provide the foundation for classification performance indices.
        \end{itemize}
    \end{frame}

    \begin{frame}{Accuracy and Error Rate}
        \textbf{Accuracy:} proportion of correctly classified instances
        \[
            \text{Accuracy} = \frac{TP + TN}{TP + TN + FP + FN}
        \]

        \textbf{Error Rate:} complement of accuracy
        \[
            \text{Error Rate} = \frac{FP + FN}{TP + TN + FP + FN} = 1 - \text{Accuracy}
        \]

        \vspace{0.3cm}
        \begin{itemize}
            \item Accuracy is reliable for balanced class distributions.
            \item Limited interpretability for skewed datasets.
        \end{itemize}
    \end{frame}

    \begin{frame}{Precision and Recall}
        \textbf{Precision (Positive Predictive Value)}
        \[
            P = \frac{TP}{TP + FP}
        \]
        \begin{itemize}
            \item Fraction of predicted positives that are truly positive.
            \item Sensitive to false positives.
        \end{itemize}

        \vspace{0.3cm}
        \textbf{Recall (Sensitivity / True Positive Rate)}
        \[
            R = \frac{TP}{TP + FN}
        \]
        \begin{itemize}
            \item Fraction of actual positives correctly identified.
            \item Sensitive to false negatives.
        \end{itemize}
    \end{frame}

    \begin{frame}{F1-Score and Trade-Off Analysis}
        \textbf{F1-Score:} harmonic mean of precision and recall
        \[
            F1 = 2 \cdot \frac{P \cdot R}{P + R}
        \]

        \vspace{0.3cm}
        \begin{itemize}
            \item Provides a single measure balancing precision and recall.
            \item Useful in imbalanced datasets or when both types of error are significant.
            \item Trade-off: Increasing recall typically decreases precision and vice versa.
        \end{itemize}
    \end{frame}

    \begin{frame}{Comparative Overview}
        \begin{center}
            \begin{tabular}{lccc}
                \toprule
                Metric & Formula & Focus & Typical Use-Case \\
                \midrule
                Accuracy & $\frac{TP+TN}{TP+FP+FN+TN}$ & Overall correctness & Balanced classes \\
                Precision & $\frac{TP}{TP+FP}$ & False positive control & Spam detection, IR \\
                Recall & $\frac{TP}{TP+FN}$ & False negative control & Medical diagnosis, anomaly detection \\
                F1-Score & $2\frac{PR}{P+R}$ & Balance & Imbalanced or skewed datasets \\
                \bottomrule
            \end{tabular}
        \end{center}
        \vspace{0.3cm}
        \begin{itemize}
            \item Metric choice should reflect operational objectives and misclassification costs.
            \item Always interpret metrics in conjunction with the confusion matrix.
        \end{itemize}
    \end{frame}


% \section{Methods}

% \subsection{Diffusion Model}

% \begin{frame}{Title}
%     \begin{itemize}
%         \item \lipsum[3][1-4]
%     \end{itemize}
%     \begin{table}[h]
%         \centering
%         \begin{tabular}{c|c}
%             Microsoft\textsuperscript{\textregistered}  Windows & Apple\textsuperscript{\textregistered}  Mac OS \\
%             \hline
%             Windows-Kernel & Unix-like \\
%             Arm, Intel & Intel, Apple Silicon \\
%             Sudden update & Stable update \\
%             Less security & More security \\
%             ... & ... \\
%         \end{tabular}
%     \end{table}
% \end{frame}

% \begin{frame}{Algorithms}
%     \begin{exampleblock}{Non-Numbering Formula}
%         \begin{equation*}
%             J(\theta) = \mathbb{E}_{\pi_\theta}[G_t] = \sum_{s\in\mathcal{S}} d^\pi (s)V^\pi(s)=\sum_{s\in\mathcal{S}} d^\pi(s)\sum_{a\in\mathcal{A}}\pi_\theta(a|s)Q^\pi(s,a)
%         \end{equation*}
%     \end{exampleblock}
%     \begin{exampleblock}{Multi-Row Formula\footnote{If text appears in the formula,use $\backslash$mathrm\{\} or $\backslash$text\{\} instead}}
%         \begin{align}
%             Q_\mathrm{target}&=r+\gamma Q^\pi(s^\prime, \pi_\theta(s^\prime)+\epsilon)\\
%             \epsilon&\sim\mathrm{clip}(\mathcal{N}(0, \sigma), -c, c)\nonumber
%         \end{align}
%     \end{exampleblock}
% \end{frame}

% \begin{frame}
%     \begin{exampleblock}{Numbered Multi-line Formula}
%         % Taken from Mathmode.tex
%         \begin{multline}
%             A=\lim_{n\rightarrow\infty}\Delta x\left(a^{2}+\left(a^{2}+2a\Delta x+\left(\Delta x\right)^{2}\right)\right.\label{eq:reset}\\
%             +\left(a^{2}+2\cdot2a\Delta x+2^{2}\left(\Delta x\right)^{2}\right)\\
%             +\left(a^{2}+2\cdot3a\Delta x+3^{2}\left(\Delta x\right)^{2}\right)\\
%             +\ldots\\
%             \left.+\left(a^{2}+2\cdot(n-1)a\Delta x+(n-1)^{2}\left(\Delta x\right)^{2}\right)\right)\\
%             =\frac{1}{3}\left(b^{3}-a^{3}\right)
%         \end{multline}
%     \end{exampleblock}
% \end{frame}

% \begin{frame}{Graphics and Columns}
%     \begin{minipage}[c]{0.3\linewidth}
%         \psset{unit=0.8cm}
%         \begin{pspicture}(-1.75,-3)(3.25,4)
%             \psline[linewidth=0.25pt](0,0)(0,4)
%             \rput[tl]{0}(0.2,2){$\vec e_z$}
%             \rput[tr]{0}(-0.9,1.4){$\vec e$}
%             \rput[tl]{0}(2.8,-1.1){$\vec C_{ptm{ext}}$}
%             \rput[br]{0}(-0.3,2.1){$\theta$}
%             \rput{25}(0,0){%
%             \psframe[fillstyle=solid,fillcolor=lightgray,linewidth=.8pt](-0.1,-3.2)(0.1,0)}
%             \rput{25}(0,0){%
%             \psellipse[fillstyle=solid,fillcolor=yellow,linewidth=3pt](0,0)(1.5,0.5)}
%             \rput{25}(0,0){%
%             \psframe[fillstyle=solid,fillcolor=lightgray,linewidth=.8pt](-0.1,0)(0.1,3.2)}
%             \rput{25}(0,0){\psline[linecolor=red,linewidth=1.5pt]{->}(0,0)(0.,2)}
% %           \psRotation{0}(0,3.5){$\dot\phi$}
% %           \psRotation{25}(-1.2,2.6){$\dot\psi$}
%             \psline[linecolor=red,linewidth=1.25pt]{->}(0,0)(0,2)
%             \psline[linecolor=red,linewidth=1.25pt]{->}(0,0)(3,-1)
%             \psline[linecolor=red,linewidth=1.25pt]{->}(0,0)(2.85,-0.95)
%             \psarc{->}{2.1}{90}{112.5}
%             \rput[bl](.1,.01){C}
%         \end{pspicture}
%     \end{minipage}\hspace{2cm}
%     \begin{minipage}{0.5\linewidth}
%         \medskip
%         % \hspace{2cm}
%         \begin{figure}[h]
%             \centering
%             \includegraphics[height=.4\textheight]{pic/sample.pdf}
%         \end{figure}
%     \end{minipage}
% \end{frame}

% \begin{frame}[fragile]{\LaTeX{} Common Commands}
%     \begin{exampleblock}{Commands}
%         \centering
%         \footnotesize
%         \begin{tabular}{llll}
%             \cmd{chapter} & \cmd{section} & \cmd{subsection} & \cmd{paragraph} \\
%             chapter & section & sub-section & paragraph \\\hline
%             \cmd{centering} & \cmd{emph} & \cmd{verb} & \cmd{url} \\
%             center & emphasize & original & hyperlink \\\hline
%             \cmd{footnote} & \cmd{item} & \cmd{caption} & \cmd{includegraphics} \\
%             footnote & list item & caption & insert image \\\hline
%             \cmd{label} & \cmd{cite} & \cmd{ref} \\
%             label & citation & refer\\\hline
%         \end{tabular}
%     \end{exampleblock}
%     \begin{exampleblock}{Environment}
%         \centering
%         \footnotesize
%         \begin{tabular}{lll}
%             \env{table} & \env{figure} & \env{equation}\\
%             table & figure & formula \\\hline
%             \env{itemize} & \env{enumerate} & \env{description}\\
%             non-numbering item & numbering item & description \\\hline
%         \end{tabular}
%     \end{exampleblock}
% \end{frame}

% \begin{frame}[fragile]{\LaTeX{} Examples of environmental commands}
%     \begin{minipage}{0.5\linewidth}
% \begin{lstlisting}[language=TeX]
% \begin{itemize}
%   \item A \item B
%   \item C
%   \begin{itemize}
%     \item C-1
%   \end{itemize}
% \end{itemize}
% \end{lstlisting}
%     \end{minipage}\hspace{1cm}
%     \begin{minipage}{0.3\linewidth}
%         \begin{itemize}
%             \item A
%             \item B
%             \item C
%             \begin{itemize}
%                 \item C-1
%             \end{itemize}
%         \end{itemize}
%     \end{minipage}
%     \medskip
%     \pause
%     \begin{minipage}{0.5\linewidth}
% \begin{lstlisting}[language=TeX]
% \begin{enumerate}
%   \item A \item B
%   \item C
%   \begin{itemize}
%     \item[n+e]
%   \end{itemize}
% \end{enumerate}
% \end{lstlisting}
%     \end{minipage}\hspace{1cm}
%     \begin{minipage}{0.3\linewidth}
%         \begin{enumerate}
%             \item A
%             \item B
%             \item C
%             \begin{itemize}
%                 \item[n+e]
%             \end{itemize}
%         \end{enumerate}
%     \end{minipage}
% \end{frame}

% \begin{frame}[fragile]{\LaTeX{} Formulas}
%     \begin{columns}
%         \begin{column}{.55\textwidth}
% \begin{lstlisting}[language=TeX]
% $V = \frac{4}{3}\pi r^3$

% \[
%   V = \frac{4}{3}\pi r^3
% \]

% \begin{equation}
%   \label{eq:vsphere}
%   V = \frac{4}{3}\pi r^3
% \end{equation}
% \end{lstlisting}
%         \end{column}
%         \begin{column}{.4\textwidth}
%             $V = \frac{4}{3}\pi r^3$
%             \[
%                 V = \frac{4}{3}\pi r^3
%             \]
%             \begin{equation}
%                 \label{eq:vsphere}
%                 V = \frac{4}{3}\pi r^3
%             \end{equation}
%         \end{column}
%     \end{columns}
%     \begin{itemize}
%         \item more information \href{https://ja.overleaf.com/learn/latex/Mathematical_expressions}{\color{purple}{here}}
%     \end{itemize}
%%%%%%%%%%%%%%%%%%%%%%%%%%%%%%%%%%%%%%%%%%%%%%%%
    \section{Extra reading}
    \subsection{Probabilistic view in classification}
%%%% 0 %%%%%

% \begin{frame}{Probabilistic view in classification problem}
%     \begin{itemize}
%         \item First consider a simple classification problem:
%             \begin{itemize}
%                 \item We want to find individuals with diabetes and detect them.
%                 \item We have two \textbf{observations}: blood cell count and Plasma glucose value.
%                 \item Observations are our \textbf{features} and having Diabetes or not is our \textbf{class label} in this classification problem.
%             \end{itemize}
%     \end{itemize}
%         %%% next frame ->
%     % \begin{center}
%     %     \includegraphics[width=0.45\textwidth]{pic/classification_plot.png}
%     %     \captionof{figure}{\scriptize [Sanja Fidler's Slides, University of Toronto, CSC411]}
%     % \end{center}

% \end{frame}
%%%%% 0.25 %%%%%%
% \begin{frame}{Probabilistic view in classification problem Cont.}
%     \begin{itemize}
%         \item Dataset visualization (x-axis denotes white blood count and y-axis represents plasma glucose value:
%     \end{itemize}
%     \begin{center}
%         \includegraphics[width=0.45\textwidth]{pic/classification_plot.png}
%         \captionof{figure}{\footnotesize [Sanja Fidler's Slides, University of Toronto, CSC411]}
%     \end{center}
% \end{frame}

%%%%% 0.3 %%%%%
    \begin{frame}{Probabilistic view in classification problem}
        \begin{itemize}
            \item In a classification problem:
            \begin{itemize}
                \item Each \textbf{feature} is a \textbf{random variable} (e.g. a person's height)
                \item The \textbf{class label} is also considered a \textbf{random variable} (e.g. a person could be overweight or not)
            \end{itemize}
            \item We observe the feature values for a random sample and intend to find its class label
            \begin{itemize}
                \item Evidence: Feature vector $x$
                \item Objective: Class label
            \end{itemize}
        \end{itemize}
    \end{frame}
%%%%%0.5%%%%%%
    \begin{frame}{Definitions}
        \begin{itemize}
            \item Posterior probability : The probability of a class label $C_k$ given a sample $x$
            \[
                P(C_k|x)
            \]

            \item Likelihood or class conditional probability : PDF of feature vector $x$ for samples of class $C_k$
            \[
                P(x|C_k)
            \]

            \item Prior probability : Probability of the label be $C_k$
            \[
                P(C_k)
            \]

            \item $P(x)$: PDF of feature vector $x$
            \begin{itemize}
                \item From total probability theorem:

                \[ P(x)=\sum_{k=1}^{K}P(x|C_k)P(C_k)
                \]
            \end{itemize}

        \end{itemize}
    \end{frame}



%%% 1 %%%%
    \subsection{Probabilistic classifiers}

    \begin{frame}{Probabilistic classifiers}
        \begin{itemize}
            \item Probabilistic approaches can be divided in two main categories:
            \begin{itemize}
                \item Generative
                \begin{itemize}
                    \item Estimate PDF $P(x, C_k)$ for each class $C_k$ and then use it to find $P(C_k|x)$. Alternatively estimate both PDF $P(x|C_k)$ and $P(C_k)$ to find $P(C_k|x)$.
                \end{itemize}
                \item Discriminative
                \begin{itemize}
                    \item Directly estimate $P(C_k|x)$ for class $C_k$
                \end{itemize}
            \end{itemize}
        \end{itemize}
    \end{frame}
%%%%%%%% 1.5 %%%%%%%%%%%%
    \begin{frame}{Probabilistic classifiers (cont.)}
        \begin{itemize}
            \item Let's assume we have input data $x$ and want to classify the data into labels $y$.
            \item A generative model learns the \textbf{joint} probability distribution $P(x,y)$.

            \item A discriminative model learns the \textbf{conditional} probability distribution $P(y|x)$

        \end{itemize}
    \end{frame}

%%%% 1.6 %%%%%%
    \begin{frame}{Discriminative vs. Generative : example}
        \begin{itemize}
            \item Suppose we have the following dataset in form of $(x, y)$:
            \[
                (1,0), (1,0), (2,0), (2,1)
            \]
            \item $P(x,y)$ is :
            \[
                \begin{array}{c|cc}
                    & y=0 & y=1 \\
                    \hline
                    x=1 & \frac{1}{2} & 0 \\
                    x=2 & \frac{1}{4} & \frac{1}{4} \\
                \end{array}
            \]
            \item $P(y|x)$ is :
            \[
                \begin{array}{c|cc}
                    & y=0 & y=1 \\
                    \hline
                    x=1 & 1 & 0 \\
                    x=2 & \frac{1}{2} & \frac{1}{2} \\
                \end{array}
            \]
        \end{itemize}
    \end{frame}

%%%%% 1.8 %%%%%

    \begin{frame}{Discriminative vs. Generative : example (cont.)}
        \begin{itemize}
            \item The distribution $P(y|x)$ is the natural distribution for classifying a given sample $x$ into class $y$.
            \begin{itemize}
                \item This is why that algorithms which model this directly are called \textbf{discriminative} algorithms.
            \end{itemize}
            \item Generative algorithms model $P(x,y)$, which can be transformed into $P(y|x)$ by Bayes rule and then used for classification.
            \begin{itemize}
                \item However, the distribution $P(x,y)$ can also be used for other purposes.
                \item For example we can use $P(x,y)$ to \textbf{generate} likely $(x,y)$ pairs
            \end{itemize}
        \end{itemize}
    \end{frame}

%%%% 2 %%%%%
    \begin{frame}{Generative approach}
        \begin{enumerate}
            \item Inference
            \begin{itemize}
                \item Determine class conditional densities $P(x|C_k)$ and priors $P(C_k)$
                \item Use Bayes theorem to find $P(C_k|x)$
            \end{itemize}
            \item Decision
            \begin{itemize}
                \item Make optimal assignment for new input (after learning the model in the inference stage)
                \item if $P(C_i|x) > P(C_j|x) \forall j \neq i$, then decide $C_i$ .
            \end{itemize}
        \end{enumerate}
    \end{frame}

%%%% 2.5 %%%%%
    \begin{frame}{Generative approach (cont.)}

        \begin{itemize}
            \item Generative approach for a binary classification problem:
        \end{itemize}
        \begin{figure}[h]
            \centering
            \includegraphics[width=0.4\textwidth]{pic/Generative.png}
            %   \caption*{\footnotesize [Bishop]}
        \end{figure}
        \vfill
        \begin{tikzpicture}[remember picture,overlay]
            \node[anchor=south west, xshift=0.1cm, yshift=0.22cm] at (current page.south west) {
                \scriptsize Figures adapted from Machine Learning and Pattern Recognition, Bishop
            };
        \end{tikzpicture}
    \end{frame}
%%%% 3 %%%%
    \begin{frame}{Discriminative approach}
        \begin{enumerate}
            \item Inference
            \begin{itemize}
                \item Determine the posterior class probabilities $P(C_k|x)$ directly.
            \end{itemize}
            \item Decision
            \begin{itemize}
                \item Make optimal assignment for new input (after learning the model in the inference stage)
                \item if $P(C_i|x) > P(C_j|x) \forall j \neq i$, then decide $C_i$ .
            \end{itemize}
        \end{enumerate}
    \end{frame}
%%%% 3.5 %%%%%%
    \begin{frame}{Discriminative approach (cont.)}
        \begin{itemize}
            \item Discriminative approach for a binary classification problem:
        \end{itemize}
        \begin{figure}[h]
            \centering
            \includegraphics[width=0.4\textwidth]{pic/Disc.png}
            %   \caption*{\footnotesize [Bishop]}
        \end{figure}
        \vfill
        \begin{tikzpicture}[remember picture,overlay]
            \node[anchor=south west, xshift=0.1cm, yshift=0.22cm] at (current page.south west) {
                \scriptsize Figures adapted from Machine Learning and Pattern Recognition, Bishop
            };
        \end{tikzpicture}
    \end{frame}
%%% 4 %%%%
    %% partition photos into 2 slides
% \begin{frame}{Discriminative vs. Generative approach}
%   \begin{figure}[h]
%   \centering
%   \includegraphics[width=\textwidth]{pic/GvD.png}
%   \caption*{[Bishop]}
%   % \label{fig:image}
% \end{figure}
% \end{frame}




    \section{References}

    \begin{frame}{Contributions}
        \begin{itemize}
            \item \textbf{These slides are authored by:}
            \begin{itemize}
                \setlength{\itemsep}{10pt} % Adjust the value to control the spacing
                \item \href{https://github.com/Danial-Gharib}{Danial Gharib}
            \end{itemize}
        \end{itemize}

    \end{frame}

    \begin{frame}[allowframebreaks]
        \bibliography{ref}
        \bibliographystyle{ieeetr}
        \nocite{*}
    \end{frame}

\end{document}
