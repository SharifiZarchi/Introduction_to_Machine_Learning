%%%%%%%%%%%%%%%%%%%%%%%%%%%%%%%%%%%%%%%%%%%%%%%%%%%%%%
% A Beamer template for University of Wollongong     %
% Based on THU beamer theme                          %
% Author: Qiuyu Lu                                   %
% Date: July 2024                                    %
% LPPL Licensed.                                     %
%%%%%%%%%%%%%%%%%%%%%%%%%%%%%%%%%%%%%%%%%%%%%%%%%%%%%%
% Customized for Sharif University of Technology     %
%%%%%%%%%%%%%%%%%%%%%%%%%%%%%%%%%%%%%%%%%%%%%%%%%%%%%%


\documentclass[serif, aspectratio=169]{beamer}
%\documentclass[serif]{beamer}  % for 4:3 ratio
\usepackage[T1]{fontenc}
\usepackage{fourier} % see "http://faq.ktug.org/wiki/uploads/MathFonts.pdf" for other options
\usepackage{hyperref}
\usepackage{latexsym,amsmath,xcolor,multicol,booktabs,calligra}
\usepackage{graphicx,pstricks,listings,stackengine}
\usepackage{lipsum}
\usepackage{ragged2e}
\usepackage{colortbl}
\usepackage{times}
\usepackage{fancyhdr,graphicx,amsmath,amssymb,algorithm,algpseudocode,mathtools,needspace}
\usepackage{tikz}
\usetikzlibrary{positioning}
% For writing comments that are aligned to the left side
\makeatletter
\NewDocumentCommand{\LeftComment}{s m}{%
    \Statex \IfBooleanF{#1}{\hspace*{\ALG@thistlm}}\(\triangleright\) #2}
\makeatother
% To manually indent states in algorithmicx
\newcommand{\IndState}{\State\hspace{\algorithmicindent}}
% To make breakable algorithms
\makeatletter
\newenvironment{nofloatalgorithmic}[2][0]
{
    \par
    \needspace{\dimexpr\baselineskip+6.8pt}
    \noindent
    \hrule height.8pt depth0pt \kern2pt
    \refstepcounter{algorithm}
    \addcontentsline{loa}{algorithm}{\numberline{\thealgorithm}#2}
    \noindent\textbf{\fname@algorithm~\thealgorithm} #2\par
    \kern2pt\hrule\kern2pt
    \begin{algorithmic}[#1]
}
{
    \end{algorithmic}
    \nobreak\kern2pt\hrule\relax
}
\makeatother
% To make vertical arrow
\newcommand\vertarrowbox[3][6ex]{%
    \begin{array}[t]{@{}c@{}} #2 \\
    \left\uparrow\vcenter{\hrule height #1}\right.\kern-\nulldelimiterspace\\
    \makebox[0pt]{\scriptsize#3}
    \end{array}%
}
\DeclareMathOperator*{\argmax}{argmax}

\author{Ali Sharifi-Zarchi}
\title{Machine Learning (CE 40717)}
\subtitle{Fall 2025}
\institute{
    CE Department \\
    Sharif University of Technology
}
%\date{\small \today}
% \usepackage{UoWstyle}
\usepackage{SUTstyle}

% defs
\def\cmd#1{\texttt{\color{red}\footnotesize $\backslash$#1}}
\def\env#1{\texttt{\color{blue}\footnotesize #1}}
\definecolor{deepblue}{rgb}{0,0,0.5}
\definecolor{deepred}{RGB}{153,0,0}
\definecolor{deepgreen}{rgb}{0,0.5,0}
\definecolor{halfgray}{gray}{0.55}

\lstset{
    basicstyle=\ttfamily\small,
    keywordstyle=\bfseries\color{deepblue},
    emphstyle=\ttfamily\color{deepred},    % Custom highlighting style
    stringstyle=\color{deepgreen},
    numbers=left,
    numberstyle=\small\color{halfgray},
    rulesepcolor=\color{red!20!green!20!blue!20},
    frame=shadowbox,
}


\begin{document}

    \begin{frame}
        \titlepage
        \vspace*{-0.6cm}
        \begin{figure}[htpb]
            \begin{center}
                \includegraphics[keepaspectratio, scale=0.25]{pic/sharif-main-logo.png}
            \end{center}
        \end{figure}
    \end{frame}

    \begin{frame}
        \tableofcontents[sectionstyle=show,
            subsectionstyle=show/shaded/hide,
            subsubsectionstyle=show/shaded/hide]
    \end{frame}

    \section{Introduction}

    \begin{frame}{Definition}
        \begin{minipage}{0.55\textwidth}
            \begin{itemize}\itemsep1em
            \item \textbf{Given:} A training dataset
            \[
                D = \{(\mathbf{x}^{(i)}, y^{(i)})\}_{i=1}^{N}, \quad
                y^{(i)} \in \{1, 2, \dots, K\}
            \]
            \item \textbf{Goal:} Learn a function that maps any new input $\mathbf{x}$ to one of $K$ classes.
            \item \textbf{Examples:}
            \begin{itemize}
                \item Email → Spam or Not Spam
                \item Image → Cat, Dog, or Bird
                \item Transaction → Fraudulent or Legitimate
            \end{itemize}
            \end{itemize}
        \end{minipage}
        \hfill
        \begin{minipage}{0.42\textwidth}
            \begin{figure}
                \centering
                \includegraphics[width=\linewidth]{pic/Figure_30.png}
            \end{figure}
        \end{minipage}
    \end{frame}

    \begin{frame}{Real-World Example of Classification}
        \begin{itemize}
            \justifying
            \item \textbf{Pima Indians Diabetes Dataset:}
            \medskip
            \begin{itemize}\itemsep.8em
            \item \textbf{Problem}:
            Predict whether a patient has diabetes based on medical diagnostics.
            \item \textbf{Context}:
            Early detection of diabetes is critical for treatment and management.
            \end{itemize}
        \end{itemize}

        \begin{table}[]
            \resizebox{\textwidth}{!}{%
                \begin{tabular}{|c|c|c|c|c|c|c|c|c|c|}
                    \hline
                    \rowcolor[HTML]{C0C0C0}
                    \cellcolor[HTML]{000000}                                                  & Number of times pregnant                          & Glucose                                           & Blood Pressure                                    & Skin Thickness                                    & Insulin                                           & Diabetes pedigree function                        & Age                                               & BMI                                               & Label                                             \\ \hline
                    \cellcolor[HTML]{C0C0C0}Patient 1                                         & 6                                                 & 148                                               & 72                                                & 35                                                & 0                                                 & 0.627                                             & 50                                                & 33.6                                              & \cellcolor[HTML]{FE0000}Positive                  \\ \hline
                    \cellcolor[HTML]{C0C0C0}Patient 2                                         & 1                                                 & 85                                                & 66                                                & 29                                                & 0                                                 & 0.351                                             & 31                                                & 26.6                                              & \cellcolor[HTML]{32CB00}Negative                  \\ \hline
                    \cellcolor[HTML]{C0C0C0}Patient 3                                         & 0                                                 & 137                                               & 40                                                & 35                                                & 168                                               & 2.288                                             & 33                                                & 43.1                                              & \cellcolor[HTML]{FE0000}Positive                  \\ \hline
                    \cellcolor[HTML]{C0C0C0}Patient 4                                         & 1                                                 & 89                                                & 66                                                & 23                                                & 94                                                & 0.167                                             & 21                                                & 28.1                                              & \cellcolor[HTML]{32CB00}Negative                  \\ \hline
                    \cellcolor[HTML]{C0C0C0}\begin{tabular}[c]{@{}c@{}}.\\ .\\ .\end{tabular} & \begin{tabular}[c]{@{}c@{}}.\\ .\\ .\end{tabular} & \begin{tabular}[c]{@{}c@{}}.\\ .\\ .\end{tabular} & \begin{tabular}[c]{@{}c@{}}.\\ .\\ .\end{tabular} & \begin{tabular}[c]{@{}c@{}}.\\ .\\ .\end{tabular} & \begin{tabular}[c]{@{}c@{}}.\\ .\\ .\end{tabular} & \begin{tabular}[c]{@{}c@{}}.\\ .\\ .\end{tabular} & \begin{tabular}[c]{@{}c@{}}.\\ .\\ .\end{tabular} & \begin{tabular}[c]{@{}c@{}}.\\ .\\ .\end{tabular} & \begin{tabular}[c]{@{}c@{}}.\\ .\\ .\end{tabular} \\ \hline
                \end{tabular}
            %
            }
        \end{table}



    \end{frame}


    \begin{frame}{Classification vs. Regression: Comparison Table}
        \begin{center}
            \renewcommand{\arraystretch}{1.5}
            \setlength{\tabcolsep}{4pt}
            \begin{tabular}{|
                    >{\columncolor[HTML]{C0C0C0}}p{2.5cm} |p{4cm}|p{5.5cm}|}
                \hline
                \textbf{Aspect} & \textbf{Regression} & \textbf{Classification} \\ \hline
                \textbf{Output Type} & Continuous value ($\mathbb{R}$) & Discrete class label \\ \hline
                \textbf{Examples} & House price, temperature & Spam detection, sentiment analysis \\ \hline
                \textbf{Evaluation Metrics} & MSE, MAE & Accuracy, Precision, Recall \\ \hline
            \end{tabular}
        \end{center}
    \end{frame}

    \begin{frame}{Classification vs. Regression: Figure}
        \begin{figure}
            \centering
            \includegraphics[width=0.6\linewidth]{pic/Figure_37.png}
        \end{figure}
    \end{frame}




    \section{Discriminant Functions}



    \begin{frame}{Discriminant Functions in Machine Learning}
        \begin{itemize}\itemsep1.2em
        \item \textbf{Conceptual Overview:}\\
        A discriminant function constitutes a mapping from the feature space to a real-valued score that quantifies the likelihood or confidence of a sample belonging to a specific class.
        \item \textbf{Formal Definition:}\\
        Let $\mathbf{x} \in \mathbb{R}^d$ denote a feature vector. A discriminant function is a function $g(\mathbf{x}) : \mathbb{R}^d \rightarrow \mathbb{R}$ such that larger values of $g(\mathbf{x})$ correspond to stronger evidence for a particular class.
        \item \textbf{Objective:}\\
        Design $g(\mathbf{x})$ to maximize correct classification over a given dataset.
        \end{itemize}
    \end{frame}




    \begin{frame}{Classification Using Discriminant Functions}
        \begin{itemize}\itemsep1.5em
        \item \textbf{Binary Classification:}\\
        \begin{itemize}
            \item Consider two discriminant functions $g_1(\mathbf{x})$ and $g_2(\mathbf{x})$ corresponding to classes $C_1$ and $C_2$, respectively.
            \item The predicted class $\hat{y}$ is determined by the criterion:
            \[
                \hat{y} =
                \begin{cases}
                    C_1 & \text{if } g_1(\mathbf{x}) > g_2(\mathbf{x}) \\
                    C_2 & \text{otherwise.}
                \end{cases}
            \]
        \end{itemize}

        \item \textbf{Multi-Class Classification:}\\
        \begin{itemize}
            \item For $k$ classes, compute $g_i(\mathbf{x})$ for each class $C_i$, $i=1,\dots,k$.
            \item Assign $\mathbf{x}$ to the class corresponding to the maximal discriminant value:
            \[
                \hat{y} = \arg \max_{i} g_i(\mathbf{x})
            \]
        \end{itemize}

        \item \textbf{Interpretation:}
        The discriminant function serves as a quantitative measure of class membership confidence.
        \end{itemize}
    \end{frame}


    \begin{frame}{Decision Boundary}
        \begin{itemize}
            \item \justifying \textbf{Definition}: A dividing hyperplane that separates different classes in a feature space, also known as "Decision Surface".
            \medskip
            \begin{figure}
                \centering
                \includegraphics[width=.85\linewidth]{pic/Figure_24.png}
            \end{figure}
        \end{itemize}
    \end{frame}

    \begin{frame}{Two-Class Discriminant Function}
        \begin{itemize}
            \item In the case of a binary classification problem, a single discriminant function suffices:
            \[
                g : \mathbb{R}^d \rightarrow \mathbb{R}.
            \]
            \item The two class-specific discriminants can be represented as:
            \[
                g_1(\mathbf{x}) = g(\mathbf{x}), \quad g_2(\mathbf{x}) = -g(\mathbf{x}).
            \]
            \item The associated decision rule is:
            \[
                \hat{y} =
                \begin{cases}
                    C_1 & \text{if } g(\mathbf{x}) > 0 \\
                    C_2 & \text{if } g(\mathbf{x}) < 0
                \end{cases}
            \]
            \item The decision boundary corresponds to the set $\{\mathbf{x} \mid g(\mathbf{x}) = 0\}$.
            \item This formulation provides a foundation for subsequent extension to multi-class discriminant analysis.
        \end{itemize}
    \end{frame}



    \section{Linear Classifiers}

    \begin{frame}{Linear Classifiers}
        \begin{itemize}\itemsep1.5em
        \item \justifying \textbf{Definition:}\\
        Linear classifiers assign class labels using a decision function that is linear in the feature vector $\mathbf{x} \in \mathbb{R}^d$, or linear in a set of transformed features of $\mathbf{x}$.
        \item \justifying \textbf{Linearly separable data:}\\
        Data points that can be perfectly separated by a linear decision boundary.
        \item \textbf{General form:}
        \[
            g(\mathbf{x}) = \mathbf{w}^T \mathbf{x} + w_0,
        \]
        where $\mathbf{w}$ defines the orientation of the decision surface and $w_0$ determines its position.
        \end{itemize}
    \end{frame}


    \begin{frame}{Two-Category Classification}
        \minipage{.5\textwidth}
        \begin{itemize}\itemsep1.2em
        \item \textbf{Linear discriminant:}
        \[
            g(\mathbf{x}) = \mathbf{w}^T\mathbf{x} + w_0
        \]
        \item \(\mathbf{x} = [x_1, \dots, x_d]^T,\; \mathbf{w} = [w_1, \dots, w_d]^T,\; w_0:\text{ bias}\)
        \item \textbf{Decision rule:}
        \[
            \hat{y} =
            \begin{cases}
                C_1, & \text{if } g(\mathbf{x}) \ge 0 \\
                C_2, & \text{otherwise}
            \end{cases}
        \]
        \item \textbf{Decision surface:} $\mathbf{w}^T\mathbf{x} + w_0 = 0$
        \end{itemize}
        \endminipage
        \hfill
        \minipage{.48\textwidth}
        \begin{figure}[bh]
            \includegraphics[width=\linewidth]{pic/Figure_1.png}
        \end{figure}
        \endminipage
    \end{frame}


    \begin{frame}{Geometric Properties of Linear Decision Boundaries}
        \begin{itemize}\itemsep1.2em
        \item The decision boundary is a $(d-1)$-dimensional hyperplane in $\mathbb{R}^d$.
        \item \textbf{Properties:}
        \begin{itemize}
            \item Orientation is determined by the normal vector $\mathbf{w}/\|\mathbf{w}\|$.
            \item Bias $w_0$ controls the displacement along the normal vector.
        \end{itemize}
        \item Points on opposite sides of the hyperplane are assigned to different classes.
        \end{itemize}
        \begin{figure}
            \includegraphics[width=\linewidth]{pic/Figure_25.png}
        \end{figure}
    \end{frame}

    \begin{frame}{Nonlinear Decision Boundaries}
        \minipage{.46\textwidth}
        \begin{itemize}\itemsep1em
        \item \textbf{Problem:}\\ Many datasets cannot be separated by a linear hyperplane.
        \item \textbf{Feature Transformation:}\\ Map input vector $\mathbf{x}$ to a higher-dimensional space $\phi(\mathbf{x})$.
        \item \textbf{Resulting Decision Boundary:}\\ Linear in the transformed space, but nonlinear in the original feature space.
        \end{itemize}
        \endminipage
        \hfill
        \minipage{.5\textwidth}
        \begin{figure}
            \includegraphics[width=\textwidth]{pic/Figure_10.png}
        \end{figure}
        \endminipage
        \vfill
        \begin{tikzpicture}[remember picture,overlay]
            \node[anchor=south west, xshift=0.1cm, yshift=0.22cm] at (current page.south west) {
                \scriptsize Figure adapted from M. Soleymani, Machine Learning course, Sharif University of Technology.
            };
        \end{tikzpicture}
    \end{frame}


    \section{Perceptron}

    \begin{frame}{What Is Perceptron?}
        \minipage{.65\textwidth}
        \begin{itemize}
            \item \textbf{Perceptron Unit:}
            \medskip
            \begin{itemize}\itemsep1em
            \item \justifying \textbf{Basic Building Block:}
            The perceptron is the simplest form of an artificial neuron used for classification tasks.

            \item \justifying \textbf{Linear Classifier:}
            It computes a linear combination of input features followed by a threshold-based decision.

            \item \justifying \textbf{Binary Decision:}
            Outputs \(1\) if the weighted sum of inputs exceeds a threshold; otherwise outputs \(0\).

            \item \justifying \textbf{Components:}
            Consists of input features, associated weights, a bias term, and an activation function (commonly a step or sigmoid function).
            \end{itemize}
        \end{itemize}
        \endminipage
        \hfill
        \minipage{.3\textwidth}
        \begin{figure}
            \centering
            \includegraphics[width=\linewidth]{pic/Figure_18.png}
        \end{figure}
        \endminipage
        \vfill
        \begin{tikzpicture}[remember picture,overlay]
            \node[anchor=south west, xshift=0.1cm, yshift=0.22cm] at (current page.south west) {
                \scriptsize Figure adapted from \textit{Grokking Machine Learning}, L. G. Serrano.
            };
        \end{tikzpicture}
    \end{frame}


    \begin{frame}{Inspired by Biology}
        \begin{itemize}
            \item \textbf{Biological Motivation Behind Perceptron:}
            \medskip
            \begin{itemize}\itemsep1em
            \item \justifying \textbf{Inspired by Neurons}:
            Perceptron mimics the basic function of biological neurons in the brain.
            \begin{itemize}
                \item \justifying Input and Output, Activation Function.
            \end{itemize}
            \end{itemize}
        \end{itemize}
        \begin{figure}
            \centering
            \includegraphics[width=0.55\linewidth]{pic/Figure_19.png}
        \end{figure}
        \vfill
        \begin{tikzpicture}[remember picture,overlay]
            \node[anchor=south west, xshift=0.1cm, yshift=0.22cm] at (current page.south west) {
                \scriptsize Figure adapted from Grokking Machine Learning, L. G. Serrano.
            };
        \end{tikzpicture}
    \end{frame}

    \begin{frame}{Single Neuron}
        \minipage{.66\textwidth}
        \begin{itemize}
            \item \textbf{Single Neuron as a Linear Decision Boundary}
            \medskip
            \begin{itemize}\itemsep1em
            \item \textbf{Mathematical Form}:
            The output of a single neuron is computed as:
            \[
                y = f(\mathbf{w}^T \mathbf{x} + w_0)
            \]
            where:
            \begin{itemize}
                \item \( \mathbf{x} \) is the input vector.
                \item \( \mathbf{w} \) is the weight vector.
                \item \( w_0 \) is the bias term.
                \item \( f \) is an activation function (e.g., step function).
            \end{itemize}

            \item \textbf{Linear Separation}:
            A neuron defines a linear decision boundary: \\
            \(
            \mathbf{w}^T \mathbf{x} + w_0 \, = \, threshold \text{ (0 for step, 0.5 for sigmoid)}
            \)
            \item \textbf{Decision Rule}:
            $C_1$ if \( \mathbf{w}^T \mathbf{x} + w_0 \geq threshold \), otherwise $C_2$.
            \end{itemize}
        \end{itemize}
        \endminipage
        \hfill
        \minipage{.3\textwidth}
        \centering Class = \(f(\mathbf{w}^T \mathbf{x} + w_0)\)
        \begin{figure}
            \centering
            \includegraphics[width=.8\linewidth]{pic/Figure_20.png}
        \end{figure}
        \endminipage
        \vfill
        \begin{tikzpicture}[remember picture,overlay]
            \node[anchor=south west, xshift=0.1cm, yshift=0.22cm] at (current page.south west) {
                \scriptsize Figure adapted from Grokking Machine Learning, L. G. Serrano.
            };
        \end{tikzpicture}
    \end{frame}

    \begin{frame}{Limitations of a Single Perceptron}
        \begin{itemize}
            \item \textbf{What a Single Perceptron Can and Can't Do:}
            \medskip
            \begin{itemize}\itemsep1em
            \item \justifying \textbf{Performs Linear Separations:}
            A perceptron can handle linearly separable problems such as:
            \begin{itemize}
                \item AND operation
                \item OR operation
            \end{itemize}
            \minipage{.6\textwidth}
            \item \justifying \textbf{Fails on Non-Linear Problems:}
            A single perceptron fails to solve non-linear problems like XOR, as the data points cannot be separated by a straight line.
            \endminipage
            \hfill
            \minipage{.3\textwidth}
            \begin{figure}
                \centering
                \includegraphics[width=\linewidth]{pic/Figure_23.png}
            \end{figure}
            \endminipage
            \end{itemize}
        \end{itemize}
    \end{frame}



    \begin{frame}{Towards Complex Decision Boundaries}
        \begin{itemize}
            \item \textbf{Multi-Layer Perceptron (MLP):}
            \medskip
            \begin{itemize}\itemsep1em
            \item \justifying \textbf{Adding Layers for More Complexity}:
            An MLP consists of multiple layers of neurons that allow us to model more complex functions than a single neuron.
            \begin{itemize}
                \item \justifying Each layer introduces new decision boundaries, making it possible to separate non-linear data.
            \end{itemize}
            \item \justifying \textbf{Two-Layer Example:}
            \begin{itemize}
                \item Input Layer $\rightarrow$ Hidden Layer $\rightarrow$ Output Layer
                \item Hidden layer introduces non-linear transformations that enable complex decision regions.
            \end{itemize}
            \end{itemize}
        \end{itemize}
        \begin{figure}
            \centering
            \includegraphics[width=0.38\linewidth]{pic/Figure_21.png}
        \end{figure}
        \vfill
        \begin{tikzpicture}[remember picture,overlay]
            \node[anchor=south west, xshift=0.1cm, yshift=0.22cm] at (current page.south west) {
                \scriptsize Figures adapted from Grokking Machine Learning, L. G. Serrano.
            };
        \end{tikzpicture}
    \end{frame}


    \begin{frame}{Refining the Decision Boundary}
        \begin{itemize}
            \item \justifying \textbf{New Neurons for Better Separation}:
            By adding more neurons to a layer, we can further refine the decision boundary to better separate complex data.
            \item \justifying Each additional neuron introduces new features that help the model make more accurate decisions.
        \end{itemize}
        \begin{figure}
            \centering
            \includegraphics[width=0.53\linewidth]{pic/Figure_22.png}
        \end{figure}
        \vfill
        \begin{tikzpicture}[remember picture,overlay]
            \node[anchor=south west, xshift=0.1cm, yshift=0.22cm] at (current page.south west) {
                \scriptsize Figure adapted from Grokking Machine Learning, L. G. Serrano.
            };
        \end{tikzpicture}
    \end{frame}


    \section{Cost Functions}


    \begin{frame}{Cost Functions}
        \begin{itemize}
            \item \textbf{Understanding the Goal}
            \medskip
            \begin{itemize}\itemsep1em
            \item In the perceptron, we use \( \mathbf{w}^T \mathbf{x} \) to make predictions.
            \item Goal is to find the optimal \(\mathbf{w}\) so that the predicted labels match the true labels as much as possible.
            \item To achieve this, we define a cost function, which measures the \textbf{difference} between \textbf{predicted} and \textbf{actual} labels.
            \item Finding discriminant functions (\(\mathbf{w}^T\), \(w_0\)) is framed as minimizing a cost function.
            \smallskip
            \begin{itemize}\itemsep0.8em
            \item Based on training set \(D \, = \, \{(\mathbf{x}^{(i)},\, y^{(i)})\}^N_{i=1},\) a cost function \(J(\mathbf{w})\) is defined.
            \item Problem converts to finding optimal \(\hat{g}(\mathbf{x}) \, = \, g(\mathbf{x; \hat{w}})\) where \[\mathbf{\hat{w}} \, = \, \arg \min_{\mathbf{w}}J(\mathbf{w})\]
            \end{itemize}
            \end{itemize}
        \end{itemize}
    \end{frame}



    \begin{frame}{Sum of Squared Error Cost Function}
        \begin{itemize}
            \item \textbf{Sum of Squared Error (SSE) Cost Function}\itemsep1em
            \medskip
            \begin{itemize}\itemsep0.8em
            \item \textbf{Formula}:
            \(
            J(\mathbf{w}) = \sum_{i=1}^{n} (y^{(i)} - \hat{y}^{(i)})^2 \, , \quad \hat{y}^{(i)} \, = \, \mathbf{w}^T\mathbf{x}^{(i)} \, + \, w_0
            \)
            \item \justifying SSE minimizes the magnitude of the error, which is ideal for regression but \textcolor{red}{irrelevant} for classification.
            \item \justifying If the model predicts close to the true class but not exactly 0 or 1, SSE still shows positive error, even for correct predictions.
            \end{itemize}
        \end{itemize}
        \minipage{0.6\textwidth}
        \begin{itemize}
            \item SSE is also prone to overfitting noisy data, as small variations can cause significant changes in the cost.
        \end{itemize}
        \endminipage
        \minipage{.4\textwidth}
        \begin{figure}
            \centering
            \includegraphics[width=.62\linewidth]{pic/Figure_11.png}
        \end{figure}
        \endminipage
        \vfill
        \begin{tikzpicture}[remember picture,overlay]
            \node[anchor=south west, xshift=0.1cm, yshift=0.22cm] at (current page.south west) {
                \scriptsize Figures adapted from slides of M. Soleymani, Machine Learning course, Sharif University of Technology.
            };
        \end{tikzpicture}
    \end{frame}


    \begin{frame}{An Alternative for SSE Cost Function}
        \begin{itemize}
            \item \textbf{Number of Misclassifications}\itemsep1em
            \medskip
            \begin{itemize}\itemsep0.8em
            \item \textbf{Definition}:
            Measures how many samples are misclassified by the model.
            \item \textbf{Formula}:
            \[
                J(\mathbf{w}) = \sum_{i=1}^{n} (\frac{y^{(i)} - \text{sign(\(\hat{y}^{(i)})\)}}{2})^2 \, , \quad \hat{y}^{(i)} \, = \, \mathbf{w}^T\mathbf{x}^{(i)} \, + \, w_0 \, , \quad y^{(i)} \in \{-1, \, +1\}
            \]
            \item \textbf{Limitations}: \\
            \minipage{0.5\linewidth}
            \begin{itemize}
                \item \justifying \textbf{Piecewise Constant}:
                The cost function is non-differentiable, so optimization techniques (like gradient descent) cannot be directly applied.
            \end{itemize}
            \endminipage
            \hspace{1cm}
            \minipage{0.28\textwidth}
            \begin{figure}[bh]
                \includegraphics[width=\textwidth]{pic/Figure_12.png}
            \end{figure}
            \endminipage
            \end{itemize}
        \end{itemize}
        \vfill
        \begin{tikzpicture}[remember picture,overlay]
            \node[anchor=south west, xshift=0.1cm, yshift=0.22cm] at (current page.south west) {
                \scriptsize Figure adapted from Machine Learning and Pattern Recognition, Bishop
            };
        \end{tikzpicture}
    \end{frame}


    \begin{frame}{Perceptron Algorithm}
        \begin{itemize}
            \item \textbf{The Perceptron Algorithm}
            \medskip
            \begin{itemize}\itemsep1em
            \item \justifying \textbf{Purpose}:
            A simple algorithm for binary classification, separating two classes with a linear boundary.
            \end{itemize}
        \end{itemize}
        \begin{figure}
            \centering
            \includegraphics[width=0.7\linewidth]{pic/Figure_13.png}
        \end{figure}
        \vfill
        \begin{tikzpicture}[remember picture,overlay]
            \node[anchor=south west, xshift=0.1cm, yshift=0.22cm] at (current page.south west) {
                \scriptsize Figures adapted from Machine Learning and Pattern Recognition, Bishop
            };
        \end{tikzpicture}
    \end{frame}


    \begin{frame}{Perceptron Criterion}
        \begin{itemize}\itemsep1.2em
        \item \textbf{Cost Function}:
        The perceptron criterion focuses on misclassified points:
        \[
            J_p(\mathbf{w}) = - \sum_{i \in M} y^{(i)} \, \mathbf{w}^T \mathbf{x}^{(i)} \, , \quad y^{(i)} \in \{-1, \, +1\}
        \]
        where \( M \) is the set of misclassified points.
        \item \textbf{Goal}:
        Minimize the loss by correctly classifying all points.
        \end{itemize}
    \end{frame}


    \begin{frame}{Batch Perceptron}
        \begin{itemize}\itemsep1.5em
        \item \justifying \textbf{Batch Perceptron}:
        Updates the weight vector using all misclassified points in each iteration.
        \item \justifying \textbf{Gradient Descent}:
        Adjusting weights in the direction that reduces the loss:
        \[
            \mathbf{w} \leftarrow \mathbf{w} - \eta \nabla_\mathbf{w} \, J_p(\mathbf{w})
        \]
        \[
            \nabla_\mathbf{w} \, J_p(\mathbf{w}) \, = \, - \sum_{i \in M} y_i \mathbf{x}_i
        \]
        \begin{itemize}
            \item Batch Perceptron converges in finite number of steps for linearly separable data.
        \end{itemize}
        \end{itemize}
    \end{frame}

    \begin{frame}{Single-sample Perceptron}
        \begin{itemize}\itemsep1.5em
        \item \justifying \textbf{Single Sample Perceptron}: Updates the weight vector after each individual point.
        \item \textbf{Stochastic Gradient Descent (SGD) Update Rule:}
        \smallskip
        \begin{itemize}\itemsep1em
        \item Using only one misclassified sample at a time:
        \[
            \mathbf{w} \leftarrow \mathbf{w} + \eta y_i \mathbf{x}_i
        \]
        \item Lower computational cost per iteration, faster convergence.
        \item \justifying If training data are linearly separable, the single-sample perceptron is also guaranteed to find a solution in a finite number of steps.
        \end{itemize}
        \end{itemize}
    \end{frame}

    \begin{frame}{Example}
        \begin{itemize}
            \item Perceptron changes \(\mathbf{w}\) in a direction that corrects error.
        \end{itemize}
        \begin{figure}
            \centering
            \includegraphics[width=\linewidth]{pic/Figure_14.png}
        \end{figure}
        \vfill
        \begin{tikzpicture}[remember picture,overlay]
            \node[anchor=south west, xshift=0.1cm, yshift=0.22cm] at (current page.south west) {
                \scriptsize Figures adapted from Grokking Machine Learning, L. G. Serrano.
            };
        \end{tikzpicture}
    \end{frame}

    \begin{frame}{Convergence of the Perceptron — Theorem}

        \textbf{Theorem:}
        For linearly separable data with margin $\gamma > 0$ and $\|x_i\| \le R$,
        the Perceptron algorithm makes at most $M \le \tfrac{R^2}{\gamma^2}$ updates.

        \vspace{0.8em}

        \textbf{Notation:}


        \begin{itemize}
            \item Dataset: $D = \{(x_i, y_i)\}_{i=1}^n$, with $x_i \in \mathbb{R}^d$, $y_i \in \{+1, -1\}$.
            \item Weight vector at step $t$: $w_t$, starting from $w_0 = 0$.
            \item Update rule (on mistake): $w_{t+1} = w_t + y_i x_i$.
            \item Assume there exists $w^*$ with $\|w^*\| = 1$ that correctly classifies all samples.
            \item Each input is bounded: $\|x_i\| \le R$ (after scaling, $R = 1$).
            \item Margin:
            \[
                \gamma = \min_{(x_i, y_i) \in D} y_i (x_i^\top w^*) > 0.
            \]
        \end{itemize}

    \end{frame}


    \begin{frame}{Convergence of the Perceptron — Proof (1)}

        Let the algorithm make $M$ mistakes.

        \begin{enumerate}
            \item Each mistake on $(x_i, y_i)$ updates
            \[
                w_{t+1} = w_t + y_i x_i.
            \]
            \item By induction, $w_M = \sum_{t=1}^{M} y_t x_t$.
            \item Inner product with $w^*$:
            \[
                w_M \cdot w^* = \sum_{t=1}^{M} y_t (x_t \cdot w^*) \ge M \gamma.
            \]
        \end{enumerate}

    \end{frame}



    \begin{frame}{Convergence of the Perceptron — Proof (2)}

        \begin{enumerate}
            \setcounter{enumi}{3}
            \item Norm growth:
            \[
                \|w_M\|^2 = \|w_{M-1}\|^2 + 2y_M (w_{M-1}\cdot x_M) + \|x_M\|^2
                \le \|w_{M-1}\|^2 + R^2.
            \]
            Hence, $\|w_M\| \le R\sqrt{M}$.
            \item Combine inequalities:
            \[
                M\gamma \le w_M\cdot w^* \le \|w_M\| \|w^*\| \le R\sqrt{M}.
            \]
            \item Therefore,
            \[
                M \le \frac{R^2}{\gamma^2}.
            \]
        \end{enumerate}

        \vfill
        \footnotesize
        Source: Novikoff (1962), \textit{On Convergence Proofs for Perceptrons};
        M. Collins, \textit{Convergence Proof for the Perceptron Algorithm}, Columbia University.
    \end{frame}



    \begin{frame}{Convergence of Perceptron Cont.}
        \begin{itemize}\itemsep1.5em
        \item \textbf{Non-Linearly Separable Data}:
        When no linear decision boundary can perfectly separate the classes, the Perceptron fails to converge.
        \medskip
        \begin{itemize}\itemsep1em
        \minipage{0.5\textwidth}
        \item If data is not linearly separable, there will always be some points that the model fails to classify.
        \item As a result, the algorithm keeps adjusting the weights to fix the misclassified points, causing it to never converge.
        \item For the data that are not linearly separable due to noise, \textbf{Pocket Algorithm} keeps in its pocket the best \(\mathbf{w}\) encountered up to now.
        \endminipage
        \minipage{0.4\textwidth}
        \hspace{1.5cm}
        \begin{figure}
            \centering
            \includegraphics[width=.9\linewidth]{pic/Figure_15.png}
        \end{figure}
        \endminipage
        \end{itemize}
        \end{itemize}
        \vfill
        \begin{tikzpicture}[remember picture,overlay]
            \node[anchor=south west, xshift=0.1cm, yshift=0.22cm] at (current page.south west) {
                \scriptsize Figure adapted from Machine Learning and Pattern Recognition, Bishop
            };
        \end{tikzpicture}
    \end{frame}

    \begin{frame}{Pocket Algorithm}
        \begin{algorithm}[H]
            \caption{Pocket Algorithm}\label{alg:Pocket Algorithm}
            \begin{algorithmic}[1]
                \State \textbf{Initialize} $\mathbf{w}$
                \For{$t = 1$ to $T$}
                    \State \(i \leftarrow t \text{ mod } N\)
                    \If{\(\mathbf{x}^{(i)}\) is misclassified}
                        \State \(\mathbf{w}^{new} \, = \, \mathbf{w} + \eta \mathbf{x}^{(i)}y^{(i)}\)
                        \If{\(E_{train}(\mathbf{w}^{new}) \, < \, E_{train}(\mathbf{w})\)} \Comment{\(E_{train}(\mathbf{w}) \, = \, J_p(\mathbf{w})\)}
                        \State \(\mathbf{w} \, = \, \mathbf{w}^{new}\)
                        \EndIf
                    \EndIf
                \EndFor
            \end{algorithmic}
        \end{algorithm}
    \end{frame}


    \section{Imbalanced Data}

    \begin{frame}{The Problem: Imbalanced Data}
        \begin{itemize}\itemsep1em
        \item Real-world datasets often contain classes with unequal representation.
        \item \textbf{Examples:}
        \begin{itemize}
            \item Fraud detection: 0.1\% Fraud, 99.9\% Non-Fraud
            \item Medical diagnosis: 2\% Disease, 98\% Healthy
        \end{itemize}
        \item \textbf{Issue:} High accuracy may hide poor performance on the minority class.
        \end{itemize}
        \begin{center}
            \includegraphics[width=0.55\textwidth]{pic/figure_32.png}\\
        \end{center}
    \end{frame}

    \begin{frame}{Solution 1: Resampling Techniques}
        \begin{columns}[T]
            \begin{column}{0.48\textwidth}
                \textbf{Undersampling}
                \begin{itemize}\itemsep0.8em
                \item Remove samples from majority class.
                \item \textbf{+} Reduces training time.
                \item \textbf{--} Risk of losing information.
                \end{itemize}
                \vspace{0.8em}
                \includegraphics[width=\textwidth]{pic/Figure_33.png}\\
                \scriptsize Majority class reduced.
            \end{column}

            \begin{column}{0.48\textwidth}
                \textbf{Oversampling (e.g., SMOTE)}
                \begin{itemize}\itemsep0.8em
                \item Generate synthetic minority samples.
                \item \textbf{+} Preserves information.
                \item \textbf{--} May cause overfitting.
                \end{itemize}
                \vspace{0.8em}
                \includegraphics[width=\textwidth]{pic/Figure_34.png}\\
                \scriptsize New minority points added via interpolation.
            \end{column}
        \end{columns}
    \end{frame}

    \begin{frame}{Solution 2: Algorithmic Approaches}

        \textbf{Weighted Loss Function}

        \begin{itemize}\itemsep0.8em
        \item Assign a higher penalty to errors on the minority class.
        \item Encourages the model to focus on rare but important cases.
        \end{itemize}

        \vspace{0.5em}

        \begin{columns}[c,onlytextwidth]
            \column{0.55\textwidth}
            \[
                J(w) = - \sum_{i} w_{y^{(i)}} \, y^{(i)} \log(\hat{y}^{(i)})
            \]
            \scriptsize
            \begin{center}
                where $w_{y^{(i)}}$ is larger for the minority class.
            \end{center}

            \column{0.45\textwidth}
            \centering
            \includegraphics[width=0.8\linewidth]{pic/Figure_36.png}\\
        \end{columns}

    \end{frame}



    \begin{frame}{Hybrid Approaches}

        \textbf{Combining Sampling Strategies}

        \begin{itemize}\itemsep0.8em
        \item Integrates both oversampling and undersampling to achieve better class balance.
        \end{itemize}

        \vspace{0.5em}

        \textbf{Typical Pipeline:}
        \begin{enumerate}\itemsep0.6em
        \item Apply \textbf{SMOTE} to synthetically augment minority samples.
        \item Randomly undersample the majority class to reduce imbalance.
        \item Train the model on the newly balanced dataset.
        \end{enumerate}

        \vspace{0.6em}

        \begin{itemize}\itemsep0.6em
        \item Provides a practical trade-off between \textbf{bias} and \textbf{variance}.
        \end{itemize}

    \end{frame}


    \begin{frame}{Comparison of Methods}

        \begin{columns}[c,onlytextwidth]
            \column{0.55\textwidth}
            \textbf{Summary}

            \begin{itemize}\itemsep0.6em
            \item \textbf{Sampling} — modifies data distribution\\
            \textcolor{green!60!black}{\small(+ Simple, improves balance)}\\
            \textcolor{red!70!black}{\small(– Risk of over/underfitting)}

            \item \textbf{Algorithmic} — adjusts model focus  \\
            \textcolor{green!60!black}{\small(+ No data change, interpretable)}\\
            \textcolor{red!70!black}{\small(– Needs careful weight tuning)}

            \item \textbf{Hybrid} — combines both strategies  \\
            \textcolor{green!60!black}{\small(+ Balanced, strong results)}  \\
            \textcolor{red!70!black}{\small(– More complex, slower training)}
            \end{itemize}

            \column{0.43\textwidth}
            \centering
            \includegraphics[width=0.9\linewidth]{pic/Figure_35.png}\\
            \scriptsize Visual summary of trade-offs
        \end{columns}

    \end{frame}





    \section{Cross Validation}

    \begin{frame}{Model Selection via Cross Validation}
        \begin{itemize}
            \item \textbf{Cross-Validation}
            \medskip
            \begin{itemize}\itemsep1em
            \item \justifying \textbf{Purpose}:
            Technique for evaluating how well a model generalizes to unseen data.
            \item \justifying \textbf{How It Works}:
            Split data into $k$ folds; train on $k-1$ folds and validate on the remaining fold.
            \item \justifying \textbf{Repeat Process}:
            Repeat $k$ times, rotating the test fold each time. Average of all scores is the final score of the model.
            \item \justifying Cross-validation
            reduces overfitting and provides a more reliable estimation of model performance.
            \end{itemize}
        \end{itemize}
    \end{frame}


    \begin{frame}{K-Fold Cross Validation}
        \begin{figure}
            \centering
            \includegraphics[width=.8\textwidth]{pic/Figure_28.png}
        \end{figure}
        \vfill
        \begin{tikzpicture}[remember picture,overlay]
            \node[anchor=south west, xshift=0.1cm, yshift=0.22cm] at (current page.south west) {
                \scriptsize Figure adapted from Introduction to Support Vector Machines and Kernel Methods, J.M. Ashfaque.
            };
        \end{tikzpicture}
    \end{frame}


    \begin{frame}{Leave-One-Out Cross-Validation (LOOCV)}
        \begin{itemize}
            \item \textbf{Leave-One-Out Cross-Validation (LOOCV)}
            \medskip
            \begin{itemize}\itemsep1em
            \item \justifying \textbf{How It Works}:
            Uses a single data point as the validation set ($k$ = 1) and the rest as the training set. Repeat for all data points.
            \item \textbf{Properties:}
            \smallskip
            \begin{itemize}\itemsep.5em
            \item \textbf{No Data Wastage}:
            Every data point is used for both training and validation.
            \item \textbf{High Variance, Low Bias}.
            \item \justifying \textbf{Computationally Expensive}:
            Requires training the model $N$ times for $N$ data points, making it slow for large datasets.
            \item \textbf{Best for small datasets}.
            \end{itemize}
            \end{itemize}
        \end{itemize}
    \end{frame}

    \begin{frame}{Cross-Validation for Choosing Regularization Term}
        \begin{figure}
            \centering
            \includegraphics[width=0.8\linewidth]{pic/Figure_29.png}
        \end{figure}
        \vfill
        \begin{tikzpicture}[remember picture,overlay]
            \node[anchor=south west, xshift=0.1cm, yshift=0.22cm] at (current page.south west) {
                \scriptsize Figures adapted from slides of F.Salehi, Machine Learning course, Sharif University of Technology.
            };
        \end{tikzpicture}
    \end{frame}

    \begin{frame}{Cross-Validation for Choosing Model Complexity}
        \begin{figure}
            \centering
            \includegraphics[width=0.8\linewidth]{pic/Figure_16.png}
        \end{figure}
    \end{frame}

    \section{Multi-Category Classification}

    \begin{frame}{Multi-Category Classification}
        \begin{itemize}
            \item \textbf{Solutions to multi-category classification problem}:
            \medskip
            \begin{itemize}\itemsep1.5em
            \item Extend the learning algorithm to support multi-class.
            \medskip
            \begin{itemize}\itemsep1em
            \item First, a function \(g_i\) for every class \(C_i\) is found.
            \item Second, \(\mathbf{x}\) is assigned to \(C_i\) if \(g_i(\mathbf{x}) > g_j(\mathbf{x}) \quad \forall i \neq j\)
            \item[] \[\hat{y} \, = \, \argmax_{i=\text{1,...,c}} \, g_i(\mathbf{x})\]
            \end{itemize}
            \item Convert to a set of two-categorical problems.
            \medskip
            \begin{itemize}
                \item Methods like \textbf{One-vs-Rest} or \textbf{One-vs-One}, where each classifier distinguishes between either \textbf{one class and the rest}, or \textbf{between pairs of classes}.
            \end{itemize}
            \end{itemize}
        \end{itemize}
    \end{frame}

    \begin{frame}{Multi-Category Classification: Ambiguity}
        \begin{itemize}
            \item \justifying One-vs-One and One-vs-Rest conversion can lead to regions in which the classification is \textbf{undefined}.
        \end{itemize}
        \vfill
        \minipage{.45\textwidth}
        \begin{figure}[bh]
            \centering
            \includegraphics[width=\linewidth]{pic/Figure_6.png}
        \end{figure}
        \endminipage
        \minipage{.45\textwidth}
        \begin{figure}[bh]
            \centering
            \includegraphics[width=\linewidth]{pic/Figure_7.png}
        \end{figure}
        \endminipage
        \vfill
        \begin{tikzpicture}[remember picture,overlay]
            \node[anchor=south west, xshift=0.1cm, yshift=0.22cm] at (current page.south west) {
                \scriptsize Figures adapted from Machine Learning and Pattern Recognition, Bishop
            };
        \end{tikzpicture}
    \end{frame}


    \begin{frame}{Multi-Category Classification: Linear Machines}
        \begin{itemize}\itemsep1.5em
        \item \justifying \textbf{Linear Machines}: Alternative to One-vs-Rest and One-vs-One methods;
        Each class is represented by its own discriminant function.
        \item \justifying \textbf{Decision Rule:}
        \[\hat{y} \, = \, \argmax_{i=\text{1,...,c}} \, g_i(\mathbf{x})\]
        The predicted class is the one with the highest discriminant function value.
        \item \textbf{Decision Boundary}:
        \(g_i(\mathbf{x}) \, = \, g_j(\mathbf{x})\)
        \[(\mathbf{w}_i - \mathbf{w}_j)^T\mathbf{x} + (w_{0i} - w_{0j}) = 0\]
        \end{itemize}
    \end{frame}


    \begin{frame}{Linear Machines Cont.}
        \begin{center}
            \minipage{.5\textwidth}
            \begin{figure}
                \includegraphics[width=\textwidth]{pic/Figure_9.png}
            \end{figure}
            \endminipage
        \end{center}
        \hspace{4cm}
        \begin{itemize}\itemsep1.5em
        \item \justifying The decision regions of this discriminant are \textbf{convex} and \textbf{singly connected}. Any point on the line between two points within the same region can be expressed as \\
        \(
        \mathbf{x} = \lambda \mathbf{x}_A + (1 - \lambda) \mathbf{x}_B
        \)
        where \( \mathbf{x}_A, \mathbf{x}_B \in C_k \).

        \end{itemize}
    \end{frame}

    \begin{frame}{Multi-Class Perceptron Algorithm}
        \begin{itemize}
            \item \textbf{Weight Vectors}:
            \medskip
            \begin{itemize}\itemsep.8em
            \item Maintain a weight matrix \( W \in \mathbb{R}^{m \times K} \), where \( m \) is the number of features and \( K \) is the number of classes.
            \item Each column \( w_k \) of the matrix corresponds to the weight vector for class \( k \).
            \end{itemize}
        \end{itemize}
        \hspace{4cm}
        \[
            \hat{y} \, = \, \argmax_{i=1,...,c} \, \mathbf{w}_i^T \mathbf{x}
        \]
        \[
            J_p(\mathbf{W}) \, = \, - \sum_{i \in M}(\mathbf{w}_{y^{(i)}} - \mathbf{w}_{\hat{y}^{(i)}})^T\mathbf{x}^{(i)}
        \]
        where \( M \) is the set of misclassified points.
    \end{frame}


    \begin{frame}{Multi-Class Perceptron Algorithm}

        \begin{algorithm}[H]
            \caption{Multi-class perceptron}\label{alg:Multi-class perceptron}
            \begin{algorithmic}[1]
                \State \textbf{Initialize} $\mathbf{W} \, = \, [\mathbf{w}_1 \, , \, ... \, , \, \mathbf{w}_c], \, k \leftarrow 0$
                \While{A pattern is misclassified}
                    \State \(k \leftarrow k \, + \, 1 \text{ mod } N\)
                    \If{\(\mathbf{x}^{(i)}\) is misclassified}
                        \State \(\mathbf{w}_{\hat{y}^{(i)}} \, = \, \mathbf{w}_{\hat{y}^{(i)}} - \eta \mathbf{x}^{(i)}\)
                        \State \(\mathbf{w}_{y^{(i)}} \, = \, \mathbf{w}_{y^{(i)}} + \eta \mathbf{x}^{(i)}\)
                    \EndIf
                \EndWhile
            \end{algorithmic}
        \end{algorithm}

    \end{frame}


    \section{References}


    \begin{frame}{Contributions}
        \begin{itemize}
            \item \textbf{This slide has been prepared thanks to:}
            \medskip
            \begin{itemize}
                \item \href{https://github.com/jefri021}{Erfan Jafari}
                \item {Aida Jalali}
            \end{itemize}
        \end{itemize}

    \end{frame}


    \begin{frame}[allowframebreaks]
        \bibliography{ref}
        \bibliographystyle{ieeetr}
        \nocite{*} % used here because no citation happens in slides
        % if there are too many try use:
        % \tiny\bibliographystyle{alpha}
    \end{frame}

\end{document}