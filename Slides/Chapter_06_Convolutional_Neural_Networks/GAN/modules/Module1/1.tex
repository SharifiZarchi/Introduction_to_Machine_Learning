\section{Introduction}
\frame{\frametitle{Generative Adversarial Networks - The intuition}}



\begin{frame}
	\begin{columns}
		\column{0.5\textwidth}
		\begin{overlayarea}{\textwidth}{1\textheight}
		\vspace*{0mm}
			\begin{center}
			\onslide<1->{\scalebox{0.35}{\centering
\vspace{0.5cm}
\tikzstyle{neuronv}=[circle,minimum size=20pt,inner sep=0pt, thick, fill=orange!30, draw=red!50]
\tikzstyle{neuronh}=[circle,minimum size=20pt,inner sep=0pt, thick, fill=blue!20, draw=blue!60]
\tikzstyle{stateTransition}=[thick]
\tikzstyle{learned}=[text=black]
\begin{tikzpicture}[scale=1.9]
    % \draw ;
    \draw[rounded corners=0.5cm, draw=red!60, thick] (-0.4, -0.25) rectangle (2.5, 0.25) {};
    \draw[rounded corners=0.5cm, draw=red!60, thick] (-0.4, 1.25) rectangle (2.5, 1.75) {};

    \node (v1)[neuronv] at (0, 0) {$v_1$};
    \node (v2)[neuronv] at (0.7, 0) {$v_2$};
    \node (v3)[] at (1.4, 0) {$\cdots$};
    \node (v4)[neuronv] at (2.1, 0) {$v_m$};
    \node[below=0.5cm of v2] (v) {$V \in \{0, 1\}^m$};
    \node[learned,below=0.1cm of v1] (bv1) {$b_1$};
    \node[learned,below=0.1cm of v2] (bv2) {$b_2$};
    % \node[learned,below=0.1cm of v3, scale=0.7] (bv3) {$b_{v_3}$};
    \node[learned,below=0.1cm of v4] (bv4) {$b_m$};

    \node (h1)[neuronh] at (0, 1.5) {$h_1$};
    \node (h2)[neuronh] at (0.7, 1.5) {$h_2$};
    \node (h3)[] at (1.4, 1.5) {$\cdots$};
    \node (h4)[neuronh] at (2.1, 1.5) {$h_n$};
    \node[above=0.5cm of h2] (h) {$H \in \{0, 1\}^n$};
    \node[learned,above=0.1cm of h1] (bv1) {$c_1$};
    \node[learned,above=0.1cm of h2] (bv2) {$c_2$};
    % \node[learned,below=0.1cm of v3, scale=0.7] (bv3) {$b_{v_3}$};
    \node[learned,above=0.1cm of h4] (bv4) {$c_n$};

    \node[learned, scale=0.7] (W) at (2.5, 0.75) {$W \in \mathbb{R}^{m \times n}$};

    \draw[learned,stateTransition] (0,0.17) -- (0,1.33) node [midway,left=-0.1cm] {$w_{1,1}$};
    \draw[stateTransition] (0,0.17) -- (0.7,1.33) node [midway,above=-0.06cm,sloped] {};
    \draw[stateTransition] (0,0.17) -- (2.1,1.33) node [midway,above=-0.06cm,sloped] {};

    \draw[stateTransition] (0.7,0.17) -- (0,1.33) node [midway,above=-0.06cm,sloped] {};
    \draw[stateTransition] (0.7,0.17) -- (0.7,1.33) node [midway,above=-0.06cm,sloped] {};
    \draw[stateTransition] (0.7,0.17) -- (2.1,1.33) node [midway,above=-0.06cm,sloped] {};

    \draw[stateTransition] (2.1,0.17) -- (0,1.33) node [midway,above=-0.06cm,sloped] {};
    \draw[stateTransition] (2.1,0.17) -- (0.7,1.33) node [midway,above=-0.06cm,sloped] {};
    \draw[learned,stateTransition] (2.1,0.17) -- (2.1,1.33) node [midway,left=-0.1cm] {$w_{m,n}$};

\end{tikzpicture}}}
			\onslide<1->{\scalebox{0.35}{\centering
\tikzstyle{input_neuron}=[circle,draw=red!50,fill=red!10,thick,minimum size=6mm]
\tikzstyle{hidden_neuron}=[circle,draw=blue!50,fill=cyan!10,thick,minimum size=6mm]
\tikzstyle{output_neuron}=[circle,draw=green!50,fill=green!10,thick,minimum size=6mm]

\tikzstyle{input}=[circle,draw=black!50,fill=black!20,thick,minimum size=6mm]

% \begin{center}
% \scalebox{0.7}{
\begin{tikzpicture}

\node [input_neuron] (neuron01) at (6.5,4.5) {};
\node [input_neuron] (neuron02) at (7.5,4.5){};
\node [input_neuron] (neuron03) at (8.5,4.5) {};
\node [input_neuron] (neuron04) at (9.5,4.5) {};
\node [input_neuron] (neuron05) at (10.5,4.5) {};
\node [hidden_neuron] (neuron51) at (6.5,6) {} ;
\node [hidden_neuron] (neuron52) at (7.5,6)  {};
\node [hidden_neuron] (neuron53) at (9.5,6)  {};
\node [hidden_neuron] (neuron54) at (10.5,6)  {};

\node [input_neuron] (neuron11) at (6.5,11)  {};
\node [input_neuron] (neuron12) at (7.5,11)  {};
\node [input_neuron] (neuron13) at (8.5,11)  {};
\node [input_neuron] (neuron14) at (9.5,11)  {};
\node [input_neuron] (neuron15) at (10.5,11)  {};

\node [output_neuron] (neuron17) at (8.5,7.9)  {\Large$\ast$};
\node [output_neuron] (neuron18) at (8.5,9.1)  {\large$+$};
\node [] (neuron19) at (9.8,7.9)  {\large$\epsilon$};
\node [] at (8.3,9.7)  {$z$};

\node[text width=0.01cm] at (8.3,3.3) {$X$};
% \node[text width=0.007cm] at (7.7,5.25) {$\theta$};
\node[text width=0.007cm] at (11.2,4.5) {$Q_\theta(z|X)$};
\node[text width=0.01cm] at (11.2,6) {$\Sigma$};
\node[text width=0.01cm] at (5.6,6) {$\mu$};
% \node[text width=0.01cm] at (10.7,6) {$\mathbf{z}$};
% \node[text width=0.007cm] at (7.7,6.75) {$W^*$};
\node[text width=0.007cm] at (11.2,11) {$P_\phi(X|z)$};
\node[text width=0.01cm] at (8.3,12.2) {$\hat{X}$};
	
\draw[red!100,thick,solid,rounded corners=15pt] (6,4) rectangle (11,5);
% \draw[red!100,thick,solid,rounded corners=15pt] (6.5,5.5) rectangle (10.5,6.5);
\draw[red!100,thick,solid,rounded corners=15pt] (6,5.5) rectangle (8.2,6.5);
\draw[red!100,thick,solid,rounded corners=15pt] (8.8,5.5) rectangle (11,6.5);
\draw[red!100,thick,solid,rounded corners=15pt] (6,10.5) rectangle (11,11.5);

\draw[thick,->] (8.5,3.5) -- (8.5,4);

\draw[thick,->] (8.5,5) -- (7.2,5.5);

\draw[thick,->] (8.5,5) -- (9.8,5.5);

\draw[thick,->] (7.2,6.5) -- (neuron18);

\draw[thick,->, brown!100] (neuron19) -- (neuron17);

\draw[thick,->] (9.8,6.5) -- (8.5,7.55);

\draw[thick,->] (8.5,8.25) -- (8.5,8.75);

\draw[thick,->] (8.5,9.45) -- (8.5,10.5);

\draw[thick,->] (8.5,11.5) -- (8.5,12);

\end{tikzpicture}
% }
% \end{center}}}\\
			\onslide<1->{\vspace*{5mm} \scalebox{0.45}{\tikzstyle{neuron}=[circle,draw=red!50,fill=red!50,thick,minimum size=6mm]
\tikzstyle{neuron_red}=[circle,draw=red!50,fill=red!50,thick,minimum size=6mm]
\tikzstyle{neuron_green}=[circle,draw=green!50,fill=green!50,thick,minimum size=6mm]
\tikzstyle{neuron_blue}=[circle,draw=blue!50,fill=blue!50,thick,minimum size=6mm]
\tikzstyle{neuron_brown}=[circle,draw=brown!50,fill=brown!50,thick,minimum size=6mm]
\tikzstyle{neuron_gray}=[circle,draw=gray!20,fill=gray!20,thick,minimum size=6mm]

\tikzstyle{hidden_neuron}=[circle,draw=blue!50,fill=cyan!10,thick,minimum size=4mm]
\tikzstyle{output_neuron}=[circle,draw=green!50,fill=green!10,thick,minimum size=6mm]

\hspace*{-1cm}\begin{tikzpicture}
  %%%% Neurons
  \node [neuron_red]   (neuron1) at (0.0,0) {\small $x_1$};
  \node [neuron_green] (neuron2) at (1.5,0) {\small $x_2$};
  \node [neuron_blue]  (neuron3) at (3.0,0) {\small $x_3$};
  \node [neuron_brown]  (neuron4) at (4.5,0) {\small $x_4$};

  \node [neuron_red] (oneuron1) at (0.0,6) {};
  \node [neuron_green] (oneuron2) at (1.5,6) {};
  \node [neuron_blue] (oneuron3) at (3.0,6) {};
  \node [neuron_brown] (oneuron4) at (4.5,6) {};

  \node [rotate=30, text width=1.5](oneuron1l) at (0.0-0.5,6.6) {\small $p(x_1)$};
  \node [rotate=30, text width=1.5](oneuron2l) at (1.5-0.5,6.6) {\small $p(x_2|x_1)$};
  \node [rotate=30, text width=1.5](oneuron3l) at (3.0-0.5,6.6) {\small $p(x_3|x_1,x_2)$};
  \node [rotate=30, text width=1.5](oneuron4l) at (4.5-0.5,6.6) {\small $p(x_4|x_1,x_2,x_3)$};

  \def\HIDDEN{$h_1$,$h_2$,$h_3$,$h_4$}
  \newcommand*{\xpos}{0}
  \newcommand*{\rectstarty}{1.3}
  \newcommand*{\rectendy}{4.7}
  \foreach \x [count=\xi] in \HIDDEN
  {
	\renewcommand*{\xpos}{-2+\xi*1.5}
	\draw[rounded corners,fill=cyan!20] (\xpos-0.4, \rectstarty) rectangle (\xpos+0.4, \rectendy) {};
  	\node (N) at (\xpos,4.1) {\x};
  	\node[hidden_neuron] (N\xi1) at (\xpos,1+1*0.65) {};
  	\node[hidden_neuron] (N\xi2) at (\xpos,1+2*0.65) {};
  	\node[hidden_neuron] (N\xi3) at (\xpos,1+3*0.65) {};
  	\node[hidden_neuron] (N\xi4) at (\xpos,1+4*0.65) {};
  }

  %%%% Labels
  \node (V) at (-0-0.7,\rectendy+0.4) {\scalebox{1.1}{$V$}};
  \node (W) at (-0-0.7,\rectstarty-0.4) {\scalebox{1.1}{$W$}};

  %%%% Edges
  \draw[->,thick,red] (neuron1) -- (-2+2*1.5-0.2,\rectstarty);
  \draw[->,thick,red] (neuron1) -- (-2+3*1.5-0.2,\rectstarty);
  \draw[->,thick,red] (neuron1) -- (-2+4*1.5-0.2,\rectstarty);

  \draw[->,thick,green] (neuron2) -- (-2+3*1.5,\rectstarty);
  \draw[->,thick,green] (neuron2) -- (-2+4*1.5,\rectstarty);

  \draw[->,thick,blue] (neuron3) -- (-2+4*1.5+0.2,\rectstarty);

  \draw[->,thick,red!70] (-2+1*1.5,\rectendy)--(oneuron1);
  \draw[->,thick,green!70] (-2+2*1.5,\rectendy)--(oneuron2);
  \draw[->,thick,blue!70] (-2+3*1.5,\rectendy)--(oneuron3);
  \draw[->,thick,brown!70] (-2+4*1.5,\rectendy)--(oneuron4);


\end{tikzpicture}}}
			\end{center}
		\end{overlayarea}

		\column{0.5\textwidth}
		\begin{overlayarea}{\textwidth}{\textheight}
			\begin{itemize}[<+->]
				\item<1-> So far we have looked at generative models which explicitly model the joint probability distribution or conditional probability distribution 
				\item<2-> For example, in RBMs we learn $P(X,H)$, in VAEs we learn $P(z|X)$ and $P(X|z)$ whereas in AR models we learn $P(X)$
     		\item<3-> What if we are only interested in sampling from the distribution and don't really care about explicit density function $P(X)$?
				\item<4-> What does this mean? \onslide<5->Let us see

			\end{itemize}
		\end{overlayarea}
	\end{columns}
\end{frame}



\begin{frame}

	\begin{columns}

		\column{1\textwidth}
		\begin{center}
			\begin{figure}
				\includegraphics[scale=0.3]{images/mnist_ground.png}
			\end{figure}
		\end{center}
		
		\begin{overlayarea}{\textwidth}{\textheight}
			\begin{itemize}[<+->]
			\item As usual we are given some training data (say, MNIST images) which obviously comes from some underlying distribution
			\item Our goal is to generate more images from this distribution (\textit{i.e.}, create images which look similar to the images from the training data)
			\item In other words, we want to sample from a complex high dimensional distribution which is intractable (recall RBMs, VAEs and AR models deal with this intractability in their own way)

			\end{itemize}
		\end{overlayarea}
	\end{columns}
\end{frame}

\begin{frame}
		\begin{overlayarea}{\textwidth}{\textheight}
		\vspace*{5mm}
		\begin{center}
			
\tikzstyle{hidden_neuron}=[circle,draw=blue!50,fill=cyan!10,thick,minimum size=4mm]
\begin{tikzpicture}
\newcommand*{\xpos}{10}
\newcommand*{\rectstarty}{1.3}
\newcommand*{\rectendy}{4}

\renewcommand*{\xpos}{20}
\draw[rounded corners,fill=cyan!20] (\xpos-0.4, \rectstarty) rectangle (\xpos+0.4, \rectendy) {};
\node (N) at (\xpos,2) {};
\node[hidden_neuron] (N1) at (\xpos,1+1*0.65) {};
\node[hidden_neuron] (N2) at (\xpos,1+2*0.65) {};
\node[hidden_neuron] (N3) at (\xpos,1+3*0.65) {};
\node[hidden_neuron] (N4) at (\xpos,1+4*0.65) {};
\node (N) at (\xpos,1) {\small{$z \sim N(0,I)$}};

\draw[->,thick,black] (\xpos+0.4,2.7) -- (\xpos+0.4+1.5,2.7);

\draw[rounded corners,fill=green!12] (\xpos+0.4+1.5, \rectstarty+1) rectangle (\xpos+0.4+5.5, \rectendy-0.7) {};
\node (N) at (\xpos+0.4+2+1.5,\rectstarty+0.5+1) {\small{Complex Transformation}};

\draw[->,thick,black] (\xpos+0.4+5.5, 2.7) -- (\xpos+0.4+6+1, 2.7);
\node (N) at (\xpos+0.4+6+1.7, 2.7) {\includegraphics[scale=0.3]{images/sample1.png}};
\node (N) at (\xpos+0.4+6+1.7,1.7) {\textcolor{red}{\small{Sample Generated}}};

\end{tikzpicture}		
		\end{center}

			\begin{itemize}[<+->]
			\item GANs take a different approach to this problem where the idea is to sample from a simple tractable distribution (say, $z \sim N(0, I)$) and then learn a complex transformation from this to the training distribution
			\item In other words, we will take a $z \sim N(0,I)$, learn to make a series of complex transformations on it so that the output looks as if it came from our training distribution 
			
			\end{itemize}		
		\end{overlayarea}			
\end{frame}
\begin{frame}
	\begin{columns}
		\column{0.45\textwidth}
		\begin{overlayarea}{\textwidth}{\textheight}
			\vspace*{10mm}
			\begin{center}
				\input{modules/Module1/tikz_images/gen_disc.tex}		
			\end{center}		
		\end{overlayarea}

		\column{0.55\textwidth}
		\begin{overlayarea}{\textwidth}{\textheight}
			\begin{itemize}[<+->]
				\item What can we use for such a complex transformation? \onslide<2-> \alert<2> {A Neural Network}
				\item<3-> How do you train such a neural network? \onslide<4-> \alert<4>{Using a two player game} 
				\item<5-> There are two players in the game: \onslide<6-> \alert<6>{a generator} \onslide<7-> and \alert<7>{a discriminator}
				\item<8-> The job of the generator is to produce images which look so natural that the discriminator thinks that the images came from the real data distribution
				\item<9-> The job of the discriminator is to get better and better at distinguishing between true images and generated (fake) images
			\end{itemize}
		\end{overlayarea}
	\end{columns}
\end{frame}


\begin{frame}
	\begin{columns}
		\column{0.45\textwidth}
		\begin{overlayarea}{\textwidth}{\textheight}
			\vspace*{10mm}
			\begin{center}
				\tikzstyle{hidden_neuron}=[circle,draw=blue!50,fill=cyan!10,thick,minimum size=4mm]
\begin{tikzpicture}
\newcommand*{\xpos}{10}
\newcommand*{\ypos}{10}
\newcommand*{\rectstarty}{6}
\newcommand*{\rectendy}{6.5}


%%%%%%%%%%%%%%Generator

% Generator
{
	\renewcommand*{\xpos}{20}
	\draw[rounded corners,fill=green!12] (\xpos-0.5, \rectstarty+1) rectangle (\xpos+2, \rectendy+1) {};
	\node (N) at (\xpos+0.75,\rectendy+0.75) {\small{Generator}};

	% Sample z
	\draw[rounded corners,fill=cyan!20] (\xpos-0.5, \rectstarty) rectangle (\xpos+2, \rectendy) {};
	\node[hidden_neuron] (N1) at (\xpos,\rectstarty+0.25) {};
	\node[hidden_neuron] (N2) at (\xpos+1*0.5,\rectstarty+0.25) {};
	\node[hidden_neuron] (N3) at (\xpos+2*0.5,\rectstarty+0.25) {};
	\node[hidden_neuron] (N4) at (\xpos+3*0.5,\rectstarty+0.25) {};
	\node (N) at (\xpos+1,\rectstarty-0.5) {\small{$z \sim N(0,I)$}};

	% Bottom arrow
	\draw[->,thick,black] (\xpos+0.75,\rectendy) -- (\xpos+0.75,\rectendy+0.5);

	% Images
	\node (N) at (\xpos+0.75,\rectstarty+2.2) {\includegraphics[scale=0.1]{images/two_gen.png}};

	% Top arrow
	\draw[->,thick,black] (\xpos+0.75,\rectendy+1) -- (\xpos+0.75,\rectendy+1.5);
}
%%%%%%%%%%%%%%%Real world images
{
	\renewcommand*{\xpos}{23}

	% Real Images
	\draw[rounded corners,fill=green!12] (\xpos-0.5, \rectstarty+1) rectangle (\xpos+2, \rectendy+1) {};
	\node (N) at (\xpos+0.75,\rectendy+0.75) {\small{Real Images}};

	% Images
	\node (N) at (\xpos+0.75,\rectstarty+2.2) {\includegraphics[scale=0.1]{images/two_real.png}};

	% Top arrow
	\draw[->,thick,black] (\xpos+0.75,\rectendy+1) -- (\xpos+0.75,\rectendy+1.5);

	\renewcommand*{\xpos}{23}
	\draw[->,thick,black] (\xpos+0.75,\rectendy+1.9) -- (\xpos+0.75-0.7,\rectstarty+3);
}
%%%%%%%%%%%%%%%% Discriminator
{
	\renewcommand*{\xpos}{21.5}
	% Discriminator
	\draw[rounded corners,fill=green!12] (\xpos-0.5, \rectstarty+3) rectangle (\xpos+2, \rectendy+3) {};
	\node (N) at (\xpos+0.75,\rectendy+2.75) {\small{Discriminator	}};
	{
		\draw[->,thick,black] (\xpos+0.75,\rectendy+3) -- (\xpos+0.75,\rectendy+3.4);
		\node (N) at (\xpos+0.75,\rectendy+3.52) {\small {Real or Fake}};
	}

	% Arrows
	\renewcommand*{\xpos}{20}
	\draw[->,thick,black] (\xpos+0.75,\rectendy+1.9) -- (\xpos+0.75+0.7,\rectstarty+3);
}
\end{tikzpicture}		
			\end{center}		
		\end{overlayarea}

		\column{0.55\textwidth}
		\begin{overlayarea}{\textwidth}{\textheight}
			\begin{itemize}[<+->]
				\item So let's look at the full picture
				\item Let $G_\phi$ be the generator and $D_\theta$ be the discriminator ($\phi$ and $\theta$ are the parameters of $G$ and $D$, respectively)
				\item We have a neural network based generator which takes as input a noise vector $z \sim N(0, I)$ and produces $G_\phi(z) = X$
				\item We have a neural network based discriminator which could take as input a real $X$ or a generated $X = G_\phi(z)$ and classify the input as real/fake
			\end{itemize}
		\end{overlayarea}
	\end{columns}
\end{frame}

\begin{frame}
	\begin{columns}
		\column{0.45\textwidth}
		\begin{overlayarea}{\textwidth}{\textheight}
			\vspace*{10mm}
			\begin{center}
				\tikzstyle{hidden_neuron}=[circle,draw=blue!50,fill=cyan!10,thick,minimum size=4mm]
\begin{tikzpicture}
\newcommand*{\xpos}{10}
\newcommand*{\ypos}{10}
\newcommand*{\rectstarty}{6}
\newcommand*{\rectendy}{6.5}


%%%%%%%%%%%%%%Generator

% Generator
{
	\renewcommand*{\xpos}{20}
	\draw[rounded corners,fill=green!12] (\xpos-0.5, \rectstarty+1) rectangle (\xpos+2, \rectendy+1) {};
	\node (N) at (\xpos+0.75,\rectendy+0.75) {\small{Generator}};

	% Sample z
	\draw[rounded corners,fill=cyan!20] (\xpos-0.5, \rectstarty) rectangle (\xpos+2, \rectendy) {};
	\node[hidden_neuron] (N1) at (\xpos,\rectstarty+0.25) {};
	\node[hidden_neuron] (N2) at (\xpos+1*0.5,\rectstarty+0.25) {};
	\node[hidden_neuron] (N3) at (\xpos+2*0.5,\rectstarty+0.25) {};
	\node[hidden_neuron] (N4) at (\xpos+3*0.5,\rectstarty+0.25) {};
	\node (N) at (\xpos+1,\rectstarty-0.5) {\small{$z \sim N(0,I)$}};

	% Bottom arrow
	\draw[->,thick,black] (\xpos+0.75,\rectendy) -- (\xpos+0.75,\rectendy+0.5);

	% Images
	\node (N) at (\xpos+0.75,\rectstarty+2.2) {\includegraphics[scale=0.1]{images/two_gen.png}};

	% Top arrow
	\draw[->,thick,black] (\xpos+0.75,\rectendy+1) -- (\xpos+0.75,\rectendy+1.5);
}
%%%%%%%%%%%%%%%Real world images
{
	\renewcommand*{\xpos}{23}

	% Real Images
	\draw[rounded corners,fill=green!12] (\xpos-0.5, \rectstarty+1) rectangle (\xpos+2, \rectendy+1) {};
	\node (N) at (\xpos+0.75,\rectendy+0.75) {\small{Real Images}};

	% Images
	\node (N) at (\xpos+0.75,\rectstarty+2.2) {\includegraphics[scale=0.1]{images/two_real.png}};

	% Top arrow
	\draw[->,thick,black] (\xpos+0.75,\rectendy+1) -- (\xpos+0.75,\rectendy+1.5);

	\renewcommand*{\xpos}{23}
	\draw[->,thick,black] (\xpos+0.75,\rectendy+1.9) -- (\xpos+0.75-0.7,\rectstarty+3);
}
%%%%%%%%%%%%%%%% Discriminator
{
	\renewcommand*{\xpos}{21.5}
	% Discriminator
	\draw[rounded corners,fill=green!12] (\xpos-0.5, \rectstarty+3) rectangle (\xpos+2, \rectendy+3) {};
	\node (N) at (\xpos+0.75,\rectendy+2.75) {\small{Discriminator	}};
	{
		\draw[->,thick,black] (\xpos+0.75,\rectendy+3) -- (\xpos+0.75,\rectendy+3.4);
		\node (N) at (\xpos+0.75,\rectendy+3.52) {\small {Real or Fake}};
	}

	% Arrows
	\renewcommand*{\xpos}{20}
	\draw[->,thick,black] (\xpos+0.75,\rectendy+1.9) -- (\xpos+0.75+0.7,\rectstarty+3);
}
\end{tikzpicture}		
			\end{center}		
		\end{overlayarea}

		\column{0.55\textwidth}
		\begin{overlayarea}{\textwidth}{\textheight}
			\begin{itemize}[<+->]
				\item What should be the objective function of the overall network?
				\item Let's look at the objective function of the generator first 
				\item Given an image generated by the generator as $G_\phi(z)$ the discriminator assigns a score $D_\theta(G_\phi(z))$ to it 
				\item This score will be between $0$ and $1$ and will tell us the probability of the image being real or fake 
				\item For a given $z$, the generator would want to maximize  $\log D_\theta(G_\phi(z))$ (log likelihood) or minimize $\log (1 - D_\theta(G_\phi(z)))$
			\end{itemize}
			\vspace{10mm}
			
		\end{overlayarea}
	\end{columns}
\end{frame}

\begin{frame}
	\begin{columns}
		\column{0.45\textwidth}
		\begin{overlayarea}{\textwidth}{\textheight}
			\vspace*{10mm}
			\begin{center}
				\tikzstyle{hidden_neuron}=[circle,draw=blue!50,fill=cyan!10,thick,minimum size=4mm]
\begin{tikzpicture}
\newcommand*{\xpos}{10}
\newcommand*{\ypos}{10}
\newcommand*{\rectstarty}{6}
\newcommand*{\rectendy}{6.5}


%%%%%%%%%%%%%%Generator

% Generator
{
	\renewcommand*{\xpos}{20}
	\draw[rounded corners,fill=green!12] (\xpos-0.5, \rectstarty+1) rectangle (\xpos+2, \rectendy+1) {};
	\node (N) at (\xpos+0.75,\rectendy+0.75) {\small{Generator}};

	% Sample z
	\draw[rounded corners,fill=cyan!20] (\xpos-0.5, \rectstarty) rectangle (\xpos+2, \rectendy) {};
	\node[hidden_neuron] (N1) at (\xpos,\rectstarty+0.25) {};
	\node[hidden_neuron] (N2) at (\xpos+1*0.5,\rectstarty+0.25) {};
	\node[hidden_neuron] (N3) at (\xpos+2*0.5,\rectstarty+0.25) {};
	\node[hidden_neuron] (N4) at (\xpos+3*0.5,\rectstarty+0.25) {};
	\node (N) at (\xpos+1,\rectstarty-0.5) {\small{$z \sim N(0,I)$}};

	% Bottom arrow
	\draw[->,thick,black] (\xpos+0.75,\rectendy) -- (\xpos+0.75,\rectendy+0.5);

	% Images
	\node (N) at (\xpos+0.75,\rectstarty+2.2) {\includegraphics[scale=0.1]{images/two_gen.png}};

	% Top arrow
	\draw[->,thick,black] (\xpos+0.75,\rectendy+1) -- (\xpos+0.75,\rectendy+1.5);
}
%%%%%%%%%%%%%%%Real world images
{
	\renewcommand*{\xpos}{23}

	% Real Images
	\draw[rounded corners,fill=green!12] (\xpos-0.5, \rectstarty+1) rectangle (\xpos+2, \rectendy+1) {};
	\node (N) at (\xpos+0.75,\rectendy+0.75) {\small{Real Images}};

	% Images
	\node (N) at (\xpos+0.75,\rectstarty+2.2) {\includegraphics[scale=0.1]{images/two_real.png}};

	% Top arrow
	\draw[->,thick,black] (\xpos+0.75,\rectendy+1) -- (\xpos+0.75,\rectendy+1.5);

	\renewcommand*{\xpos}{23}
	\draw[->,thick,black] (\xpos+0.75,\rectendy+1.9) -- (\xpos+0.75-0.7,\rectstarty+3);
}
%%%%%%%%%%%%%%%% Discriminator
{
	\renewcommand*{\xpos}{21.5}
	% Discriminator
	\draw[rounded corners,fill=green!12] (\xpos-0.5, \rectstarty+3) rectangle (\xpos+2, \rectendy+3) {};
	\node (N) at (\xpos+0.75,\rectendy+2.75) {\small{Discriminator	}};
	{
		\draw[->,thick,black] (\xpos+0.75,\rectendy+3) -- (\xpos+0.75,\rectendy+3.4);
		\node (N) at (\xpos+0.75,\rectendy+3.52) {\small {Real or Fake}};
	}

	% Arrows
	\renewcommand*{\xpos}{20}
	\draw[->,thick,black] (\xpos+0.75,\rectendy+1.9) -- (\xpos+0.75+0.7,\rectstarty+3);
}
\end{tikzpicture}		
			\end{center}		
		\end{overlayarea}

		\column{0.55\textwidth}
		\begin{overlayarea}{\textwidth}{\textheight}
			\begin{itemize}[<+->]
				\item This is just for a single $z$ and the generator would like to do this for all possible values of $z$,
				\item For example, if $z$ was discrete and drawn from a uniform distribution (\textit{i.e.}, $p(z) = \frac{1}{N} ~ \forall z$) then the generator's objective function would be 

				$$ \min\limits_{\phi}  \sum_{i=1}^{N} \frac{1}{N} \log (1 - D_{\theta}(G_{\phi}(z))) $$

				\item However, in our case, z is continuous and not uniform ($z \sim N(0, I)$) so the equivalent objective function would be 

				$$\min \limits_{\phi} \int p(z) \log(1 - D_{\theta}(G_{\phi}(z)))$$
				$$\min \limits_{\phi}  E_{_{z \sim p(z)}} [\log (1 - D_{\theta}(G_{\phi}(z)))] $$

			\end{itemize}
		\end{overlayarea}
	\end{columns}
\end{frame}

\begin{frame}
	\begin{columns}
		\column{0.45\textwidth}
		\begin{overlayarea}{\textwidth}{\textheight}
			\vspace*{10mm}
			\begin{center}
				\tikzstyle{hidden_neuron}=[circle,draw=blue!50,fill=cyan!10,thick,minimum size=4mm]
\begin{tikzpicture}
\newcommand*{\xpos}{10}
\newcommand*{\ypos}{10}
\newcommand*{\rectstarty}{6}
\newcommand*{\rectendy}{6.5}


%%%%%%%%%%%%%%Generator

% Generator
{
	\renewcommand*{\xpos}{20}
	\draw[rounded corners,fill=green!12] (\xpos-0.5, \rectstarty+1) rectangle (\xpos+2, \rectendy+1) {};
	\node (N) at (\xpos+0.75,\rectendy+0.75) {\small{Generator}};

	% Sample z
	\draw[rounded corners,fill=cyan!20] (\xpos-0.5, \rectstarty) rectangle (\xpos+2, \rectendy) {};
	\node[hidden_neuron] (N1) at (\xpos,\rectstarty+0.25) {};
	\node[hidden_neuron] (N2) at (\xpos+1*0.5,\rectstarty+0.25) {};
	\node[hidden_neuron] (N3) at (\xpos+2*0.5,\rectstarty+0.25) {};
	\node[hidden_neuron] (N4) at (\xpos+3*0.5,\rectstarty+0.25) {};
	\node (N) at (\xpos+1,\rectstarty-0.5) {\small{$z \sim N(0,I)$}};

	% Bottom arrow
	\draw[->,thick,black] (\xpos+0.75,\rectendy) -- (\xpos+0.75,\rectendy+0.5);

	% Images
	\node (N) at (\xpos+0.75,\rectstarty+2.2) {\includegraphics[scale=0.1]{images/two_gen.png}};

	% Top arrow
	\draw[->,thick,black] (\xpos+0.75,\rectendy+1) -- (\xpos+0.75,\rectendy+1.5);
}
%%%%%%%%%%%%%%%Real world images
{
	\renewcommand*{\xpos}{23}

	% Real Images
	\draw[rounded corners,fill=green!12] (\xpos-0.5, \rectstarty+1) rectangle (\xpos+2, \rectendy+1) {};
	\node (N) at (\xpos+0.75,\rectendy+0.75) {\small{Real Images}};

	% Images
	\node (N) at (\xpos+0.75,\rectstarty+2.2) {\includegraphics[scale=0.1]{images/two_real.png}};

	% Top arrow
	\draw[->,thick,black] (\xpos+0.75,\rectendy+1) -- (\xpos+0.75,\rectendy+1.5);

	\renewcommand*{\xpos}{23}
	\draw[->,thick,black] (\xpos+0.75,\rectendy+1.9) -- (\xpos+0.75-0.7,\rectstarty+3);
}
%%%%%%%%%%%%%%%% Discriminator
{
	\renewcommand*{\xpos}{21.5}
	% Discriminator
	\draw[rounded corners,fill=green!12] (\xpos-0.5, \rectstarty+3) rectangle (\xpos+2, \rectendy+3) {};
	\node (N) at (\xpos+0.75,\rectendy+2.75) {\small{Discriminator	}};
	{
		\draw[->,thick,black] (\xpos+0.75,\rectendy+3) -- (\xpos+0.75,\rectendy+3.4);
		\node (N) at (\xpos+0.75,\rectendy+3.52) {\small {Real or Fake}};
	}

	% Arrows
	\renewcommand*{\xpos}{20}
	\draw[->,thick,black] (\xpos+0.75,\rectendy+1.9) -- (\xpos+0.75+0.7,\rectstarty+3);
}
\end{tikzpicture}		
			\end{center}		
		\end{overlayarea}

		\column{0.55\textwidth}
		\begin{overlayarea}{\textwidth}{\textheight}
			\begin{itemize}[<+->]
				\item Now let's look at the discriminator
				\item The task of the discriminator is to assign a high score to real images and a low score to fake images
				\item And it should do this for all possible real images and all possible fake images
				\item In other words, it should try to maximize the following objective function
				$$ \max_{\theta} E_{_{x \sim p_{data}}} [\log D_{\theta}(x)] + E_{z \sim p(z)} [\log (1 - D_{\theta}(G_{\phi}(z)))]$$
			\end{itemize}
		\end{overlayarea}
	\end{columns}
\end{frame}

\begin{frame}
	\begin{columns}
		\column{0.4\textwidth}
		\begin{overlayarea}{\textwidth}{\textheight}
			\vspace*{10mm}
			\begin{center}
				\tikzstyle{hidden_neuron}=[circle,draw=blue!50,fill=cyan!10,thick,minimum size=4mm]
\begin{tikzpicture}
\newcommand*{\xpos}{10}
\newcommand*{\ypos}{10}
\newcommand*{\rectstarty}{6}
\newcommand*{\rectendy}{6.5}


%%%%%%%%%%%%%%Generator

% Generator
{
	\renewcommand*{\xpos}{20}
	\draw[rounded corners,fill=green!12] (\xpos-0.5, \rectstarty+1) rectangle (\xpos+2, \rectendy+1) {};
	\node (N) at (\xpos+0.75,\rectendy+0.75) {\small{Generator}};

	% Sample z
	\draw[rounded corners,fill=cyan!20] (\xpos-0.5, \rectstarty) rectangle (\xpos+2, \rectendy) {};
	\node[hidden_neuron] (N1) at (\xpos,\rectstarty+0.25) {};
	\node[hidden_neuron] (N2) at (\xpos+1*0.5,\rectstarty+0.25) {};
	\node[hidden_neuron] (N3) at (\xpos+2*0.5,\rectstarty+0.25) {};
	\node[hidden_neuron] (N4) at (\xpos+3*0.5,\rectstarty+0.25) {};
	\node (N) at (\xpos+1,\rectstarty-0.5) {\small{$z \sim N(0,I)$}};

	% Bottom arrow
	\draw[->,thick,black] (\xpos+0.75,\rectendy) -- (\xpos+0.75,\rectendy+0.5);

	% Images
	\node (N) at (\xpos+0.75,\rectstarty+2.2) {\includegraphics[scale=0.1]{images/two_gen.png}};

	% Top arrow
	\draw[->,thick,black] (\xpos+0.75,\rectendy+1) -- (\xpos+0.75,\rectendy+1.5);
}
%%%%%%%%%%%%%%%Real world images
{
	\renewcommand*{\xpos}{23}

	% Real Images
	\draw[rounded corners,fill=green!12] (\xpos-0.5, \rectstarty+1) rectangle (\xpos+2, \rectendy+1) {};
	\node (N) at (\xpos+0.75,\rectendy+0.75) {\small{Real Images}};

	% Images
	\node (N) at (\xpos+0.75,\rectstarty+2.2) {\includegraphics[scale=0.1]{images/two_real.png}};

	% Top arrow
	\draw[->,thick,black] (\xpos+0.75,\rectendy+1) -- (\xpos+0.75,\rectendy+1.5);

	\renewcommand*{\xpos}{23}
	\draw[->,thick,black] (\xpos+0.75,\rectendy+1.9) -- (\xpos+0.75-0.7,\rectstarty+3);
}
%%%%%%%%%%%%%%%% Discriminator
{
	\renewcommand*{\xpos}{21.5}
	% Discriminator
	\draw[rounded corners,fill=green!12] (\xpos-0.5, \rectstarty+3) rectangle (\xpos+2, \rectendy+3) {};
	\node (N) at (\xpos+0.75,\rectendy+2.75) {\small{Discriminator	}};
	{
		\draw[->,thick,black] (\xpos+0.75,\rectendy+3) -- (\xpos+0.75,\rectendy+3.4);
		\node (N) at (\xpos+0.75,\rectendy+3.52) {\small {Real or Fake}};
	}

	% Arrows
	\renewcommand*{\xpos}{20}
	\draw[->,thick,black] (\xpos+0.75,\rectendy+1.9) -- (\xpos+0.75+0.7,\rectstarty+3);
}
\end{tikzpicture}		
			\end{center}		
		\end{overlayarea}

		\column{0.6\textwidth}
		\begin{overlayarea}{\textwidth}{\textheight}
			\begin{itemize}[<+->]
				\item<1-> If we put the objectives of the generator and discriminator together we get a minimax game
				\begin{align*}
				 \min\limits_{\phi}\hspace*{2mm} \max\limits_{\theta} \hspace*{4mm} [&\mathbb{E}_{x \sim p_{data}} \log D_{\theta}(x) \\
				 & ~~~+ \mathbb{E}_{z \sim p(z)} \log(1- D_{\theta}(G_{\phi}(z)) ) ] 
				\end{align*}
				\item<2-> The first term in the objective is only w.r.t. the parameters of the discriminator ($\theta$)

				\item<3-> The second term in the objective is w.r.t. the parameters of the generator ($\phi$) as well as the discriminator ($\theta$)
				\item<4-> The discriminator wants to maximize the second term whereas the generator wants to minimize it (hence it is a two-player game)
			\end{itemize}
		\end{overlayarea}
	\end{columns}
\end{frame}

\begin{frame}
	\begin{columns}
		\column{0.45\textwidth}
		\begin{overlayarea}{\textwidth}{\textheight}
			\vspace*{10mm}
			\begin{center}
				\tikzstyle{hidden_neuron}=[circle,draw=blue!50,fill=cyan!10,thick,minimum size=4mm]
\begin{tikzpicture}
\newcommand*{\xpos}{10}
\newcommand*{\ypos}{10}
\newcommand*{\rectstarty}{6}
\newcommand*{\rectendy}{6.5}


%%%%%%%%%%%%%%Generator

% Generator
{
	\renewcommand*{\xpos}{20}
	\draw[rounded corners,fill=green!12] (\xpos-0.5, \rectstarty+1) rectangle (\xpos+2, \rectendy+1) {};
	\node (N) at (\xpos+0.75,\rectendy+0.75) {\small{Generator}};

	% Sample z
	\draw[rounded corners,fill=cyan!20] (\xpos-0.5, \rectstarty) rectangle (\xpos+2, \rectendy) {};
	\node[hidden_neuron] (N1) at (\xpos,\rectstarty+0.25) {};
	\node[hidden_neuron] (N2) at (\xpos+1*0.5,\rectstarty+0.25) {};
	\node[hidden_neuron] (N3) at (\xpos+2*0.5,\rectstarty+0.25) {};
	\node[hidden_neuron] (N4) at (\xpos+3*0.5,\rectstarty+0.25) {};
	\node (N) at (\xpos+1,\rectstarty-0.5) {\small{$z \sim N(0,I)$}};

	% Bottom arrow
	\draw[->,thick,black] (\xpos+0.75,\rectendy) -- (\xpos+0.75,\rectendy+0.5);

	% Images
	\node (N) at (\xpos+0.75,\rectstarty+2.2) {\includegraphics[scale=0.1]{images/two_gen.png}};

	% Top arrow
	\draw[->,thick,black] (\xpos+0.75,\rectendy+1) -- (\xpos+0.75,\rectendy+1.5);
}
%%%%%%%%%%%%%%%Real world images
{
	\renewcommand*{\xpos}{23}

	% Real Images
	\draw[rounded corners,fill=green!12] (\xpos-0.5, \rectstarty+1) rectangle (\xpos+2, \rectendy+1) {};
	\node (N) at (\xpos+0.75,\rectendy+0.75) {\small{Real Images}};

	% Images
	\node (N) at (\xpos+0.75,\rectstarty+2.2) {\includegraphics[scale=0.1]{images/two_real.png}};

	% Top arrow
	\draw[->,thick,black] (\xpos+0.75,\rectendy+1) -- (\xpos+0.75,\rectendy+1.5);

	\renewcommand*{\xpos}{23}
	\draw[->,thick,black] (\xpos+0.75,\rectendy+1.9) -- (\xpos+0.75-0.7,\rectstarty+3);
}
%%%%%%%%%%%%%%%% Discriminator
{
	\renewcommand*{\xpos}{21.5}
	% Discriminator
	\draw[rounded corners,fill=green!12] (\xpos-0.5, \rectstarty+3) rectangle (\xpos+2, \rectendy+3) {};
	\node (N) at (\xpos+0.75,\rectendy+2.75) {\small{Discriminator	}};
	{
		\draw[->,thick,black] (\xpos+0.75,\rectendy+3) -- (\xpos+0.75,\rectendy+3.4);
		\node (N) at (\xpos+0.75,\rectendy+3.52) {\small {Real or Fake}};
	}

	% Arrows
	\renewcommand*{\xpos}{20}
	\draw[->,thick,black] (\xpos+0.75,\rectendy+1.9) -- (\xpos+0.75+0.7,\rectstarty+3);
}
\end{tikzpicture}		
			\end{center}		
		\end{overlayarea}

		\column{0.55\textwidth}
		\begin{overlayarea}{\textwidth}{\textheight}
		
			\begin{itemize}[<+->]
			\item So the overall training proceeds by alternating between these two step
			
			\item \textbf{Step 1:} Gradient Ascent on Discriminator 
				$$ \max\limits_{\theta}\hspace*{2mm} [\mathbb{E}_{x \sim p_{data}} \log D_{\theta}(x)
                \newline
                + \mathbb{E}_{z \sim p(z)} \log(1- D_{\theta}(G_{\phi}(z)) ) ]$$

			\item \textbf{Step 2:} Gradient Descent on Generator
				$$   \min\limits_{\phi}\hspace*{2mm} \mathbb{E}_{z \sim p(z)} \log(1- D_{\theta}(G_{\phi}(z)) )$$
			
			\item In practice, the above generator objective does not work well and we use a slightly modified objective
			\item Let us see why
			\end{itemize}
		\end{overlayarea}
	\end{columns}
\end{frame}

% \begin{frame}
% 	\begin{columns}
% 		\column{0.45\textwidth}
% 		\begin{overlayarea}{\textwidth}{\textheight}
% 			\vspace*{10mm}
% 			\begin{center}
% 				{\scalebox{0.45}{%
% 				\usepackage{pgfplots}
% \pgfplotsset{width=10cm,compat=1.9}

\begin{tikzpicture}
	\begin{axis}[ymin=-4,ymax=4,xlabel=$D(G(z))$,ylabel=$Loss$]
		\addplot[domain=0:1, samples=100,unbounded coords=jump,color=red]{ln(1-x)/ln(2)};
		% \addlegendentry{$\log(1-D(g(x)))$}
		\only<3->
		{
		    \addplot[domain=0:1, samples=100,unbounded coords=jump,color=blue]{-ln(x)/ln(2)};
		    \addlegendentry{$-\log(D(g(x)))$}
	    }
	\end{axis}
\end{tikzpicture}}}
% 				% \only<2>{\scalebox{0.7}{%
% 				% \usepackage{pgfplots}
% \pgfplotsset{width=10cm,compat=1.9}

\begin{tikzpicture}
	\begin{axis}[ymin=-4,ymax=4,xlabel=$D(G(z))$,ylabel=$Loss$]
		\addplot[domain=0:1, samples=100,unbounded coords=jump,color=red]{ln(1-x)/ln(2)};
		% \addlegendentry{$\log(1-D(g(x)))$}
		\only<3->
		{
		    \addplot[domain=0:1, samples=100,unbounded coords=jump,color=blue]{-ln(x)/ln(2)};
		    \addlegendentry{$-\log(D(g(x)))$}
	    }
	\end{axis}
\end{tikzpicture}}}
% 				% \only<3->{\scalebox{0.7}{%
% 				% \input{modules/Module1/tikz_images/gradient2.tex}}}
% 			% LHS: <Show figure from Slide 111>
% 			\end{center}

% 		\end{overlayarea}
% 		\column{0.55\textwidth}
% 		\begin{overlayarea}{\textwidth}{\textheight}
% 			\begin{itemize}[<+->]
% 				\item When the sample is likely fake, we want to give a feedback to the generator (using gradients)
% 				\item However, in this region where $D(G(z))$ is close to 0, the curve of the loss function is very flat and the gradient would be close to $0$
% 				\item Trick: Instead of minimizing the likelihood of the discriminator being correct, maximize the likelihood of the discriminator being wrong 
% 				\item In effect, the objective remains the same but the gradient signal becomes better
% 			\end{itemize}
% 		\end{overlayarea}
% 	\end{columns}
% \end{frame}
