% !TeX program = xelatex
\documentclass[a4paper,12pt]{article}
\usepackage{tikz}
\usepackage[bottom]{footmisc}
\usepackage{subcaption}
\usepackage{listings}
\usepackage{xcolor}
\usepackage[colorlinks = true,
linkcolor = blue,
urlcolor  = blue,
citecolor = blue,
anchorcolor = blue]{hyperref}
\usepackage{fullpage,amsmath,hyperref,color,clrscode,amsfonts,graphicx,textcomp}
\usepackage{tabularx}
\usepackage{adjustbox}
\usepackage{setspace}
\usepackage{upgreek}
\usepackage{amsmath}
\usepackage{amssymb}
\usepackage{tikz}
\usepackage[inline]{enumitem}
\usepackage{pgfplots}
\usetikzlibrary{calc}
\newcommand{\drawaline}[4]{
	\draw [extended line=1cm,stealth-stealth] (#1,#2)--(#3,#4);
}
\usepackage{float}
\usetikzlibrary {positioning}
%\usepackage {xcolor}
%\definecolor {processblue}{cmyk}{0.96,0,0,0}
\definecolor {processblue}{cmyk}{1,1,1,0}
\usetikzlibrary{automata,arrows.meta}
\usepackage{hyperref}
\usepackage{multirow}
\usepackage{graphicx}
\usepackage{neuralnetwork}
\usepackage{tkz-graph}
\hypersetup{
	colorlinks=true,
	linkcolor=red,
	filecolor=blue,      
	urlcolor=magenta,
}


\usetikzlibrary{arrows, positioning}
\usepackage{xepersian}

\definecolor{codegreen}{rgb}{0,0.6,0}
\definecolor{codegray}{rgb}{0.5,0.5,0.5}
\definecolor{codepurple}{rgb}{0.58,0,0.82}
\definecolor{backcolour}{rgb}{0.95,0.95,0.95}

\lstdefinestyle{mystyle}{
	backgroundcolor=\color{backcolour},   
	commentstyle=\color{codegreen},
	keywordstyle=\color{magenta},
	numberstyle=\tiny\color{codegray},
	stringstyle=\color{codepurple},
	basicstyle=\ttfamily\footnotesize,
	breakatwhitespace=false,         
	breaklines=true,                 
	captionpos=b,                    
	keepspaces=true,                 
	numbers=left,                    
	numbersep=5pt,                  
	showspaces=false,                
	showstringspaces=false,
	showtabs=false,                  
	tabsize=2
}

\lstset{style=mystyle}

\settextfont[Path="./fonts/",Scale=0.99 , BoldFont = XB Zar bold.ttf]{XB Zar.ttf}
\setlatintextfont[Path="./fonts/"]{Times New Roman.ttf}

\addtolength{\textwidth}{0.2in}
\addtolength{\oddsidemargin}{-0.1in}
\addtolength{\evensidemargin}{-0.1in}

\addtolength{\textheight}{0.5in}
\addtolength{\topmargin}{-0.25in}
\defpersianfont\iranic[Path="./fonts/",Scale=1]{XB Zar.ttf} 	
\newcommand{\IP}{\mathbb{P}}

\makeatletter
\def\LTRfootnote{\@ifnextchar[\@xLTRfootnote{\stepcounter\@mpfn
		\protected@xdef\@thefnmark{\persianfont\thempfn}%
		\@footnotemark\@LTRfootnotetext}}
\makeatother	
\setdigitfont[Path="./fonts/",Scale=1]{XB Zar.ttf} 
\setmathdigitfont[Path="./fonts/",Scale=1]{XB Zar.ttf} 
\usepackage{assets/template}
\newcommand{\hidesolutions}{}

\begin{document}
{\centering به نام زیباترین}\\
	\def\ci{\perp\!\!\!\perp}

	% header{<Assignment-Number>}{<Assignment-Title>}{<Deadline-Date>}{<Gathered-by>}{<Supervised-by>}
	\header
		{6 دی ۱۴۰۳}
		{آزمون میان‌ترم دوم}
		{زمان:  3  ساعت}
		{امیرمهدی میقانی - علی رحیمی اکبر - پیام تائبی}{}
	\begin{itemize}
	\small
	\setlength\itemsep{0.05em}

\end{itemize}
\vspace{-3mm}
نام و نام خانوادگی: \hspace{8cm} شماره دانشجویی:

\hrule
\vspace{2mm}
لطفا پاسخ هر سوال را زیر آن بنویسید. بالای \textbf{هر برگه نام و شماره دانشجویی} خود را حتما قید کنید. در پایان آزمون، برگه‌ی سوال‌های مختلف را از هم جدا کنید و هر برگه را در دسته‌ی مربوط به خود قرار دهید.
این آزمون از 110 نمره است و نمره کامل 100 است. توجه کنید نمره‌ی بالای 100 امتیازی نخواهد داشت.



\vspace{0.5cm}
\textbf{اصلی}

در یک شرکت بزرگ کاریابی کار می‌کنید. هر فرد یک پروفایل دارد که بعضی ویژگی‌های آن (نظیر سن، آخرین حقوق) عدد پیوسته و بعضی ویژگی‌های دیگر (نظیر رشته‌ی تحصیلی) Categorical است. 
همچنین بعضی از ویژگی‌ها (نظیر آشنایی با هریک از زبان‌های برنامه‌نویسی) به صورت صفر و یک درج شده است. 

\begin{enumerate}
\item می‌خواهید برای هر کاربر جدید که پروفایل خود را تکمیل کرده‌است، یک مبلغ حقوق تخمین بزنید. از چه الگوریتمی استفاده می‌کنید؟ چه تغییری روی ویژگی‌ها می‌دهید؟ برای هریک از ویژگی‌ها از چه پیش‌پردازشی استفاده می‌کنید؟ جزئیات را توضیح دهید.
\vspace{5cm}
\item
حال می‌خواهید داده‌های این شرکت را به صورت یک نمودار نمایش دهید. برای این کار تصمیم دارید از تحلیل مولفه‌های اصلی (PCA) استفاده کنید.
آیا به نظر شما تغییر مقیاس ویژگی‌ها لازم است؟ اگر بله، چرا؟ و چطور این کار را انجام می‌دهید؟ اگر خیر، دلیل‌ شما چیست؟
\vspace{5cm}
\item
فرض کنید ماتریس کوواریانس زیر را داشته باشید. چطور از روی آن مولفه‌های اصلی را مشخص می‌کنید؟ محاسبات خود را بنویسید.
{\latin
\centering
$C=$\begin{bmatrix}
3 & -4\\
-4 & 3
\end{bmatrix}
}
\vspace{5cm}
\item
فرض کنید می‌خواهید با PCA ابعاد داده‌هاهای شرکت را با بردن آن‌ها فضای جدیدی کاهش دهید که عمده‌ی اطلاعات حفظ شود. چطور تعداد بعدهای فضای جدید را مشخص می‌کنید؟
\vspace{6cm}
\item
ثابت کنید تصویرکردن داده‌ها توسط بردارویژه‌‌ی با بزرگ‌ترین مقدار ویژه‌ی ماتریس کوواریانس، واریانس داده‌های تصویرشده را بیشینه می‌کند.
\vspace{6cm}
\item
به دل‌خواه یک مساله‌ی زیبای یادگیری ماشین بنویسید و آن را حل کنید. نمره‌ی این بخش به زیبایی و منحصر به فرد بودن مساله‌ و درستی راه‌حل شما اختصاص می‌یابد.

\end{enumerate}

\pagebreak
\vspace{1cm}

پاسخ سوال اول) گزینه‌های الف و ب

\vspace{1cm}

پاسخ سوال دوم) گزینه‌های الف و د




\vspace{1cm}
پاسخ سوال سوم)



الف نادرست و ب درست می‌باشد


مدل‌های خودرمز‌گذار پراکنده، بدون نیاز به regularization خاصی شامل گلوگاه اطلاعاتی می‌باشند. گزاره ب نیز تعریف این دسته از مدل‌ها می‌باشد.

\vspace{1cm}


ج درست و د درست می‌باشد




در گزاره اول، با کاهش تعداد نورون‌ها می‌توان میزان اطلاعات در مدل خودرمز‌گذار را کاهش و درنتیجه آن را به مدل خودرمزگذار ناقص تبدیل کرد. در گزاره دوم، به صورت‌کلی مدل‌های خودرمزگذار را میتوان به‌عنوان یک ابزار کاهش ابعاد داده ( reduction dimensionality ) درنظر گرفت که برای این امر، مدل ابرصفحه‌هایی را برای تصویر کردن داده به ابعاد پایین‌تر را یاد می‌گیرد.

\vspace{1cm}

ه درست  و درست می‌باشد




این مدل برای از بین بردن نیز در داده ورودی استفاده می‌شود، درنتیجه تابع هزینه زمان آموزش این مدل سعی می‌کند تفاوت میان نسخه اصلی داده ورودی و بازسازی داده براساس نسخه نویزی آن را کاهش دهد. خروجی بخش رمز‌گذار مدل (و ورودی بخش رمز‌گشا) ویژگی‌های استخراج‌شده از داده توسط مدل می‌باشند.
\pagebreak
\hrule
% % نام و نام خانوادگی: \hspace{8cm} شماره دانشجویی:
% % \hrule
% % \vspace{0.5cm}
% \textbf{سوال چهارگزینه ای(9 نمره)}



در هریک از سوال‌های زیر، با انتخاب گزینه مناسب گزاره‌های درست را انتخاب کنید.

\vspace{0.5cm}
۱.

الف) در مدل‌های خودرمز‌گذار پراکنده \footnote{autoncoder Sparse} ایجاد تعداد نورون‌ها در لایه نهان مدل \footnote{layer Hidden} باعث ایجاد گلوگاه اطلاعاتی \footnote{bottleneck Information} می‌شود.

\vspace{0.25cm}
ب) ایده کلی در مدل خودرمز‌گذار پراکنده این است که روند رمز‌گذاری و رمزگشایی تنها با فعال نگه‌داشتن تعداد کمی از نورون‌ها صورت بگیرد

\vspace{0.5cm}
۱)  الف درست و ب نادرست است.

\vspace{0.25cm}
۲) هردو نادرست هستند.

\vspace{0.25cm}
۳) هردو درست هستند.

\vspace{0.25cm}
۴)  الف نادرست و ب درست است.

\vspace{0.5cm}
۲.

الف) یک راه برای ایجاد مدل خودرمز‌گذار ناقص \footnote{autoencoder Sparse} این است که تعداد نورون‌های موجود در لایه‌های نهان \footnote{layers Hidden} را کاهش دهیم.

\vspace{0.25cm}
ب) مدل‌های خودرمزگذار قادر به یادگیری منیفولد‌های \footnote{Manifold} غیرخطی می‌باشند.

\vspace{0.5cm}
۱) الف درست و ب نادرست است.

\vspace{0.25cm}
۲) هردو نادرست هستند.

\vspace{0.25cm}
۳) هردو درست هستند.

\vspace{0.25cm}
۴) الف نادرست و ب درست است.

\vspace{0.5cm}
۳.

الف) در مدل خودرمزگذار بدون نویز \footnote{autoencoder Denoising}، تابع هزینه میان نسخه اصلی ورودی و بازسازی آن براساس نسخه نویزی آن، محاسبه می‌شود.

\vspace{0.25cm}
ب) از مدل خودرمزگذار بدون نویز میتوان به‌عنوان ابزاری برای استخراج ویژگی استفاده کرد.

\vspace{0.5cm}
۱) الف درست و ب نادرست است.

\vspace{0.25cm}
۲) هردو نادرست هستند.

\vspace{0.25cm}
۳) هردو درست هستند.

\vspace{0.25cm}
۴) الف نادرست و ب درست است.

% \input{solutions/q5_sol}
% \pagebreak
% \hrule
نام و نام خانوادگی: \hspace{8cm} شماره دانشجویی:
\hrule
\vspace{0.5cm}


ماتریس داده زیر را در نظر  بگیرید که چهار نمونه $X_i \in R^2$ را نشان می‌دهد:

$$X= \begin{pmatrix}
 4 & 1  \\
 2& 3 \\
 5& 4 \\
 1& 0  \\
\end{pmatrix}$$

\begin{enumerate}
    \item [الف)] جهت‌های مولفه اصلی با طول واحد $X$ را محاسبه کنید و بیان کنید که اگر فقط به دنبال یک مولفه اصلی باشیم،  الگوریتم $PCA$ کدام یک را انتخاب می‌کند. 
    \item[ب)] بهترین (با حداقل خطای بازسازی) projection از $X$ را در یک زیرفضای یک بعدی با مبدا صفر پیدا کنید.
\end{enumerate}


\pagebreak
\textbf{پاسخ}
(الف) $Ausgmentation$ های مختلف یکی از جواب ها است.

(ب)
\begin{equation}
z_1 = \begin{bmatrix}
0.2&0.4&-0.5 \\
-0.3&0.1&0.2
\end{bmatrix} 
\begin{bmatrix}
1 \\
0 \\
1
\end{bmatrix} + \begin{bmatrix}
-0.4\\
0.2
\end{bmatrix} = \begin{bmatrix}
-0.7\\
0.1
\end{bmatrix}
\end{equation}

\begin{equation}
a_1=\begin{bmatrix}
0.\\
0.1
\end{bmatrix}
\end{equation}

\begin{equation}
z_2 = \begin{bmatrix}
-0.3&0.2 \\
\end{bmatrix} 
\begin{bmatrix}
0 \\
0.1
\end{bmatrix} + \begin{bmatrix}
0.1
\end{bmatrix} = 0.08
\end{equation}

\begin{equation}
    \hat{y}(i) = \frac{1}{1+\epsilon^{-0.08}}\approx 0.52
\end{equation}
\begin{equation}
    L(i) = 0.5(log0.52) \approx 0.142
\end{equation}

(ج) وزن دهی به اینکه هرکلاس چقدر در تابع ضرر مشارکت دارد می تواند به نزول گرادیان کمک کند، زیرا شبکه هنگام یادگیری از نمونه های کلاس کم تعدادتر، قدم های بزرگتری خواهد داشت.
$\alpha = 0.1$ و $\beta = 1$ بطور تقریبی، نسبت باید جایی در حدود $\beta =10 \cdot a $ باشد، اما نه خیلی بزرگ یا کوچک باشد.

(د) 

\begin{equation}
    \frac{\partial j}{\partial \hat{y}} = -\frac{1}{m} \sum_{i} \delta_1^{(i)}    
\end{equation}
که در آن :
\begin{equation}
    \delta_1^{(i)} = a \cdot y^{(i)}/ \hat{y}^{(i)} - \beta \cdot (1-y^{(i)})/(1-\hat{y}^{(i)})
\end{equation}

\begin{equation}
    \frac{\partial \hat{y}^{(i)}}{\partial z_2} = \delta_2 = \sigma(z_2)(1-\sigma(z_2))
\end{equation}

\begin{equation}
    \frac{\partial z_2}{\partial a_1}=\delta_3=W_2
\end{equation}
\begin{equation}
    \frac{\partial z_2}{\partial W_2}=\delta^\prime_3=a^T_1
\end{equation}
\begin{equation}
    \frac{\partial z_1}{\partial W_1}=\delta_4= 
    \begin{cases}
    0&  z_1<0\\
    1& z_1\geq 0
\end{cases}
\end{equation}
\begin{equation}
    \frac{\partial z_1}{\partial W_1}=x^{(i)T}
\end{equation}
\begin{equation}
    \frac{\partial J}{\partial W_1}=\delta_6=-\frac{1}{m} \sum_i \delta_1^{(i)}\cdot \delta_2 \cdot (\delta_3 \odot \delta_4) \cdot \delta_5
\end{equation}
\begin{equation}
    \frac{\partial J}{\partial W_2}=\delta_7=-\frac{1}{m} \sum_i \delta_1^{(i)}\cdot \delta_2 \delta^\prime_3
\end{equation}
که در آن $\delta_1^{(i)}$ و $\delta_2$ اسکالر هستند و $\delta_3 , \delta_4 \in \mathbb{R}^{D_{w1} \times 1}$ و $\delta_5 \in \mathbb{R}^{1 \times D_x}$
\begin{equation}
    W_1^{(0)} = W_1 - \eta \cdot \frac{\partial J}{\partial W_1}
\end{equation}
\begin{equation}
    W_2^{(0)} = W_2 - \eta \cdot \frac{\partial J}{\partial W_2}
\end{equation}

(ه)

\begin{equation}
    J2 = - \sum_i (\alpha \cdot (1-y^{(i)}).log(1-\hat{y}^{(i)})+\beta \cdot y^{(i)} \cdot log(\hat{y}^{(i)})) + \frac{1}{2}||W_2||_2^2+\frac{1}{2}||W_1||_2^2
\end{equation}
\begin{equation}
    J1 = - \sum_i (\alpha \cdot (1-y^{(i)}).log(1-\hat{y}^{(i)})+\beta \cdot y^{(i)} \cdot log(\hat{y}^{(i)})) + \frac{1}{2}||W_2||_1+\frac{1}{2}||W_1||_1
\end{equation}
\begin{equation}
    W_1^{(0)} = W_1 -\eta \cdot (\frac{\partial j}{\partial W_1}+2 \cdot W_1)
\end{equation}

\pagebreak
\hrule
نام و نام خانوادگی: \hspace{8cm} شماره دانشجویی:
\hrule
\vspace{0.5cm}


شبکه عصبی زیر را در نظر بگیرید.  در این شبکه، گره‌های تک‌دایره‌ای نشان‌دهنده متغیرها (به عنوان مثال $x_1$ متغیر ورودی، $h_1$ متغیر میانی و $\hat{y}$ متغیر خروجی است.) و گره‌های دودایره‌ای نشان‌دهنده توابع هستند.

\begin{figure}[H]
\centering 
\includegraphics[width=0.7\textwidth]{questions/image.png}
\end{figure}


فرض کنید تابع هزینه $L_2$ به صورت $L(y, \hat{y})=||y-\hat{y}||^2_2$ تعریف شده است. اگر یک داده با مقدار $(x_1,x_2,x_3,x_4)=(1,2,-1,0)$ و برچسب واقعی $1$ داشته باشیم، از الگوریتم  $backpropagation$ برای محاسبه مشتق جزئی $L$ نسبت به $w_1$ استفاده کنید.

\begin{center}
    $f(x) = ReLU(x) - 2 ReLU(x-1) + 0.5ReLU(x+1)$\\
    $(w_1, w_2, w_3, w_4, w_5, w_6) = (1, -0.25, 0, 1, 2, -1)$
\end{center}
\pagebreak
    \textbf{پاسخ1:} 
    \begin{itemize}
        \item ابعاد فعال‌سازی: $126 \times 126 \times 64$
        \item تعداد پارامترها: $(3 \times 3 \times 3 + 1) \times 64 = 1792$
    \end{itemize}


        \textbf{پاسخ2:} 
    \begin{itemize}
        \item ابعاد فعال‌سازی: $128 \times 128 \times 16$
        \item تعداد پارامترها: $(5 \times 5 \times 3 + 1) \times 16 = 1216$
    \end{itemize}

        \textbf{پاسخ3:} 
    \begin{itemize}
        \item ابعاد فعال‌سازی: $64 \times 64 \times 32$
        \item تعداد پارامترها: $(2 \times 2 \times 3 + 1) \times 32 = 416$
    \end{itemize}


        \textbf{پاسخ4:} 
    \begin{itemize}
        \item ابعاد فعال‌سازی بعد از سه ماژول: $14 \times 14 \times 64$
        \item تعداد پارامترها: $(3 \times 3 \times 3 + 1) \times 64 + (3 \times 3 \times 64 + 1) \times 64 \times 2 = 75,648$
    \end{itemize}

    \textbf{پاسخ5:} 
    \begin{itemize}
        \item ابعاد فعال‌سازی بعد از سه ماژول: $16 \times 16 \times 16$
        \item تعداد پارامترها: $(5 \times 5 \times 3 + 1) \times 16 + (5 \times 5 \times 16 + 1) \times 16 \times 2 = 14,048$
    \end{itemize}



    
    \textbf{پاسخ6:} 
    \begin{itemize}
        \item ابعاد فعال‌سازی بعد از سه ماژول: $2 \times 2 \times 32$
        \item تعداد پارامترها: $(2 \times 2 \times 3 + 1) \times 32 + (2 \times 2 \times 32 + 1) \times 32 \times 2 = 8672$
    \end{itemize}

        \textbf{پاسخ7:} 
    \begin{itemize}
        \item کمترین تعداد پارامترها مربوط به معماری قسمت ۶ است، با ابعاد فعال‌سازی $2 \times 2 \times 32$.
        \item تعداد پارامترها: $(2 \times 2 \times 32 + 1) \times 10 = 1290$
    \end{itemize}
\pagebreak
\hrule
نام و نام خانوادگی: \hspace{8cm} شماره دانشجویی:
\hrule
\vspace{0.5cm}


\begin{itemize}
    \item[الف)]
ما به یک مسئله رگرسیون خطی تک‌بعدی خاص علاقه‌مند هستیم. داده‌های مربوط به این مسئله شامل \(n\) نمونه \((x_1, y_1), \dots, (x_n, y_n)\) هستند که \(x_i\) و \(y_i\) اعداد حقیقی برای همه \(i\) می‌باشند. فرض کنید \( \mathbf{w}^* = [w_0^*, w_1^*]^T \) جواب حداقل مربعات باشد که به دنبال آن هستیم. به عبارت دیگر، \( \mathbf{w}^* \) کمینه‌کننده عبارت زیر است:
\[
J(\mathbf{w}) = \frac{1}{n} \sum_{i=1}^n (y_i - w_0 - w_1 x_i)^2.
\]

می‌توانید فرض کنید که برای مقاصد ما، جواب یکتا است.

هر کدام از گزاره‌های زیر که اگر 
\( \mathbf{w}^* = [w_0^*, w_1^*]^T \)
پاسخ کمترین مربعات باشد، درست است  مشخص کنید و کامل توضیح دهید.

\[
\frac{1}{n} \sum_{i=1}^n (y_i - w_0^* - w_1^* x_i)y_i = 0 \quad 
\]
\[
\frac{1}{n} \sum_{i=1}^n (y_i - w_0^* - w_1^* x_i)(y_i - \bar{y}) = 0 \quad 
\]
\[
\frac{1}{n} \sum_{i=1}^n (y_i - w_0^* - w_1^* x_i)(x_i - \bar{x}) = 0 \quad 
\]
\[
\frac{1}{n} \sum_{i=1}^n (y_i - w_0^* - w_1^* x_i)(w_0^* + w_1^* x_i) = 0 \quad 
\]

که در آن \(\bar{x}\) و \(\bar{y}\) میانگین نمونه‌ها بر اساس مجموعه داده هستند.


\item[ب)]
\\
چندین آماره از داده‌ها وجود دارند که می‌توانیم برای تخمین \( \mathbf{w}^* \) استفاده کنیم. این اعداد عبارتند از:
\[
\bar{x} = \frac{1}{n} \sum_{i=1}^n x_i \quad 
\]
\[
\bar{y} = \frac{1}{n} \sum_{i=1}^n y_i \quad 
\]
\[
C_{xx} = \frac{1}{n} \sum_{i=1}^n (x_i - \bar{x})^2 \quad 
\]
\[
C_{xy} = \frac{1}{n} \sum_{i=1}^n (x_i - \bar{x})(y_i - \bar{y}) \quad 
\]
\[
C_{yy} = \frac{1}{n} \sum_{i=1}^n (y_i - \bar{y})^2 \quad
\]

فرض کنید فقط به مقدار \(w_1^*\) اهمیت می‌دهیم. می‌خواهیم \(w_1^*\)  را بر اساس تنها دو آماره از آماره‌های بالا تعیین کنیم. کدام دو آماره برای این کار لازم است؟ توضیح دهید.
\end{itemize}
\pagebreak

آ) می‌توانیم مقادیر  را به صورت زیر بازنویسی کنیم:

\begin{align*}
& y_1 = \frac{x_1+x_2}{2} + \frac{|x_1-x_2|}{2} \\
& y_2 = \frac{x_1+x_2}{2} - \frac{|x_1-x_2|}{2}
\end{align*}

یعنی داریم:

$$
\rightarrow \begin{cases}
x_1 \ge x_2 \rightarrow \begin{cases} y_1 =  \frac{x_1+x_2}{2} + \frac{x_1-x_2}{2} = x_1 = \max(x_1, x_2) \\
y_2 = \frac{x_1+x_2}{2} - \frac{x_1 - x_2}{2} = x_2 = \min(x_1, x_2) \end{cases}\\

x_1 < x_2 \rightarrow \begin{cases} y_1 =  \frac{x_1+x_2}{2} - \frac{x_1-x_2}{2} = x_2 = \max(x_1, x_2) \\
y_2 = \frac{x_1+x_2}{2} + \frac{x_1 - x_2}{2} = x_1 = \min(x_1, x_2) \end{cases}
\end{cases}$$

در این صورت می‌توان گفت:

\begin{align*}
W^{(1)} &= \begin{bmatrix}1 & 1 \\ 1 & -1\end{bmatrix}
& b^{(1)} &= \begin{bmatrix}0 \\ 0\end{bmatrix}
& \phi^{(1)}(x) &= \begin{bmatrix}x_1 \\ |x_2|\end{bmatrix} \\
W^{(2)} &= \begin{bmatrix} 0.5 & 0.5 \\ 0.5 & -0.5\end{bmatrix}
&b^{(2)} &= \begin{bmatrix}0 \\ 0\end{bmatrix}
&\phi^{(2)}(x) &= \begin{bmatrix}x_1 \\ x_2\end{bmatrix}
\end{align*}

ب) با استفاده از ماژول پیاده‌سازی شده در بخش پیش داریم:

$$
\begin{bmatrix}
y_1 \\ y_2 \ \\ y_3 \ \\ y_4 \ 
\end{bmatrix} =
\begin{bmatrix}
\max \left(\max(x_1, x_2),~\max(x_3, x_4) \right)\\

\max \left(\min \left(\max(x_1, x_2),~\max(x_3, x_4) \right),~\max \left(\min(x_1, x_2),~\min(x_3, x_4) \right) \right)\\

\min \left(\min \left(\max(x_1, x_2),~\max(x_3, x_4) \right),~\max \left(\min(x_1, x_2),~\min(x_3, x_4) \right) \right)\\

\min \left(\min(x_1, x_2),~\min(x_3, x_4) \right)\\

\end{bmatrix} 
$$
\pagebreak
\hrule
نام و نام خانوادگی: \hspace{8cm} شماره دانشجویی:
\hrule
\vspace{0.5cm}
در هر بخش، درستی یا نادرستی گزینه را مشخص کرده و کامل توضیح دهید.
\begin{itemize}

    \item[الف)]
    در هنگام آموزش هرگونه مدل روی هر نوع مساله، از زمانی که تعداد پارامترها متناسب با تعداد ورودی‌ها شود به حداقل خطای تست می‌رسیم و پس از آن با افزایش تعداد پارامترها همیشه شاهد روند افزایشی خطای تست خواهیم بود.
    \item[ب)]
    معماری RNN سریع‌تر از Transformer است ولی دقت آن نسبت به Transformer کم‌تر است.

    \item[ج)]
    ابعاد فضای Embedding متن و تصویر در CLIP می‌تواند متفاوت باشد.
    
    \item[د)]
در یادگیری انتقالی با \lr{CNN}، زمانی که توزیع داده‌های هدف مشابه توزیع داده‌های پیش‌آموزش باشد، معمولاً چند لایه‌ی \lr{Fully Connected} آخر جایگزین شده و دوباره آموزش داده می‌شود، در حالی که وزن لایه‌های قبلی ثابت می‌مانند.

\item[ه)]
اگر یک مدل شبکه‌ی عصبی ۳ لایه برای رسیدن به دقت ۹۰ درصد روی یک مساله‌ی ورودی نیاز به ۱ میلیارد پارامتر داشته باشد، یک شبکه‌ی عصبی ۱۰ لایه برای رسیدن به همان سطح دقت روی همان مساله نیاز به تقریبا همان تعداد پارامتر دارد که البته عرض هر لایه کاهش می‌یابد.

\item[و)]
اگر از یک میلیارد تصویر داده شده، فقط یک میلیون تصویر برچسب داشته باشد برای خوشه‌بندی (Clustering) از کل یک میلیارد تصویر می‌توان استفاده کرد ولی برای دسته‌بندی (Classification) صرفا می‌توانیم از یک میلیون تصویر برچسب‌دار استفاده کنیم.
\item[ز)]
برای یک CNN با ابعاد کرنل $5\times 5$ و استراید ۱ و روی تصاویر $1000\times 1000$ داشتن سه لایه برای استخراج ویژگی‌های سراسری از تصویر کافی است.
\item[ح)]
استفاده از Patch ها به صورت انکد شده به عنوان ورودی ترانسفورمر ViT کافی نیست و حتما باید از 
\lr{Positional Embedding} استفاده کنیم.
\end{itemize}

\pagebreak
\textbf{پاسخ}


\begin{center}
\setLR
\begin{tabular}{|l|l|l|}
\hline
\lr{Layer} & \lr{Activation Volume Dimensions} & \lr{Number of parameters} \\
\hline
\lr{Input} & $32 \times 32 \times 3$ & $0$ \\
\hline
\lr{CONV3-8} & $32 \times 32 \times 8$ & $8 \times (3 \times 3 \times 3+1)=224$ \\
\hline
\lr{Leaky ReLU} & $32 \times 32 \times 8$ & $0$ \\
\hline
\lr{POOL-2} & $16 \times 16 \times 8$ & $0$ \\
\hline
\lr{BATCHNORM} & $16 \times 16 \times 8$ & $2 \times 8$ \\
\hline
\lr{CONV3-16} & $16 \times 16 \times 16$ & $16 \times (3 \times 3 \times 8+1)=1168$ \\
\hline
\lr{Leaky ReLU} & $16 \times 16 \times 16$ & $0$ \\
\hline
\lr{POOL-2} & $8 \times 8 \times 16$ & $0$ \\
\hline
\lr{FLATTEN} & 1024 & $0$ \\
\hline
\lr{FC-10} & $10$ & $(8 \times 8 \times 16+1) \times 10=10250$ \\
\hline
\end{tabular}
\end{center}








\end{document}
