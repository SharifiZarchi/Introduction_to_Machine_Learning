
\vspace{5cm}
پاسخ سوال ۱. گزینه ۴:‌الف نادرست و ب درست می‌باشد.

مدل‌های خودرمز‌گذار پراکنده، بدون نیاز به regularization خاصی شامل گلوگاه اطلاعاتی می‌باشند. گزاره ب نیز تعریف این دسته از مدل‌ها می‌باشد.

\vspace{1cm}

پاسخ سوال ۲. گزینه ۳:‌ هردو گزاره درست هستند.

در گزاره اول، با کاهش تعداد نورون‌ها می‌توان میزان اطلاعات در مدل خودرمز‌گذار را کاهش و درنتیجه آن را به مدل خودرمزگذار ناقص تبدیل کرد. در گزاره دوم، به صورت‌کلی مدل‌های خودرمزگذار را میتوان به‌عنوان یک ابزار کاهش ابعاد داده ( reduction dimensionality ) درنظر گرفت که برای این امر، مدل ابرصفحه‌هایی را برای تصویر کردن داده به ابعاد پایین‌تر را یاد می‌گیرد.

\vspace{1cm}

پاسخ سوال ۳. گزینه ۳: هردو گزاره درست هستند.

این مدل برای از بین بردن نیز در داده ورودی استفاده می‌شود، درنتیجه تابع هزینه زمان آموزش این مدل سعی می‌کند تفاوت میان نسخه اصلی داده ورودی و بازسازی داده براساس نسخه نویزی آن را کاهش دهد. خروجی بخش رمز‌گذار مدل (و ورودی بخش رمز‌گشا) ویژگی‌های استخراج‌شده از داده توسط مدل می‌باشند.