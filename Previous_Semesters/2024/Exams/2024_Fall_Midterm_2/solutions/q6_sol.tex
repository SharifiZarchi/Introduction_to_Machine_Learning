\vspace{1cm}

پاسخ سوال اول) گزینه‌های الف و ب

\vspace{1cm}

پاسخ سوال دوم) گزینه‌های الف و د




\vspace{1cm}
پاسخ سوال سوم)



الف نادرست و ب درست می‌باشد


مدل‌های خودرمز‌گذار پراکنده، بدون نیاز به regularization خاصی شامل گلوگاه اطلاعاتی می‌باشند. گزاره ب نیز تعریف این دسته از مدل‌ها می‌باشد.

\vspace{1cm}


ج درست و د درست می‌باشد




در گزاره اول، با کاهش تعداد نورون‌ها می‌توان میزان اطلاعات در مدل خودرمز‌گذار را کاهش و درنتیجه آن را به مدل خودرمزگذار ناقص تبدیل کرد. در گزاره دوم، به صورت‌کلی مدل‌های خودرمزگذار را میتوان به‌عنوان یک ابزار کاهش ابعاد داده ( reduction dimensionality ) درنظر گرفت که برای این امر، مدل ابرصفحه‌هایی را برای تصویر کردن داده به ابعاد پایین‌تر را یاد می‌گیرد.

\vspace{1cm}

ه درست  و درست می‌باشد




این مدل برای از بین بردن نیز در داده ورودی استفاده می‌شود، درنتیجه تابع هزینه زمان آموزش این مدل سعی می‌کند تفاوت میان نسخه اصلی داده ورودی و بازسازی داده براساس نسخه نویزی آن را کاهش دهد. خروجی بخش رمز‌گذار مدل (و ورودی بخش رمز‌گشا) ویژگی‌های استخراج‌شده از داده توسط مدل می‌باشند.