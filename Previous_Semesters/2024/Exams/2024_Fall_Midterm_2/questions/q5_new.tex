\textbf{سوال چهارگزینه ای(9 نمره)}



در هریک از سوال‌های زیر، با انتخاب گزینه مناسب گزاره‌های درست را انتخاب کنید.

\vspace{0.5cm}
۱.

الف) در مدل‌های خودرمز‌گذار پراکنده \footnote{autoncoder Sparse} ایجاد تعداد نورون‌ها در لایه نهان مدل \footnote{layer Hidden} باعث ایجاد گلوگاه اطلاعاتی \footnote{bottleneck Information} می‌شود.

\vspace{0.25cm}
ب) ایده کلی در مدل خودرمز‌گذار پراکنده این است که روند رمز‌گذاری و رمزگشایی تنها با فعال نگه‌داشتن تعداد کمی از نورون‌ها صورت بگیرد

\vspace{0.5cm}
۱)  الف درست و ب نادرست است.

\vspace{0.25cm}
۲) هردو نادرست هستند.

\vspace{0.25cm}
۳) هردو درست هستند.

\vspace{0.25cm}
۴)  الف نادرست و ب درست است.

\vspace{0.5cm}
۲.

الف) یک راه برای ایجاد مدل خودرمز‌گذار ناقص \footnote{autoencoder Sparse} این است که تعداد نورون‌های موجود در لایه‌های نهان \footnote{layers Hidden} را کاهش دهیم.

\vspace{0.25cm}
ب) مدل‌های خودرمزگذار قادر به یادگیری منیفولد‌های \footnote{Manifold} غیرخطی می‌باشند.

\vspace{0.5cm}
۱) الف درست و ب نادرست است.

\vspace{0.25cm}
۲) هردو نادرست هستند.

\vspace{0.25cm}
۳) هردو درست هستند.

\vspace{0.25cm}
۴) الف نادرست و ب درست است.

\vspace{0.5cm}
۳.

الف) در مدل خودرمزگذار بدون نویز \footnote{autoencoder Denoising}، تابع هزینه میان نسخه اصلی ورودی و بازسازی آن براساس نسخه نویزی آن، محاسبه می‌شود.

\vspace{0.25cm}
ب) از مدل خودرمزگذار بدون نویز میتوان به‌عنوان ابزاری برای استخراج ویژگی استفاده کرد.

\vspace{0.5cm}
۱) الف درست و ب نادرست است.

\vspace{0.25cm}
۲) هردو نادرست هستند.

\vspace{0.25cm}
۳) هردو درست هستند.

\vspace{0.25cm}
۴) الف نادرست و ب درست است.
