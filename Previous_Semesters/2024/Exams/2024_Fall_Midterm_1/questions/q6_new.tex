\textbf{اصلی}

در یک شرکت بزرگ کاریابی کار می‌کنید. هر فرد یک پروفایل دارد که بعضی ویژگی‌های آن (نظیر سن، آخرین حقوق) عدد پیوسته و بعضی ویژگی‌های دیگر (نظیر رشته‌ی تحصیلی) Categorical است. 
همچنین بعضی از ویژگی‌ها (نظیر آشنایی با هریک از زبان‌های برنامه‌نویسی) به صورت صفر و یک درج شده است. 

\begin{enumerate}
\item می‌خواهید برای هر کاربر جدید که پروفایل خود را تکمیل کرده‌است، یک مبلغ حقوق تخمین بزنید. از چه الگوریتمی استفاده می‌کنید؟ چه تغییری روی ویژگی‌ها می‌دهید؟ برای هریک از ویژگی‌ها از چه پیش‌پردازشی استفاده می‌کنید؟ جزئیات را توضیح دهید.
\vspace{5cm}
\item
حال می‌خواهید داده‌های این شرکت را به صورت یک نمودار نمایش دهید. برای این کار تصمیم دارید از تحلیل مولفه‌های اصلی (PCA) استفاده کنید.
آیا به نظر شما تغییر مقیاس ویژگی‌ها لازم است؟ اگر بله، چرا؟ و چطور این کار را انجام می‌دهید؟ اگر خیر، دلیل‌ شما چیست؟
\vspace{5cm}
\item
فرض کنید ماتریس کوواریانس زیر را داشته باشید. چطور از روی آن مولفه‌های اصلی را مشخص می‌کنید؟ محاسبات خود را بنویسید.
{\latin
\centering
$C=$\begin{bmatrix}
3 & -4\\
-4 & 3
\end{bmatrix}
}
\vspace{5cm}
\item
فرض کنید می‌خواهید با PCA ابعاد داده‌هاهای شرکت را با بردن آن‌ها فضای جدیدی کاهش دهید که عمده‌ی اطلاعات حفظ شود. چطور تعداد بعدهای فضای جدید را مشخص می‌کنید؟
\vspace{6cm}
\item
ثابت کنید تصویرکردن داده‌ها توسط بردارویژه‌‌ی با بزرگ‌ترین مقدار ویژه‌ی ماتریس کوواریانس، واریانس داده‌های تصویرشده را بیشینه می‌کند.
\vspace{6cm}
\item
به دل‌خواه یک مساله‌ی زیبای یادگیری ماشین بنویسید و آن را حل کنید. نمره‌ی این بخش به زیبایی و منحصر به فرد بودن مساله‌ و درستی راه‌حل شما اختصاص می‌یابد.

\end{enumerate}
